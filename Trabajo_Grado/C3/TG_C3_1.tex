\section{La integral de Itô.}

Ahora, ¿Por qué es necesario construir una nueva integral? Veamos el objetivo inicial, solucionar una ecuación diferencial que tiene cierto ruido:

\[
	\frac{dX}{dt} = b(t, X_t) + \sigma (t, X_t) \cdot W_t
\]

Note que el ruido se puede representar como el proceso estocástico $W_t$. Bajo experimentación, se interponen las siguientes condiciones sobre el ruido:

\begin{itemize}
	\item Dos variables del proceso $W_{t_1}$ y $W_{t_2}$ con $t_1 \neq t_2$ son independientes.
	\item $\{ W_t \}$ es un proceso estacionario.
	\item $\mathbb{E}[W_t] = 0$ para todo $t$.
\end{itemize} 

No hay algún proceso estocástico tradicional que cumpla las condiciones dadas. Por ende, lo podemos ver como un proceso estocástico generalizado, un \textbf{proceso de ruido blanco}, esto es, un proceso que se puede construir como medida de probabilidad en cierto espacio sútil de funcionales $\mathcal{C}[ 0, \infty )$. Por ende, se nos sugiere que el proceso $\{W_t\}$ será el movimiento Browniano. 

Ahora, la ecuación diferencial estocástica se puede representar forma integral, usando el hecho que $W_t$ será el movimiento Browniano, como:

\[
	X_t = X_0 + \int_0^t b(s, X_s) ds + \int_0^t  \sigma (s, X_s) \cdot B_s
\]

Sin embargo, el inconveniente se presenta con la segunda integral de la derecha, cuyo integrador no es de variación acotada, y por ende, no existe la \textit{integral de Riemann-Stieljes } en este caso:


\[
	\int_0^t f(s,w) dB_s(w)
\]

Por ende, se debe realizar una construcción nueva. La idea será análoga a construir la integral de Lebesgue (Consulte \cite{Measure_Theory_DC}). Esto es, definir la integral para un conjunto de funciones simples, luego tomar sucesión de funciones simples uniformemente convergente a una función positiva (Donde ya se conoce que existe esta sucesión), y luego ver el caso para funciones en general.\\

Sea $f$ de la forma:

\[
	\phi (t, w) = \sum_{j \geq 0} e_j (w) \chi_{ [j \cdot 2^{-n}, (j+1)\cdot 2^{-n}  ] }(t)
\]
para $n \in \mathbb{N}$. Entonces, se define la integral en este caso como:

\[
	\int_S^T \phi(t,w) dB_t(w) = \sum_{j \geq 0} e_j (w) [B_{t_{j+1}} - B_{t_j}] (t)
\]

donde

\[
	t_k = t_k^{(n)} = \left\{  \begin{array}{lr}
		k \cdot 2^{-n} & \text{ Si } S \leq k \cdot 2^{-n} \leq T \\
		S & \text{ Si } k \cdot 2^{-n} < S \\
		T & \text{ Si } T < k \cdot 2^{-n}
	\end{array} \right.
\]

pero, si no hay condiciones adicionales sobre $e_j$, habrán ciertos problemas, como los presentados en el siguiente ejemplo.

\textbf{Ejemplo:} Suponga, que tenemos:

\[
	\phi_1(t,w)  = \sum_{j \geq 0} B_{j \cdot 2^{-n} } (w) \cdot \chi_{ [j \cdot 2^{-n}, (j+1) \cdot 2^{-n} ]  } (t)
\]

y

\[
	\phi_2(t,w)  = \sum_{j \geq 0} B_{(j+1) \cdot 2^{-n} } (w) \cdot \chi_{ [j \cdot 2^{-n}, (j+1) \cdot 2^{-n} ]  } (t)
\]

Entonces

\[
\mathbb{E}\bigg[\int\limits_{0}^{T}\phi_{1}(t,\omega)\,\mathrm{d}B_{t}(\omega)\bigg] = \sum_{j \geq 0}\mathbb{E}[B_{t_{j}}(B_{t_{j+1}}-B_{t_{j}})] = 0\,,
\]

dado que $\{B_{t}\}$ tiene incrementos independientes. Pero

\begin{align*}
\mathbb{E}\bigg[\int\limits_{0}^{T}\phi_{2}(t,\omega)\,\mathrm{d}B_{t}(\omega)\bigg] &= \sum_{j\geq 0}\mathbb{E}[B_{t_{j+1}}\cdot(B_{t_{j+1}}-B_{t_{j}})] \\
&= \sum_{j\geq 0}\mathbb{E}[(B_{t_{j+1}}-B_{t_{j}})^{2}] = T
\end{align*}

Por ser la varianza del movimiento Browniano. Entonces, aunque ambas son muy buenas aproximaciones a 

\[
	f(t, \omega) = B_t (\omega)
\]

las integrales no coinciden, y casi ni se parecen.

\begin{flushright}
	$\Box$
\end{flushright}

El problema al considerar las sumas de Riemann usando como integrador el movimiento Browniano, es la regularidad de la trayectoría que tendrá variaciones muy grandes respecto  las trayectorias. Además, el movimiento Browniano no es diferenciable en casi todo punto. Por ende, al tomar el límite, en el sentido de la probabilidad, de:

\[
	\sum_{j} f(t_j^*, \omega) [B_{t_{j+1}} - B_{t_j}](\omega) \rightarrow \int_S^T f(t, \omega) dB_t (\omega) 
\]

mientras $n \mapsto \infty$. Además, el punto seleccionado $t_j^*$ afectará. Al tomar el punto izquierdo $t_j^* =t_j$, se tiene la \textbf{integral de Itô}:

\[
	\int_S^T f(t, \omega) dB_t (\omega) 
\]

y al tomar el punto medio $t_j^* = \frac{t_j + t_{j + 1}}{2}$, se tiene la \textbf{integral de Stratonovich}:

\[
	\int_S^T f(t, \omega) \circ dB_t (\omega) 
\]

Se denotará, durante este capítulo a $\mathcal{F}_t$ como la menor $\sigma$-álgebra generada por $\left\{ B_s (\omega) \right\}_{s \leq t}$


\begin{boxDef}
	Sea $\mathcal{V} = \mathcal{V}(S,T)$ la clase de funciones:

	\[
		f(t, \omega): [0, \infty) \times \Omega \mapsto \mathbb{R}	
	\]

	tal que:

	\begin{enumerate}
		\item $(t, \omega) \mapsto f(t, \omega)$ es $\mathcal{B} \times \mathcal{F}$-medible, con $\mathcal{B}$ siendo la $\sigma$-álgebra de Borel en $[0, \infty)$
		\item $f(t, \omega)$ es $\mathcal{F}_t$-adaptada.
		\item $\mathbb{E}\left[ \int_S^T f^2 (t,\omega) dt \right] < \infty$ 
	\end{enumerate}

\end{boxDef}

La idea, es definir la integral de Itô para este tipo de funciones. Se va a omitir la construcción rigurosa de la integral, los interesados pueden consultar \cite{EDE_Oksendal}.

% ¿Construir integral de Itô? PREGUNTAR!!

\begin{boxDef}
	Sea \( f \in \mathcal{V}(S, T) \). La integral de Itō de \( f \) (de \( S \) a \( T \)) se define como:
	\[
	\int\limits_{S}^{T} f(t, \omega) \, dB_{t}(\omega) = \lim_{n \to \infty} \int\limits_{S}^{T} \phi_{n}(t, \omega) \, dB_{t}(\omega) \qquad (\text{límite en } L^{2}(P)),
	\]
	donde \( \{\phi_{n}\} \) es una sucesión de funciones elementales tal que:
	\[
	E\left[ \int\limits_{S}^{T} \big(f(t, \omega) - \phi_{n}(t, \omega)\big)^{2} \, dt \right] \to 0 \qquad \text{cuando } n \to \infty.
	\]	
\end{boxDef}
	
Una importante propiedad corresponde a la isometría de Itô:

\begin{coro}[Isometría de Itô.]
	\[
		\mathbb{E} \left[ \left( \int_S^T f(t, \omega) dB_t \right)^2 \right] = \mathbb{E} \left[ \int_S^T f^2 (t,\omega) dt \right]
	\]
\end{coro}

\textbf{Ejemplo:} En \cite{EDE_Oksendal} se demuestra que:

\[
	\int_0^t B_s dB_s = \frac{1}{2} B^2_t - \frac{1}{2}t
\]

\begin{flushright}
	$\Box$
\end{flushright}

También, se tienen las siguientes propiedades:

\begin{theorem}[Propiedades de la integral de Itô.]
Sea $ f, g \in \mathcal{V}(0, T) $ y  $ 0 \leq S < U < T $. Entonces:

\begin{enumerate}
    \item[(i)] \[ 
        \int_{S}^{T} f \, dB_t = \int_{S}^{U} f \, dB_t + \int_{U}^{T} f \, dB_t \quad \text{para casi toda } \omega.
    \]
    
    \item[(ii)] \[
        \int_{S}^{T} (c f + g) \, dB_t = c \cdot \int_{S}^{T} f \, dB_t + \int_{S}^{T} g \, dB_t \quad (c \text{ constante}) \quad \text{para casi toda } \omega.
    \]
    
    \item[(iii)] \[
        E \left[ \int_{S}^{T} f \, dB_t \right] = 0.
    \]
    
    \item[(iv)] \[
        \int_{S}^{T} f \, dB_t \text{ es } \mathcal{F}_T\text{-medible}.
    \]
\end{enumerate}
\end{theorem}

