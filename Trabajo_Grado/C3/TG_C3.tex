\chapter{La Integral de Itô}

En este capítulo, comenzaremos la construcción de la integral de Itô. Para cumplir este objetivo, usaremos fuertemente los hechos vistos en el capítulo anterior del movimiento Browniano. 

Ahora, ¿Por qué es necesario construir una nueva integral? Veamos el objetivo inicial, solucionar una ecuación diferencial que tiene cierto ruido:

\[
	\frac{dX}{dt} = b(t, X_t) + \sigma (t, X_t) \cdot W_t
\]

Note que el ruido se puede representar como el proceso estocástico $W_t$. Bajo experimentación, se interponen las siguientes condiciones sobre el ruido:

\begin{itemize}
	\item Dos variables del proceso $W_{t_1}$ y $W_{t_2}$ con $t_1 \neq t_2$ son independientes.
	\item $\{ W_t \}$ es un proceso estacionario.
	\item $\mathbb{E}[W_t] = 0$ para todo $t$.
\end{itemize} 

No hay algún proceso estocástico tradicional que cumpla las condiciones dadas. Por ende, lo podemos ver como un proceso estocástico generalizado, un \textbf{proceso de ruido blanco}, esto es, un proceso que se puede construir como medida de probabilidad en cierto espacio sútil de funcionales $\mathcal{C}[ 0, \infty )$. 

Por ende, se nos sugiere que el proceso $\{W_t\}$ será el movimiento Browniano. Discretizando la ecuación inicial...

\[
	\int_0^t f(s,w) dB_s(w)
\]

%  (Oksendal 3.11, Kallianpur 1980 p.10