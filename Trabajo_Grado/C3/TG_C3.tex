\chapter{Ecuaciones diferenciales estocásticas: Enfoque de Itô.}

En este capítulo, se comenzará la construcción de la integral de Itô. Para cumplir este objetivo, usaremos fuertemente los hechos vistos en el capítulo anterior del movimiento Browniano. Pero antes, se verán algunos ejemplos de ecuaciones diferenciales estocásticas, que poseen cierto ruido.

\begin{enumerate}
	\item Se puede considerar un modelo de crecimiento poblacional, sujeto a un ruido:

	\[
		\frac{dN}{dt} = (r(t) + \textit{Ruido})N(t)
	\] 
	con la condición inicial $N(0) = N_0$. El término $(r(t) + \textit{Ruido})$ corresponde a la tasa de crecimiento, esta puede variar en función de una distribución de probabilidad.

	\item Otro problema conocido, es el \textbf{problema de filtrado}. Suponga que se hacen observaciones $Z(s)$ de, por ejemplo, la carga electrica en cierto medio, cuya función es denotada por $Q(s)$, para $s \leq t$. Se puede ver la relación entre ambas funciones como:

	\[
		Z(s) = Q(s) + \textit{Ruido}	
	\]

	donde el ruido proviene de errores al realizar las mediciones. Realmente, uno obtiene una versión peturbada de $Q(s)$. El problema consiste en determinar: \textit{¿Cuál es la mejor estimación de} $Q(t)$ \textit{que cumple la relación dada, basado en las observaciones} $Z(s)$ \textit{?}. En otras palabras, se desea filtrar el ruido, y obtener la medición más precisa. Este problema se puede plantear como una ecuación diferencial lineal con ruido.

	\item En matemáticas financiera, es muy común el uso de estas ecuaciones diferenciales para modelar diferentes fenómenos financieros. Por ejemplo, suponga que una persona tiene dos posibles inversiones:

	\begin{enumerate}
		\item Una inversión riesgosa, donde el precio $p_1 (t)$ satisface la ecuación diferencial estocástica (Modelo poblacional estocástico):

		\[
			\frac{dp_1}{dt} = (a + \alpha \cdot \textit{Ruido})p_1
		\]
		donde $a > 0$ y $\alpha \in \mathbb{R}$.

		\item Una inversión segura, donde el precio $p_2 (t)$ crece exponencialmente:

		\[
			\frac{dp_2}{dt} = bp_2	
		\]

		donde $0 < b < a$.

	\end{enumerate}

	La persona, en cada instante $t$ desea saber qué porcentaje $u_t$ de su fortuna, sea $X_t$, colocar en la inversión riesgosa. Así, coloca $(1 - u_t)X_t$ de su fortuna en la inversión segura. Dada una \textit{función de utilidad} $U$ y un tiempo final $T$, se desea hallar la \textit{cartera} óptima $u_t \in [0, 1]$ que maximizar la utilidad de la fortuna terminal, esto es:

	\[
		\max_{0 \leq u_t \leq 1} \left\{  \mathbb{E}[ U(X_T^{(n)}) ] \right\}
	\]

	Si se supone que la inversión riesgosa, es una \textit{opción de llamada Europea}, se deriva la famosa ecuación de \textbf{Black-Scholes}.


\end{enumerate}


Con estos ejemplos, ya se sabe el por qué es necesario, en algunos casos, recurrir a estas ecuaciones diferenciales con ruido. En este capítulo, siguiendo el tratamiento de \cite{EDE_Oksendal}, se verá un enfoque \textit{probabilístico} para solucionar estas ecuaciones, y en el próximo capítulo, se verán algunas desventajas de usar este método, proponiendo un método más analítico. Se enunciarán algunos resultados, y se suprimirán algunas demostraciones. El lector interesado, puede consultar la fuente ya citada.

% ========================
% ======= SECCIÓN 1: La integral de Itô =====
% ========================

\section{La integral de Itô.}

Ahora, ¿Por qué es necesario construir una nueva integral? Veamos el objetivo inicial, solucionar una ecuación diferencial que tiene cierto ruido:

\[
	\frac{dX}{dt} = b(t, X_t) + \sigma (t, X_t) \cdot W_t
\]

Note que el ruido se puede representar como el proceso estocástico $W_t$. Bajo experimentación, se interponen las siguientes condiciones sobre el ruido:

\begin{itemize}
	\item Dos variables del proceso $W_{t_1}$ y $W_{t_2}$ con $t_1 \neq t_2$ son independientes.
	\item $\{ W_t \}$ es un proceso estacionario.
	\item $\mathbb{E}[W_t] = 0$ para todo $t$.
\end{itemize} 

No hay algún proceso estocástico tradicional que cumpla las condiciones dadas. Por ende, lo podemos ver como un proceso estocástico generalizado, un \textbf{proceso de ruido blanco}, esto es, un proceso que se puede construir como medida de probabilidad en cierto espacio sútil de funcionales $\mathcal{C}[ 0, \infty )$. Por ende, se nos sugiere que el proceso $\{W_t\}$ será el movimiento Browniano. 

Ahora, la ecuación diferencial estocástica se puede representar forma integral, usando el hecho que $W_t$ será el movimiento Browniano, como:

\[
	X_t = X_0 + \int_0^t b(s, X_s) ds + \int_0^t  \sigma (s, X_s) \cdot B_s
\]

Sin embargo, el inconveniente se presenta con la segunda integral de la derecha, cuyo integrador no es de variación acotada, y por ende, no existe la \textit{integral de Riemann-Stieljes } en este caso:


\[
	\int_0^t f(s,w) dB_s(w)
\]

Por ende, se debe realizar una construcción nueva. La idea será análoga a construir la integral de Lebesgue (Consulte \cite{Measure_Theory_DC}). Esto es, definir la integral para un conjunto de funciones simples, luego tomar sucesión de funciones simples uniformemente convergente a una función positiva (Donde ya se conoce que existe esta sucesión), y luego ver el caso para funciones en general.\\

Sea $f$ de la forma:

\[
	\phi (t, w) = \sum_{j \geq 0} e_j (w) \chi_{ [j \cdot 2^{-n}, (j+1)\cdot 2^{-n}  ] }(t)
\]
para $n \in \mathbb{N}$. Entonces, se define la integral en este caso como:

\[
	\int_S^T \phi(t,w) dB_t(w) = \sum_{j \geq 0} e_j (w) [B_{t_{j+1}} - B_{t_j}] (t)
\]

donde

\[
	t_k = t_k^{(n)} = \left\{  \begin{array}{lr}
		k \cdot 2^{-n} & \text{ Si } S \leq k \cdot 2^{-n} \leq T \\
		S & \text{ Si } k \cdot 2^{-n} < S \\
		T & \text{ Si } T < k \cdot 2^{-n}
	\end{array} \right.
\]

pero, si no hay condiciones adicionales sobre $e_j$, habrán ciertos problemas. Suponga, que tenemos:

\[
	\phi_1(t,w)  = \sum_{j \geq 0} B_{j \cdot 2^{-n} } (w) \cdot \chi_{ [j \cdot 2^{-n}, (j+1) \cdot 2^{-n} ]  } (t)
\]

y

\[
	\phi_2(t,w)  = \sum_{j \geq 0} B_{(j+1) \cdot 2^{-n} } (w) \cdot \chi_{ [j \cdot 2^{-n}, (j+1) \cdot 2^{-n} ]  } (t)
\]

% ========================
% ========================
% ========================









% ========================
% ======= SECCIÓN 2: Fórmula de Itô =====
% ========================


% ========================
% ========================
% ========================



\section{Fórmula de Itô.}

De la sección anterior, se puede intuir que las integrales no suelen ser muy útiles al evaluar, al tener que enfrentarse a numerosas cuentas que pueden resultar agobiantes. Además, no existe teoría de diferenciación, únicamente de integración.\\

Sin embargo, a pesar de este panorama tan desalentador, existe una fórmula bastante útil al evaluar integrales de Itô: \textit{fórmula de Itô}.

\begin{boxDef}
	Sea $B_t$ un movimiento Browniano (En $1$-dimensión), en $(\Omega, \mathcal{F}, \mathbb{P})$. Un \textbf{proceso de Itô} o integral estocástica (En 1-dimensión) es un proceso estocástico $X_t$ en $(\Omega, \mathcal{F}, \mathbb{P})$ de la forma:

	\[
		X_t = X_0 + \int_0^t u(s, \omega) ds + \int_0^t v(s, \omega) dB_s	
	\]

	donde

	\[
		\mathbb{P}\left[ \int_0^t v(s,\omega)^2 ds < \infty \text{ para todo } t \geq 0 \right] = 1
	\]

\end{boxDef}

También se puede escribir de forma diferencial:

\[
	dX_t = u dt + v dB_t
\]

\begin{theorem}[Fórmula de Itô en 1-dimensión.]
	Sea \( X_{t} \) un proceso de Itō dado por:
	\[
	dX_{t} = u \, dt + v \, dB_{t}
	\]
	Sea \( g(t,x) \in C^{2}([0,\infty) \times \mathbb{R}) \). Entonces:
	\[
	Y_{t} = g(t, X_{t})
	\]
	es también un proceso de Itō, y se cumple:
	\[
	dY_{t} = \frac{\partial g}{\partial t}(t, X_{t}) \, dt + \frac{\partial g}{\partial x}(t, X_{t}) \, dX_{t} + \frac{1}{2} \frac{\partial^{2} g}{\partial x^{2}}(t, X_{t}) \cdot (dX_{t})^{2}
	\]
	donde \( (dX_{t})^{2} = (dX_{t}) \cdot (dX_{t}) \) se calcula mediante las reglas:
	\[
	dt \cdot dt = dt \cdot dB_{t} = dB_{t} \cdot dt = 0, \quad dB_{t} \cdot dB_{t} = dt
	\]
\end{theorem}

\textbf{Ejemplo.} Suponga que se desea calcular:

\[
	\int_0^t s dB_s
\]

Se toma la función auxiliar $g(t, x) = tx$, de tal forma de obtener:

\[
	Y_t = g(t, B_t) = tB_t
\]

Aplicando la fórmula de Itô:

\[
	d(tB_t) = B_t dt + tdB_t
\]

o de forma integral:

\[
	\int_0^t s dB_s = tB_t - \int_0^t B_s ds
\]

\begin{flushright}
	$\Box$
\end{flushright}

Más generalmente, se tiene:

\begin{theorem}[Integración por partes.]
	Suponga que $f(s,\omega) = f(s)$ (Únicamente depende de $s$), y $f$ es continua, y de variación acotada en $[0,t]$. Luego:

	\[
		\int_0^t f(s) dB_s = f(t) B_t - \int_0^t B_s df_s
	\]
\end{theorem}

% ========================
% ======= SECCIÓN 3: Ecuaciones Diferenciales Estocásticas =====
% ========================


% ========================
% ========================
% ========================

\section{Ecuaciones diferenciales estocásticas.}

Ahora, ya se puede interpretar de una forma menos rigurosa a las ecuaciones diferenciales estocásticas:

\[
	\frac{dX_t}{dt} = b(t, X_t) + \sigma(t, X_t) W_t
\]

ya que, al escribir su forma integral, se podrá aplicar la integral de Itô.

\textbf{Ejemplo:} Considere el siguiente modelo de crecimiento poblacional:

\[
	\frac{dN_t}{dt} = r N_t + \alpha N_t W_t
\]

donde $W_t$ es un proceso de ruido blanco. Al escribir el problema de forma diferencial, se obtiene:

\[
	dN_t = r N_t d_t + \alpha N_t dB_t
\]

o

\[
	\frac{dN_t}{N_t} = r d_t + \alpha dB_t
\]

Al aplicar la fórmula de Itô, con la función $g(t,x)=\ln x$, se llega a la solución:

\[
	N_t = N_0 \exp \left( \left(r - \frac{1}{2} \alpha^2 \right)t +\alpha B_t  \right)
\]

Este es un proceso estocástico del tipo \textit{movimiento Browniano geométrico.}

\begin{flushright}
	$\Box$
\end{flushright}

Antes de finalizar el capítulo, se presenta el teorema de existencia y unicidad de soluciones (fuertes) de ecuaciones diferenciales estocásticas.

\begin{theorem}[Existencia y unicidad para ecuaciones diferenciales estocásticas.]
	
	Sea $ T > 0 $ y sean \( b(\cdot,\cdot): [0,T] \times \mathbb{R}^{n} \to \mathbb{R}^{n} \), \( \sigma(\cdot,\cdot): [0,T] \times \mathbb{R}^{n} \to \mathbb{R}^{n \times m} \) 
funciones medibles que satisfacen:

\begin{align*}
|b(t,x)| + |\sigma(t,x)| &\leq C(1 + |x|); & x \in \mathbb{R}^{n},\; t \in [0,T], \\
|b(t,x) - b(t,y)| + |\sigma(t,x) - \sigma(t,y)| &\leq D|x - y|; & x,y \in \mathbb{R}^{n},\; t \in [0,T]
\end{align*}

para constantes \( C, D > 0 \) (donde \( |\sigma|^2 = \sum |\sigma_{ij}|^2 \)). 

Sea \( Z \) una variable aleatoria independiente de la \( \sigma \)-álgebra \( \mathscr{F}_{\infty}^{(m)} \) 
generada por \( B_s(\cdot) \) con \( s \geq 0 \), y tal que:
\[
\mathbb{E}[|Z|^2] < \infty.
\]

Entonces, la ecuación diferencial estocástica:
\[
dX_t = b(t,X_t) dt + \sigma(t,X_t) dB_t, \quad 0 \leq t \leq T, \; X_0 = Z 
\]

tiene una única solución \( t \)-continua \( X_t(\omega) \) con la propiedad que $X_t (\omega)$ es adaptada a la filtración $\mathcal{F}^Z_t$ generada por $Z$ y por $B_s (\cdot)$; $s \leq t$, y además:

\[
	\mathbb{E} \left[ \int_0^T \lvert X_t \rvert^2 dt \right] < \infty
\]


\end{theorem}


Los interesados en la demostración, pueden consultar \cite{EDE_Oksendal}.


%  (Oksendal 3.11, Kallianpur 1980 p.10