\section{Construcción del Movimiento Browniano}

Primero, se da la definición de un movimiento Browniano, y luego, se hará la construcción. Esta, se puede hacer por tres métodos distintas (Consulte \cite{Brownian_Motion_Karatzas}):

\begin{enumerate}
	\item Teoremas de existencia y continuidad de Kolmogorov.
	\item Construcción propuesta por Levý, Wiener y Ciesielski, que usa fuertemente conceptos de espacios de Hilbert, aprovechando que el movimiento Browniano es Gaussiano. Usando funciones de Haar.
	\item Principo de invarianza de Donsker, donde se busca una convergencia débil a una medida de Wiener.
\end{enumerate}

En el presente trabajo, se enunciarán los teoremas de Kolmogorov, y en el apéndice se discutirán brevemente las demás construcciones.

\begin{boxDef}
	Dado $\{ W_t \}$ un proceso estocástico, en el espacio de probabilidad $(\Omega, \mathcal{F}, P)$. El proceso $\{ W_t \}$ es un \textbf{movimiento Browniano} en una dimensión, si se cumplen las siguientes condiciones:

	\begin{itemize}
		\item Para casi todo $\omega$, los caminos $W_t (\omega)$ son continuos (En el sentido de la probabilidad).
		\item $\{ W_t \}$ es un proceso Gaussiano, es decir, para $k \geq 1$, y todo $0 \leq t_1 \leq \cdots \leq t_k$, el vector aleatorio, $Z = (W_{t_1}, \cdots, W_{t_k}) \in \mathbb{R}^{n}$ tiene distribución multinormal (O Gaussiana), con media el vector $0$, y la matriz de covarianza como $B(t_i, t_j) = \mathbb{E}[W_{t_i} W_{t_j}] = \min(t_i, t_j)$.
	\end{itemize}

\end{boxDef}

Más general, podemos considerar el movimiento Browniano respecto a una filtración.

\begin{boxDef}
	Un proceso $W_t$ en un espacio de probabilidad $(\Omega,\mathcal{F}, P)$ adaptado a una filtración $(\mathcal{F}_t)_{t > 0}$ es un \textbf{movimiento Browniano relativo a la filtración} $\mathcal{F}_t$, si:

	\begin{itemize}
		\item Los caminos $W_t(\omega)$ son trayectorias continuas de $t$ para casi todo $\omega$.
		\item $W_0 (\omega) = 0$ para casi todo $\omega$.
		\item Para $0 \leq s \leq t$, los incrementos $W_t - W_s$ son variables aleatorias Faussianas con media $0$ y varianza $t - s$.
		\item Para $0 \leq s \leq t$, los incrementos $W_T - W_s$ son independientes de las $\sigma$-álgebras $\mathcal{F}_s$.
	\end{itemize}

\end{boxDef}

Una pregunta que natural, es acerca de la existencia y de la unicidad de este proceso. Kolmogorov propone dos teoremas para verificar que el proceso existe y es único (En el sentido de distribuciones). Antes, se define el concepto de familia \textit{consistente}.

\begin{boxDef}
	Sea $T$ un conjunto de sucesiones finitas $\tilde{t} = (t_1, \cdots, t_n)$ de números positivos distintos. Suponga que para cada $\tilde{t}$ de longitud $n$, existe una medida de probabilidad $Q_{\tilde{t}}$ en $(\mathbb{R}, \mathcal{B}(\mathbb{R}^n))$. Luego, $\left\{ Q_{\tilde{t}} \right\}_{\tilde{t} \in T}$ es una \textbf{familia de distribuciones finito-dimensionales}. La familia se dice \textbf{consistente} si cumple las siguientes condiciones:

	\begin{itemize}
		\item Si $\tilde{s} = (t_{i_1}, \cdots, t_{i_n})$ es una permutación de $\tilde{t} = (t_1, \cdots, t_n)$, luego para todo $A_i \in \mathcal{B}(\mathbb{R})$ con $i = 1, \cdots, n$ tenemos:

		\[
			Q_{\tilde{t}} (A_1 \times \cdots \times A_n) = Q_{\tilde{s}} (A_{i_1} \times \cdots \times A_{i_n})
		\]

		esto es, es invariante bajo permutaciones.

		\item Si $\tilde{t} = (t_1, \cdots, t_n)$ con $n \geq 1$, y $\tilde{s} = (t_1, \cdots, t_{n-1})$, con $A \in \mathcal{B}(\mathbb{R}^{n-1})$, luego:

		\[
			Q_{\tilde{t}} (A \times \mathbb{R}) = Q_{\tilde{s}} (A)
		\]
	\end{itemize}

\end{boxDef}

Note que, al tener un espacio de probabilidad, $(\mathbb{R}^{[0, \infty)}, \mathcal{B}(\mathbb{R}^{[0, \infty)} ), P )$, se puede definir una familia de distribuciones finito-dimensionales:

\[
	Q_{\tilde{t}} (A) = P[ \left\{ w \in \mathbb{R}^{ [0,\infty) } \vert ( w(t_1), \cdots, w(t_n) ) \in A  \right\} ]
\]

Para construir el movimiento Browniano, se usará el hecho, de tener las distribuciones finito-dimensaionales, para construir una medida de probabilidad en el espacio, como primer paso. Para ello, se usa el siguiente teorema:


\begin{theorem}[Teorema de consistencia de Kolmogorov y Daniell.]
	Sea $\left\{ Q_{ \tilde{t} } \right\}$ una familia de distribuciones finito-dimensonales consistentes. Así, existe una medida de probabilidad $P$ en $(\mathbb{R}^{ [0, \infty) }, \mathcal{B}(\mathbb{R}^{ [0, \infty) })  )$ tal que:

	\[
		Q_{\tilde{t}} (A) = P[ \left\{ w \in \mathbb{R}^{ [0,\infty) } | (w(t_1), w(t_2), \cdots, w(t_n)) \right\} ]
	\] 	
	se cumpla para todo $\tilde{t} = \left\{ t_1, t_2, \cdots, t_n \right\} \in T$.
\end{theorem}

Como consecuencia, se tiene el siguiente resultado:

\begin{coro}
	Existe una medida de probabilidad $P$ en $(\mathbb{R}^{ [0, \infty) }, \mathcal{B}(\mathbb{R}^{ [0, \infty) })  )$ tal que el proceso: 

	\[
		B_t (w) = w(t) \text{, } w \in \mathbb{R}^{[0, \infty)}, t \geq 0	
	\]
	tiene incrementos estacionarios e independientes. Además, los incrementos $B_t - B_s$, con $0 \leq s \leq t$, es normalmente distribuido con media $0$ y varianza $t - s$,

\end{coro}

Ya se tiene una medida de probabilidad. Ahora, se debe construir el proceso en todo el espacio $\mathbb{R}^{[0, \infty)}$. Sin embargo, hay que notar que $C[0, \infty) \notin \mathcal{B}( \mathbb{R}^{ [0, \infty) } )$, y por ende, se debe construir una modificación continua del proceso. Se tiene entonces, el segundo teorema de Kolmogorov.

\begin{theorem}[Teorema de continuidad de Kolmogorov y Chenstov.]
	Suponga que un proceso $X = \left\{  X_t \vert 0 \leq t \leq T \right\}$ en un espacio de probabilidad $(\Omega, \mathcal{F}, P)$ cumple la condición:

	\[
		\mathbb{E}[ \lvert X_t - X_s \rvert^{\alpha} ] \leq C \lvert t - s \rvert^{ 1 + \beta}
	\]
	con $0 \leq s, t \leq T$, para constantes positivas $\alpha$, $\beta$ y $C$. Entonces, existe una modificación $\tilde{X} = \left\{ \tilde{X_t} \vert  0 \leq t \leq T  \right\}$ de $X$, que es localmente Hölder-continua, con exponente $\gamma$, para todo $\gamma \in (0, \beta / \alpha)$, esto es:

	\[
		P \left[ w \vert \sup_{  0 < t - s < h(w)\text{, } s,t\in [0,T]}  \frac{\lvert  \tilde{X_t}(w) - \tilde{X_s}(w) \rvert }{\lvert t - s \rvert^{\gamma}}  \leq \delta \right] = 1
	\]

	donde $h(w)$ es una variable aleatoria positiva \textit{casi siempre}m y $\delta > 0$ una constante apropiada.

\end{theorem}

Como resultado final, tendremos:

\begin{coro}
	Hay una medida de probabilidad $P$ en $(\mathbb{R}^{ [0, \infty) }, \mathcal{B}(\mathbb{R}^{ [0, \infty) })  )$ y un proceso estocástico $W = \left\{ W_t, \mathcal{F}_t^W \vert t \leq 0 \right\}$ en el mismo espacio, tal que $W$ es un movimiento Browniano respecto a $P$.
\end{coro}

Como consecuencia, cada trayectoria del movimiento Browniano $\left\{ W_t(w) \vert 0 \leq t < \infty \right\}$ es localmente Hölder-continua con exponente $\gamma$, para $\gamma \in \left(0, \frac{1}{2}\right)$.