
\section{Miscelánea del Movimiento Browniano (Opcional)}

El movimiento Browniano satisface varias propiedades. Como

\begin{itemize}
	\item $\left\{ B_t \right\}$ es un proceso Gaussiano (Multinormal).
	\item $\mathbb{E}[B_t] = 0$ y $\mathbb{E}[B_t B_s] = \min\{ s, t \}$
	\item Para casi todo $\omega \in \Omega$, $t \mapsto B_t (\omega)$ es continua.
\end{itemize}

% https://www.stat.berkeley.edu/~pitman/s205s03/lecture15.pdf

Observe lo siguiente:

\begin{itemize}
	\item \textbf{Escalamiento: } Para todo $s > 0$, $\left\{ s^{-1/2} B_{st} \vert t \geq 0 \right\}$ es un movimiento Browniano comenzando en 0. Más aún, se cumple que:

	\[
		\left\{ B_{s,t} \vert t \geq 0 \right\}  \stackrel{d}{=}  \left\{ s^{1/2} B_t \vert t \geq 0 \right\}
	\]
\end{itemize}

