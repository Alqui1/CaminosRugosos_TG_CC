
\section{Miscelánea de Procesos Estocásticos (Opcional)}

\subsection{Procesos Markovianos.}



\subsection{Movimiento Browniano.}

El movimiento Browniano satisface varias propiedades. Como

\begin{itemize}
	\item $\left\{ B_t \right\}$ es un proceso Gaussiano (Multinormal).
	\item $\mathbb{E}[B_t] = 0$ y $\mathbb{E}[B_t B_s] = \min\{ s, t \}$
	\item Para casi todo $\omega \in \Omega$, $t \mapsto B_t (\omega)$ es continua.
\end{itemize}

% https://www.stat.berkeley.edu/~pitman/s205s03/lecture15.pdf

Observe lo siguiente:

\begin{itemize}
	\item \textbf{Escalamiento: } Para todo $s > 0$, $\left\{ s^{-1/2} B_{st} \vert t \geq 0 \right\}$ es un movimiento Browniano comenzando en 0. Más aún, se cumple que:

	\[
		\left\{ B_{s,t} \vert t \geq 0 \right\}  \stackrel{d}{=}  \left\{ s^{1/2} B_t \vert t \geq 0 \right\}
	\]
\end{itemize}

\subsection{Movimiento Browniano Fraccionario}.

% https://nms.kcl.ac.uk/martin.forde/FractionalBrownianMotion.pdf

Corresponde a una generalización del movimiento Browniano, este último es importante en modelamiento de volatilidad y trading óptimo, además, también en mecánica cuántica.

\begin{boxDef}
	Un proceso $X$ se dice \textbf{gaussiano} si para todo $t_1 < \cdots < t_n$, $X_{t_1}, \cdots, X_{t_n}$ tiene una distribución normal multivariada. Un proceso Gaussiano $W_t^H$ con media $0$, es un \textbf{movimiento Browniano fraccionario estándar} (fBM) con \textbf{exponente/parámero de Hurst} $H \in (0,1)$ si tiene la función de covarianza:

	\[
		\mathbb{E}( W_t^H W_s^H ) - \mathbb{E}(W_t^H) \mathbb{E}(W_s^H) = \frac{1}{2} (  \lvert t \rvert^{2H} + \lvert s \rvert^{2H} - \lvert t - s \rvert^{2H} )
	\]
\end{boxDef}


El fBM preserva propiedades de incrementos estacionarios, autosimilaridad (¿?) y distribuciones finito-dimensionales Gaussianas.
% https://nms.kcl.ac.uk/martin.forde/ConditionalLawLCGF.pdf

