\section{Algunos conceptos de topología}

Algunos temas de topología, necesaria para entender algunas pruebas y conceptos puntuales.


Ver concepto de base en espacios topológicos (Protter había una equivalencia entre espacios topológicos y medibles... ¿Cuál?)

\begin{boxDef}[Base contable]
	Un espacio $X$ se dice que tiene una \textbf{base contable en x}, si existe una colección contable $\mathcal{B}$ de vecindades de $x$ tal que para cada vecindad de $x$, contiene al menos uno de los elementos de $\mathcal{B}$ <- Primer axioma de numerabilidad.
\end{boxDef}

\begin{boxDef}[Espacio separable]

	Un espacio medible $(\Omega, \mathcal{A})$ se dice \textbf{separable} si $\mathcal{A}$ es generado por una colección contable de conjuntos. 

	\textbf{Otra definición!} (Munkres) Espacio X con un subconjunto denso contable. (Conjunto denso es que el conjunto de puntos de adherencia sea el espacio, $\bar{A} = X$)

\end{boxDef}

\begin{boxDef}[Espacio localmente compacto]

	Un espacio $E$ es un espacio \textbf{localmente compacto} si para todo $x \in E$, $x$ tiene un vecindario compacto.

\end{boxDef}





