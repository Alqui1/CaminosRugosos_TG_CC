\section{Temas de Probabilidad avanzados}

Algunos conceptos, teoremas, etc... de probabilidad, procesos estocásticas que pueden estar algo fuera de la tesis.

\subsection{Semimartingalas, Martingalas locales}

Conceptos necesarios para entender algunas definiciones acerca el movimiento Browniano. Asuma el espacio filtrado de probabilidad $(\Omega, \mathcal{F}, \mathcal{F}_t, P)$

\begin{boxDef}
	Un proceso $A$ es \textbf{creciente} o de \textbf{variación finita}, si es adaptado (A la filtración del espacio), y las trayectorias $t \mapsto A_t (\omega)$ son finitas, continuas a derecha y crecientes (O variación finita), para casi todo $\omega$
\end{boxDef} 


Recuerde conceptos de martingalas y tiempos de parada (Cuándo debo parar, información hasta el momento presente). Puede ver las martingalas locales, como un proceso estocástico que localmente es una martingala. 

\begin{boxDef}
	(Wikipedia) Un proceso $X$ es una \textbf{martingala local} si existe una sucesión de tiempos de parada $T_n$ con $T_n \nearrow \infty$ siempre, $T_n < T $ casi siempre en $\left\{ T > 0 \right\}$ y $\lim_{n \rightarrow \infty} T_n = T$ casi siempre.

	Otra definición! (Science Direct) $X$ es \textbf{martingala local} si existe una sucesión creciente $\left\{ T_n \right\}$ de tiempos de parada con $T_n \nearrow \infty$ a.s, tal que para cada $n$, $X^{T_n}$ es una martingala.

	(Stochastic Calculus, Capinski) Dado $\left\{ X_t \right\}$ un proceso adaptado a $\mathcal{F}_t$ es una \textbf{martingala local} si existe una sucesión $\left\{ \tau_n \right\}_{n \leq 1}$ de tiempos de parada, tal que para todo $\omega$ existe un $N(\omega)$ (Constante), tal que $n \leq N(\omega)$ implica $\tau_n (\omega) \leq T $ (casi siempre eventualmente), tal que para cada $n$, $X_{\tau_n}$ es una martingala respecto a $\mathcal{F}_t$
\end{boxDef}



\subsection{Variación Cuadrática}

\begin{boxDef}
	Dado $\{  X_t \}$ un proceso estocástico con valores reales definidos en un espacio de probabilidad $(\Omega, \mathcal{F}, P)$. Definimos la \textbf{variación cuadrática} como:

	\[
		\langle X \rangle_t = \lim_{ \lVert P \rVert \rightarrow 0 } \sum_{k = 1}^n (X_{t_k} - X_{t_{k-1}})^2	
	\]

\end{boxDef}







\subsection{Desigualdad de Burkholder-Davis-Gundy}

Usado en la demostración que el movimiento Browniano con la integral de Itô es un camino rugoso.

% 2

%  Daniel Revuz and Marc Yor. Continuous martingales and Brownian motion,

Primero, verificar la caracterización de Lévŷ para el movimiento Browniano.

\begin{theorem}[Lévy]
	Sea $X$ un proceso $d$-dimensional $\mathcal{F}_t$-adaptado y continuo, con $X_0 = 0$. Se tienen las siguientes equivalencias:

	\begin{itemize}

		\item $X$ es un movimiento Browniano respecto a $\mathcal{F}_t$.

		\item $X$ es una martingala local continua (¿?) y $\langle X^i, X^j \rangle_t = \delta_{i,j} t$.

		\item $X$ es una martingala local continua y para cada elección de funciones $f_1, \cdots, f_d \in L^2 (\mathbb{R}_+)$, el proceso

		\[
			\mathcal{E}_t = \exp \left( i \sum_{k = 1}^d \int_0^t f_k (s) dX^k_s + \frac{1}{2} \sum_{k=1}^d \sum_0^t f^2_k(s) ds \right)
		\]

		es una martingala compleja.
	
	\end{itemize}

\end{theorem}

\textbf{Demostración:}

\begin{flushright}
	$\Box$
\end{flushright}

Note que las normas $\lVert \cdot \rVert_1$ y $\lVert \cdot \rVert_2$ definido en el espacio de martingalas continuas acotadas en $L^2$, que se anulan en $0$, son equivalentes por la desigualdad de Doob's (Ver preliminares en PE):

\[
	\lVert M \rVert_1 = \mathbb{E}[M^2_{\infty}]^{1/2} = \mathbb{E}[ \langle  M, M \rangle_{\infty}]^{1/2} \qquad \lVert M \rVert_2 = \mathbb{E}[ (M^*_{\infty})^2 ]^{1/2}
\]

Con esto en mente, se enuncia la desigualdad de \textit{Burkholder-Davis-Gundy}:

\begin{theorem}[Desigualdad de Burkholder-Davis-Gundy] 
Para cualquier $p > 0$, existen dos constantes $c_p$ y $C_p$ tal que para toda martingala local continua $M$ que se anula en $0$, tenemos:

\[
	c_p \mathbb{E}[ \langle M,M \rangle_{\infty}^{p/2} ] \leq \mathbb{E}[ (M^*_{\infty})^p ] \leq C_p \mathbb{E}[ \langle M,M \rangle_{\infty}^{p/2} ]
\]

\end{theorem}

45¿Qué es $M^*_{\infty}$ y $\langle M,M \rangle$?

% https://math.uchicago.edu/~may/REU2019/REUPapers/Carlstein.pdf



% =========================================

\subsection{Área de Lévy}

Proceso estocástico que describe el área encerrada entre una trayectoria del movimiento Browniano y su cuerda. Introducido por Paul Lévy en $1940$.

%  Lévy, Paul M. (1940). "Le Mouvement Brownien Plan". American Journal of Mathematics. 62 (1): 487–550. doi:10.2307/2371467. JSTOR 2371467. 

Tiene relación curiosda como con las soluciones por solitones de la ecuación KdV (Korteweg-De Vries), y la función zeta de Riemann.

Para $W = (W_s^1, W_s^2)_{s \leq 0}$ un movimiento Browniano en $2$ dimensiones, se define el \textbf{área estocástica de Lévy} como:

\[
	S(t, W) = \frac{1}{2} \int_0^t (W_s^1 dW_s^2 - W_s^2 dW_s^1)
\]

donde se usa la integral de Itô.

\subsubsection{La KdV y el área de Lévy.}

% The Itˆo–Nisio theorem, quadratic Wiener functionals, and 1-solitons



\subsection{Análisis Gaussiano, Espacios de Wiener Abstractos.}

 % Consultar http://www.tjsullivan.org.uk/pdf/MA482_Stochastic_Analysis.pdf

% Análisis Gaussiano

¿Por qué trabajar con estas medidas? POr el teorema del límite central... una maravilla.


Dimensión finita. 

\begin{boxDef}
	Sea $n \in N$, $B_0 (\mathbb{R}^n)$ la completez (Hacer todos los conjuntos nulos medibles) de la $\sigma$-álgebra de Borel en $\mathbb{R}^n$. Sea $\lambda^n: B_0 (\mathbb{R}^n) \rightarrow [0, \infty]$ la medida de Lebesgue usual $n$-dimensional. Defina \textbf{medida Gaussiana estándar} $\gamma^n: B_0 (\mathbb{R}^n) \rightarrow [0,1]$ como:

	\[
		\gamma^n (A) = \frac{1}{\sqrt{(2\pi)^n}} \exp \left( -\frac{1}{2} \lVert x \rVert_{\mathbb{R}^n}^2 \right) d \lambda^n (x)
	\]
\end{boxDef}

O en función de la derivada de \textit{Radon-Nikodym} (Relación entre dos medidas, en un mismo espacio)

\[
	\frac{d\gamma^n}{d\lambda^n} (x) = \frac{1}{(\sqrt{2\pi})^n} \exp\left( -\frac{1}{2} \lVert x \rVert_{\mathbb{R}^n}^2 \right)
\]

Y para medidas Gaussianas con media $\mu \in \mathbb{R}^n$ y varianza $\sigma^2 > 0$

\[
	\gamma_{\mu, \sigma^2}^n (A) = \frac{1}{\sqrt{(2\pi \sigma^2)^n}} \int_A \exp\left( -\frac{1}{2\sigma^2} \lVert x - \mu \rVert_{\mathbb{R}^n}^2 \right) d\lambda^n (x)
\]


\begin{boxDef}
	Una \textbf{medida $\mu$ Gaussiana con media cero}, en un espacio de Banach separable $E$ equipado con una $\sigma$-álgebra $\mathcal{B}$ y una norma $\lvert \cdot \rvert$, es una medida en $(E, \mathcal{B})$ tal que la distribución de cada funcional lineal (Elemento del dual, función del espacio al cuerpo asociado) en $E$ es una variable aleatoria Gaussiana con media cero.
\end{boxDef}



\subsection{Teorema del Soporte de Stroock-Varadhan}




\begin{theorem}[Teorema de soporte de Stroock-Varadhan]
	Sea $V = \left( V_1, \cdots, V_d \right)$ una colección de campos vectoriales $\text{Lip}^2$ en $\mathbb{R}^e$, y $V_0$ es $\text{Lip}^1 (\mathbb{R}^e)$. Sea $B$ un movimiento Browniano $d$-dimensional, y sea (Bajo indistiguibilidad) la única solución de Stratonovich de la ecuación diferencial estocástica $Y$ en $[0,T]$ en:

	\[
		dY = \sum_{i = 1}^d V_i(Y) \circ dB^i + V_0 (Y) dt, \quad Y_0 = y_0 \in \mathbb{R}^e
	\]

	o de forma integral:

	\[
		Y_t = Y_0 + \sum_{i = 1}^d  \int_0^t V_i(Y) \circ dB^i + \int_0^t V_0 (Y) dt, \quad Y_0 = y_0 \in \mathbb{R}^e
	\]

	(Suma incluye una suma de integrales estocásticas de Stratonovich para los términos $\text{Lip}^2$).

	Escribamos $y^h = \pi_{(V, V_0)} (0, y_0; (h,t))$ (Mapa de Itô) para la solución de la ecuación diferencial ordinaria (asociada)

	\[
		dy = \sum_{i = 1}^d V_i (Y) dh^i + V_0 (Y) dt
	\]

	comenzando en $y_0 \in \mathbb{R}^e$ donde $h$ es un camino de \textit{Cameron-Martin} (Esto es, $h \in W_0^{1,2} ([0,T], \mathbb{R}^d)$).

	Entonces, para $\alpha \in \left[ 0, 1/2 \right)$, y cualquier $\delta > 0$,

	\[
		\lim_{\epsilon \rightarrow 0} \mathbb{P} \left( \lVert Y - y^h \rVert_{ \alpha }  \leq \delta \quad \vert \quad \lvert B - h \rvert_{\infty, [0,T]}  \leq \epsilon \right) \rightarrow 1
	\]

	(Del lado derecho, tenemos norma Euclídea $\lvert B - h \rvert_{\infty, [0,T]} $).

\end{theorem}

