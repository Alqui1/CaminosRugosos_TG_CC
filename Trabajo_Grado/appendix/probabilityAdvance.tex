\section{Temas de Probabilidad avanzados}

Algunos conceptos, teoremas, etc... de probabilidad, procesos estocásticas que pueden estar algo fuera de la tesis.

\subsection{Semimartingalas, Martingalas locales}

Conceptos necesarios para entender algunas definiciones acerca el movimiento Browniano. Asuma el espacio filtrado de probabilidad $(\Omega, \mathcal{F}, \mathcal{F}_t, P)$

\begin{boxDef}
	Un proceso $A$ es \textbf{creciente} o de \textbf{variación finita}, si es adaptado (A la filtración del espacio), y las trayectorias $t \mapsto A_t (\omega)$ son finitas, continuas a derecha y crecientes (O variación finita), para casi todo $\omega$
\end{boxDef} 


\subsection{Variación Cuadrática}

\begin{boxDef}
	Dado $\{  X_t \}$ un proceso estocástico con valores reales definidos en un espacio de probabilidad $(\Omega, \mathcal{F}, P)$. Definimos la \textbf{variación cuadrática} como:

	\[
		\langle X \rangle_t = \lim_{ \lVert P \rVert \rightarrow 0 } \sum_{k = 1}^n (X_{t_k} - X_{t_{k-1}})^2	
	\]

\end{boxDef}


\subsection{Desigualdad de Burkholder-Davis-Gundy}

Usado en la demostración que el movimiento Browniano con la integral de Itô es un camino rugoso.

% https://yelmaazouz.org/content/documents/BGD_inequalities.pdf

%  Daniel Revuz and Marc Yor. Continuous martingales and Brownian motion,

Primero, verificar la caracterización de Lévŷ para el movimiento Browniano.

\begin{theorem}[Lévy]
	Sea $X$ un proceso $d$-dimensional $\mathcal{F}_t$-adaptado y continuo, con $X_0 = 0$. Se tienen las siguientes equivalencias:

	\begin{itemize}

		\item $X$ es un movimiento Browniano respecto a $\mathcal{F}_t$.

		\item $X$ es una martingala local continua (¿?) y $\langle X^i, X^j \rangle_t = \delta_{i,j} t$.

		\item $X$ es una martingala local continua y para cada elección de funciones $f_1, \cdots, f_d \in L^2 (\mathbb{R}_+)$, el proceso

		\[
			\mathcal{E}_t = \exp \left( i \sum_{k = 1}^d \int_0^t f_k (s) dX^k_s + \frac{1}{2} \sum_{k=1}^d \sum_0^t f^2_k(s) ds \right)
		\]

		es una martingala compleja.
	
	\end{itemize}

\end{theorem}

\textbf{Demostración:}

\begin{flushright}
	$\Box$
\end{flushright}

Note que las normas $\lVert \cdot \rVert_1$ y $\lVert \cdot \rVert_2$ definido en el espacio de martingalas continuas acotadas en $L^2$, que se anulan en $0$, son equivalentes por la desigualdad de Doob's (Ver preliminares en PE):

\[
	\lVert M \rVert_1 = \mathbb{E}[M^2_{\infty}]^{1/2} = \mathbb{E}[ \langle  M, M \rangle_{\infty}]^{1/2} \qquad \lVert M \rVert_2 = \mathbb{E}[ (M^*_{\infty})^2 ]^{1/2}
\]

Con esto en mente, se enuncia la desigualdad de \textit{Burkholder-Davis-Gundy}:

\begin{theorem}[Desigualdad de Burkholder-Davis-Gundy] 
Para cualquier $p > 0$, existen dos constantes $c_p$ y $C_p$ tal que para toda martingala local continua $M$ que se anula en $0$, tenemos:

\[
	c_p \mathbb{E}[ \langle M,M \rangle_{\infty}^{p/2} ] \leq \mathbb{E}[ (M^*_{\infty})^p ] \leq C_p \mathbb{E}[ \langle M,M \rangle_{\infty}^{p/2} ]
\]

\end{theorem}

45¿Qué es $M^*_{\infty}$ y $\langle M,M \rangle$?

% https://math.uchicago.edu/~may/REU2019/REUPapers/Carlstein.pdf

