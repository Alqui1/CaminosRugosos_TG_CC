\section{Teoría de la Medida}

Algunas notas acerca de teoría de la medida, usadas para el presente trabajo de grado.

\subsection{Funciones $L^p$ y $\mathcal{L}^p$}

Sea $(X, \mathcal{A}, \mu)$ un espacio de medida, con $1 < p < \infty$ ($p$ no necesariamente entero). Denotará a:

\[
	\mathcal{L}^p (X, \mathcal{A}, \mu, \mathbb{R})
\]

el conjunto de funciones medibles respecto a $\mathcal{A}$, tal que $f: X \rightarrow \mathbb{R}$, donde $\lvert f \rvert^p$ es integrable, esto es:

\[
	\int \lvert f \rvert^p d\mu < \infty
\]

Ese conjunto conforma un espacio vectorial sobre $\mathbb{R}$ (¿Por qué?). Más aún, $\mathcal{L}^p (X, \mathcal{A}, \mu, \mathbb{C})$ también es un espacio vectorial sobre $\mathbb{C}$.

Si $p = \infty$, $\mathcal{L}^p (X, \mathcal{A}, \mu, \mathbb{R})$ es el conjunto de funciones acotadas y medibles respecto a $\mathcal{A}$ (O \textit{esencialmente acotada}).

Podemos definir una \textbf{seminorma} en $\mathcal{L}^p$ como:

\[
	\lVert f \rVert_p = \left( \int \lvert f \rvert^p d\mu \right)^{1/p}
\]

Falla al ser norma al pedir que $\lVert x \rVert = 0 \Leftrightarrow x = 0$. Podemos construir a partir de este espacio, un espacio Banach normado, $L^p (X, \mathcal{A}, \mu)$. Sea $\mathcal{N}^p (X, \mathcal{A}, \mu) \subset \mathcal{L}^p (X, \mathcal{A}, \mu)$ las funciones que cumplen:

\[
	f \in \mathcal{L}^p (X, \mathcal{A}, \mu) \text{ y } \lVert f \rVert_p = 0
\] 

Esto es, funciones que en casi todo punto se anulan para $1 < p < \infty$. Si $p = \infty$, son funciones acotadas medibles respecto a $\mathcal{A}$ en X, tal que se anulan en casi todo punto localmente. $\mathcal{N}^p$ es subespacio lineal de $\mathcal{L}^p$. Defina:

\[
	L^p (X, \mathcal{A}, \mu) = \mathcal{L}^p (X, \mathcal{A}, \mu) / \mathcal{N}^p (X, \mathcal{A}, \mu)
\]

Esto es, una colección de cosets, que están definidos por la relación de equivalencia definida por:

\[
	f \tilde g \Leftrightarrow f - g \in \mathcal{N}^p
\]

Esto es, si las funciones son iguales en casi todo punto. (Para $p = \infty$, corresponde a las funciones iguales en casi todo punto localmente). En este caso, $\lVert \cdot \rVert_p$ es una norma en $L^p (X, \mathcal{A}, \mu)$.

Consulte: https://en.wikipedia.org/wiki/Quotient\_space\_(linear\_algebra).





