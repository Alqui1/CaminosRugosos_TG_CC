\chapter{Caminos Rugosos e Integración Rugosa}

La teoría de caminos rugosos (Vea, por ejemplo \cite{Rough_Paths_TL})  permite extender una teoría de ecuaciones diferenciales controladas, para poder trabajar el caso al tener una señal de entrada que sea \textit{ruidosa}, tal como una \textit{semimartingala}, como lo es el movimiento Browniano. Sea una ecuación diferencial controlada:

\[
	d Y_t = f(Y_t) dX_t, \quad Y_0 = \zeta
\]

Si la ecuación determinística admite solución única, entonces se denota como $Y = I_f(X, \zeta)$, y se conoce a $I_f$ como \textbf{mapeo de Itô asociado a $f$}. Podemos expresar, de forma \textit{integral} la solución de esta ecuación diferencial como:

\[
	Y_t = Y_0 + \int_{0}^t f(Y_s) dX_s
\]

Y, el problema de solucionar la ecuación diferencial, implica en definir la integral:

\[
	\int_{0}^t f(Y_s) dX_s	
\]

Recordando el Capítulo 1, se sabe que el integrador $X_s$ debe, en principio, cumplir algunas condiciones de regularidad (Que sea $\alpha$-Hölder, con $\alpha \geq \frac{1}{2}$) para permitir que exista su integral de Young (Consulte teorema \ref{thm:Young}).

Sin embargo, suponga que $X_t$ es un camino mucho menos regular, como las trayectorias de un movimiento Browniano, que son $\alpha$-Hölder con $\alpha \in ( \frac{1}{3}, \frac{1}{2} ]$. Se puede usar el enfoque de integración estocástica propuesto por Itô, pero en este caso, el mapa de Itô, carecerá de continuidad. Además, la integral de Young, no estará bien definida en este caso, por lo que dependerá de la elección de la partición, y más aún, el límite puede no existir. También, otro problema de la integral de Itô, es acerca de la elección de puntos, que puede afectar el valor de la integral, puesto que, las trayectorías tienen una variación muy rápida, y una integral como la de Riemann-Stieljes no es capaz de capturar esta información y más en intervalos de tiempos pequeños.

Se puede dar otro enfoque al problema. Sea $f: \mathbb{R}^d \rightarrow \mathbb{R}$ una función suave, $X: [0,T]\rightarrow \mathbb{R}^d$ un camino $\alpha$-Hölder continuo. Suponga que se desea darle un significado a

\[
	\int_0^T f(X_r) dX_r
\]

Para eso, se usará la expansión de Taylor. Tomando $[s,t] \subset [0, T]$ un intervalo de tiempo lo suficientemente pequeño y $r \in [s,t]$, tenemos:

\[
	f(X_r) = f(X_s) + \nabla f(X_s) (X_r - X_s) + \cdots
\]

Integrando respecto al camino $X$, obtenemos:

\[
	\int_s^t f(X_r) dX_r = f(X_s) (X_t - X_s) + \nabla f(X_s) \int_s^t (X_r - X_s) \otimes dX_r + \cdots
\]

Véase el apéndice XX para una introducción a los tensores. Dado que $\alpha > \frac{1}{3}$, se pueden omitir los términos de grado superior en la expansión. Ahora, suponiendo que $\alpha > \frac{1}{2}$, por la condición de Young, teorema \ref{thm:Young}, se puede probar que:

\[
	\lim_{ \lvert \pi \rvert \rightarrow 0 } \sum_{ [s,t]\in\pi } \int_s^t (X_r - X_s) \otimes dX_r = 0
\]

Sin embargo, al tomar $\alpha \leq \frac{1}{2}$, el término de segundo orden, no necesariamente se anula, por lo que quedará:

\begin{align*}
	\int_0^T f(X_r) dX_r &= \lim_{ [s,t] \in \pi } \int_s^t f(X_r) dX_r \\
	&= \lim_{ [s,t] \in \pi } \left(  f(X_s) (X_t - X_s) + \nabla f(X_s) \int_s^t (X_r - X_s) \otimes dX_r  \right)
\end{align*}


Note que, se le debe dar un significado a la integral, al término de segundo orden. Este se conocerá como \textbf{levantamiento} de $X$, que será un tipo de candidato para el valor de la integral. Este levantamiento se expresa como:

\[
	\int_s^t (X_r - X_s) \otimes dX_r \coloneqq \mathbb{X}_{s,t} 
\]

Y vea que la integral se define como el valor propuesto para el levantamiento (No al contrario, como se podría pensar). Mientras se define qué es un camino rugoso, puede pensar a $\mathbb{X}$ como mayor información codificada por $X$.

Con esto en mente, la idea es, obtener un camino rugoso para $X$, de tal manera, que se puede definir la integral $\int f(X) \otimes X$ como una \textit{integral rugosa}, integrando respecto al camino rugosos $(X, \mathbb{X})$. Con ello, el mapeo solución $(X, \mathbb{X}) \mapsto Y$ será continuo en una topología sutil.

Entonces, resolver ecuaciones diferenciales rugosas, implicará hallar dos funciones:

\[
	X \mapsto (X, \mathbb{X}) \mapsto Y
\]

donde el primer mapeo consiste en agregar más información a $X$, y el segundo mapeo, conocido como \textbf{mapa de Itô-Lyons}, va a la solución del problema. Dado el levantamiento, este mapeo será continuo, e inclusive, en algunos casos, será localmente Lipschitz.

Ya con esta idea acerca de caminos rugosos, se puede ver algunas ventajas al trabajar con este enfoque. En el capítulo, antes de pasar a la definición formal de caminos rugosos, se hablará de caminos $\alpha$-Hölder y algunas de sus propiedades. Con esto, ya se hablará de caminos rugosos, algunas propiedades, y también se verá algunos ejemplos, como lo es el \textit{movimiento Browniano.}





% ========================================
%================ SECCIÓN 1. ALPHA-HÖLDER
% ========================================




\section{Caminos $\alpha$-Hölder}


Sea $\alpha \in (0, 1]$. Recuerde que una trayectoria $X: [O, T] \in \mathbb{R}^d$ es $\alpha$\textbf{-Hölder continua}, si se cumple que:

\[
	\lvert X_t - X_s \rvert \leq C \lvert t - s \rvert^{\alpha}
\]

para $s, t \in [0,T]$ con $s < t$.

\begin{boxDef}
	Para $\alpha \in (0, 1]$, defina una \textbf{seminorma $\alpha$-Hölder} de $X$, como:

	\[
		\lVert X \rVert_{\alpha} = \sup_{0 \leq s < t \leq T} \frac{ \lvert X_{s,t} \rvert }{ \lvert t - s \rvert^{\alpha} }
	\]

	Si $\lVert X \rVert_{\alpha} < \infty$, el camino se denomina \textbf{ $\alpha$-Hölder continuo }. El espacio de los caminos $\alpha$-Hölder continuo se denota por $\mathcal{C}^{\alpha} = \mathcal{C}^{\alpha}([0,T]; \mathbb{R}^d) $

\end{boxDef}






% Problemas con integral de Itô, motivación a la integración rugosa.

% Espacios de Hölder, Propiedades.
% Colocar simulaciones.

% Caminos Rugosos, Caminos Rugosos Geométricos (Interpretación prof. Freddy).



% Movimiento Browniano como ejemplo a camino rugosos.

\section{Caminos Rugosos y el movimiento Browniano.}

Recuerde, que la integral de Itô está definina como:

\[
	\int_s^t B_{s,r} dB_r
\]
 
esto es, tomando el punto izquierdo. En esta sección, se va a ver que el movimiento Browniano, en conjunto con esta integral, conforman un camino rugoso para $\alpha \in \left[\frac{1}{3}, \frac{1}{2} \right]$. Con esto, se puede aplicar la teoría de caminos rugosos e integración rugosa, a la integración estocástica, lo que abre un gran mundo de posibilidades.

En primer lugar, se va a demostrar el teorema fuerte de esta sección, que corresponde al \textit{criterio de Kolmogorov de caminos rugosos}, el cuál, parafraseando, mostrará la existencia del una modificación para ciertos procesos estocásticos medibles, cuya modificación será un camino rugoso. Más específicamente, el enunciado dice:

\begin{theorem}[Criterio de Kolmogorov para caminos rugosos]

Sea $(X, \mathbb{X}): \Omega \times [0, T] \rightarrow \mathbb{R}^d \times \mathbb{R}^{d \times d}$ un proceso estocástico medible (Respecto a la $\sigma$-álgebra producto $\mathcal{F} \otimes \mathcal{B}[0, T]$) que, para casi todo $\omega \in \Omega$, satisface la relación de Chen. Sea $q \leq 2$ y $\beta > \frac{1}{q}$. Suponga que existe una constante $C > 0$, tal que, para todo $(s,t)\in \Delta_{[0,T]} $,

\[
	\lVert X_{s,t} \rVert_{L^q} \leq C \lvert t-s \rvert^{\beta}, \quad \lVert \mathbb{X}_{s,t} \rVert_{L^{q/2}} \leq C \lvert t - s \rvert^{2\beta}
\]

(Esta elección que constantes $q$ y $\beta$ se hace para concuerde con la definición de camino rugoso, que $X$ sea $\beta$-Hölder y el levantamiento $\mathbb{X}$ sea $2\beta$-Hölder).

Entonces, para todo $\alpha \in [0, \beta - \frac{1}{q}]$, existe una modificación $(\tilde{X}, \tilde{\mathbb{X}})$ de $(X, \mathbb{X})$ y también existen variables aleatorias $K_{\alpha} \in L^q$, $\mathbb{K}_{\alpha} \in L^{q/2}$ tal que, para todo $(s,t) \in \Delta_{[0,T]}$,

\[
	\lvert \tilde{X}_{s,t} \rvert \leq K_{\alpha} \lvert t - s \rvert^{\alpha}, \quad \lvert \tilde{ \mathbb{X} }_{s,t} \rvert \leq \mathbb{K}_{\alpha} \lvert t - s \rvert^{2\alpha}
\]

(Esta relación es para casi toda $\omega \in \Omega$ ¿?). En particular, si $\beta - \frac{1}{q} > \frac{1}{3}$, entonces, para todo $\alpha \in (\frac{1}{3}, \beta - \frac{1}{q})$, tenemos que $(\tilde{X}, \tilde{\mathbb{X}})$

\end{theorem}

En primer lugar, se puede ver el proceso estocástico como la siguiente función:

\[
	Y_t (\omega) = (X_{s,t} (\omega), \mathbb{X}_{s,t} (\omega) )
\]

donde $\mathbb{X}_{s,t} (\omega) = \int_s^t X_{s,r} (\omega) \otimes dX_{r}$, tal que para casi todo $\omega \in \Omega$, se cumple que:

\[
	\mathbb{X}_{s,t} (\omega) = \mathbb{X}_{s,u} (\omega) + \mathbb{X}_{u,t} (\omega) + X_{s,u} (\omega) \otimes X_{u,t} (\omega)
\]

Esto es, se cumple para casi todo $\omega$ la relación del Chen.

\textbf{Demostración:} En este caso, se probará primero el enunciado, para el conjunto de particiones diádicas, $\mathcal{D}_n = \left\{\ \frac{k}{2^n} \vert k = 0, 1, \cdots, 2^n - 1 \right\}$. Además, se puede, sin pérdida de generalidad, tomar $T = 1$. Defina, así:

\[
	K_n = \max_{t \in \mathcal{D}_n} \lvert X_{t, t + 2^{-n}} \rvert \quad \mathbb{K}_n = \max_{t \in \mathcal{D}_n} \lvert \mathbb{X}_{t, t + 2^{-n}} \rvert
\]

(Este corresponde a la modificación propuesta!)

Echando cuentas, tenemos:

\begin{align*}
	\mathbb{E}[K^q_n] &= \mathbb{E}\left[  \max_{t \in \mathcal{D}_n} \lvert X_{t, t + 2^{-n}} \rvert^q \right] \quad \text{Sumar sobre los otros términos positivos de } \mathcal{D}_n \\
	&\leq \mathbb{E} \left[ \sum_{t \in \mathcal{D}_n} \lvert X_{t, t + 2^{-n}} \rvert^q \right] \quad \text{Aplicar desigualdad triangular} \\
	&\leq \sum_{t \in \mathcal{D}_n} \mathbb{E} [ \lvert X_{t, t + 2^{-n}} \rvert^q ] \quad \text{Recuerde} \lVert X_{s,t} \rVert_{L^q} = \mathbb{E}[ \lvert X_{s,t} \rvert^q ]^{1/q} \\
	&= \sum_{t \in \mathcal{D}_n} \lVert X_{t,t+ 2^{-n}} \rVert^q \quad \text{Use las hipótesis acerca de la norma}  \\
	&\leq \sum_{t \in \mathcal{D}_n} C^q \lvert t - (t - 2^{-n}) \rvert^{q \beta} \quad \text{ Hay } 2^n \text{elementos en la partición} \\
	&= C^q \lvert 2^{-n} \rvert^{q \beta} \cdot 2^n \\
	&= C^q 2^{-n \beta q + n} = C^q 2^{n (- \beta q + 1)}
\end{align*}

Puede repetir esta cuenta con el levantamiento $\mathbb{K}_n$, ¡Vamos a ello!: 

\begin{align*}
	\mathbb{E}[ \mathbb{K}_n^q ] &= \mathbb{E}[ \max_{t \in \mathcal{D}_n} \lvert \mathbb{X}_{t, t + 2^{-n}}  \rvert^q ] 
\end{align*}

% Integración, Lema de Costura, Integración de Young.

% Caminos Controlados (Ejemplos) e Integración Rugosa.