\chapter{Caminos Rugosos e Integración Rugosa}

La teoría de caminos rugosos (Vea, por ejemplo \cite{Rough_Paths_TL})  permite extender una teoría de ecuaciones diferenciales controladas, para poder trabajar el caso al tener una señal de entrada que sea \textit{ruidosa}, tal como una \textit{semimartingala}, como lo es el movimiento Browniano. Sea una ecuación diferencial controlada:

\[
	d Y_t = f(Y_t) dX_t, \quad Y_0 = \zeta
\]

Si la ecuación determinística admite solución única, entonces se denota como $Y = I_f(X, \zeta)$, y se conoce a $I_f$ como \textbf{mapeo de Itô asociado a $f$}. Podemos expresar, de forma \textit{integral} la solución de esta ecuación diferencial como:

\[
	Y_t = Y_0 + \int_{0}^t f(Y_s) dX_s
\]

Y, el problema de solucionar la ecuación diferencial, implica en definir la integral:

\[
	\int_{0}^t f(Y_s) dX_s	
\]

Recordando el Capítulo 1, se sabe que el integrador $X_s$ debe, en principio, cumplir algunas condiciones de regularidad (Que sea $\alpha$-Hölder, con $\alpha \geq \frac{1}{2}$) para permitir que exista su integral de Young (Consulte teorema \ref{thm:Young}).

Sin embargo, suponga que $X_t$ es un camino mucho menos regular, como las trayectorias de un movimiento Browniano, que son $\alpha$-Hölder con $\alpha \in ( \frac{1}{3}, \frac{1}{2} ]$. Se puede usar el enfoque de integración estocástica propuesto por Itô, pero en este caso, el mapa de Itô, carecerá de continuidad. Además, la integral de Young, no estará bien definida en este caso, por lo que dependerá de la elección de la partición, y más aún, el límite puede no existir. También, otro problema de la integral de Itô, es acerca de la elección de puntos, que puede afectar el valor de la integral, puesto que, las trayectorías tienen una variación muy rápida, y una integral como la de Riemann-Stieljes no es capaz de capturar esta información y más en intervalos de tiempos pequeños.

Se puede dar otro enfoque al problema. Sea $f: \mathbb{R}^d \rightarrow \mathbb{R}$ una función suave, $X: [0,T]\rightarrow \mathbb{R}^d$ un camino $\alpha$-Hölder continuo. Suponga que se desea darle un significado a

\[
	\int_0^T f(X_r) dX_r
\]

Para eso, se usará la expansión de Taylor. Tomando $[s,t] \subset [0, T]$ un intervalo de tiempo lo suficientemente pequeño y $r \in [s,t]$, tenemos:

\[
	f(X_r) = f(X_s) + \nabla f(X_s) (X_r - X_s) + \cdots
\]

Integrando respecto al camino $X$, obtenemos:

\[
	\int_s^t f(X_r) dX_r = f(X_s) (X_t - X_s) + \nabla f(X_s) \int_s^t (X_r - X_s) \otimes dX_r + \cdots
\]

Véase el apéndice XX para una introducción a los tensores. Dado que $\alpha > \frac{1}{3}$, se pueden omitir los términos de grado superior en la expansión. Ahora, suponiendo que $\alpha > \frac{1}{2}$, por la condición de Young, teorema \ref{thm:Young}, se puede probar que:

\[
	\lim_{ \lvert \pi \rvert \rightarrow 0 } \sum_{ [s,t]\in\pi } \int_s^t (X_r - X_s) \otimes dX_r = 0
\]

Sin embargo, al tomar $\alpha \leq \frac{1}{2}$, el término de segundo orden, no necesariamente se anula, por lo que quedará:

\begin{align*}
	\int_0^T f(X_r) dX_r &= \lim_{ [s,t] \in \pi } \int_s^t f(X_r) dX_r \\
	&= \lim_{ [s,t] \in \pi } \left(  f(X_s) (X_t - X_s) + \nabla f(X_s) \int_s^t (X_r - X_s) \otimes dX_r  \right)
\end{align*}


Note que, se le debe dar un significado a la integral, al término de segundo orden. Este se conocerá como \textbf{levantamiento} de $X$, que será un tipo de candidato para el valor de la integral. Este levantamiento se expresa como:

\[
	\int_s^t (X_r - X_s) \otimes dX_r \coloneqq \mathbb{X}_{s,t} 
\]

Y vea que la integral se define como el valor propuesto para el levantamiento (No al contrario, como se podría pensar). Mientras se define qué es un camino rugoso, puede pensar a $\mathbb{X}$ como mayor información codificada por $X$.

Con esto en mente, la idea es, obtener un camino rugoso para $X$, de tal manera, que se puede definir la integral $\int f(X) \otimes X$ como una \textit{integral rugosa}, integrando respecto al camino rugosos $(X, \mathbb{X})$. Con ello, el mapeo solución $(X, \mathbb{X}) \mapsto Y$ será continuo en una topología sutil.

Entonces, resolver ecuaciones diferenciales rugosas, implicará hallar dos funciones:

\[
	X \mapsto (X, \mathbb{X}) \mapsto Y
\]

donde el primer mapeo consiste en agregar más información a $X$, y el segundo mapeo, conocido como \textbf{mapa de Itô-Lyons}, va a la solución del problema. Dado el levantamiento, este mapeo será continuo, e inclusive, en algunos casos, será localmente Lipschitz.

Ya con esta idea acerca de caminos rugosos, se puede ver algunas ventajas al trabajar con este enfoque. En el capítulo, antes de pasar a la definición formal de caminos rugosos, se hablará de caminos $\alpha$-Hölder y algunas de sus propiedades. Con esto, ya se hablará de caminos rugosos, algunas propiedades, y también se verá algunos ejemplos, como lo es el \textit{movimiento Browniano.}





% ========================================
%================ SECCIÓN 1. ALPHA-HÖLDER
% ========================================




\section{Caminos $\alpha$-Hölder}


Sea $\alpha \in (0, 1]$. Recuerde que una trayectoria $X: [O, T] \in \mathbb{R}^d$ es $\alpha$\textbf{-Hölder continua}, si se cumple que:

\[
	\lvert X_t - X_s \rvert \leq C \lvert t - s \rvert^{\alpha}
\]

para $s, t \in [0,T]$ con $s < t$.

\begin{boxDef}
	Para $\alpha \in (0, 1]$, defina una \textbf{seminorma $\alpha$-Hölder} de $X$, como:

	\[
		\lVert X \rVert_{\alpha} = \sup_{0 \leq s < t \leq T} \frac{ \lvert X_{s,t} \rvert }{ \lvert t - s \rvert^{\alpha} }
	\]

	Si $\lVert X \rVert_{\alpha} < \infty$, el camino se denomina \textbf{ $\alpha$-Hölder continuo }. El espacio de los caminos $\alpha$-Hölder continuo se denota por $\mathcal{C}^{\alpha} = \mathcal{C}^{\alpha}([0,T]; \mathbb{R}^d) $

\end{boxDef}









% Problemas con integral de Itô, motivación a la integración rugosa.

% Espacios de Hölder, Propiedades.
% Colocar simulaciones.

% Caminos Rugosos, Caminos Rugosos Geométricos (Interpretación prof. Freddy).





% ========================================
%================ SECCIÓN 2. CAMINOS RUGOSOS
% ========================================

Una vez con los preliminares en trayectorias $\alpha$-Hölder, se comienza a trabajar acerca de caminos rugosos.


\begin{boxDef}
	Para $\alpha \in \left( \frac{1}{3}, \frac{1}{2} \right]$, un \textbf{camino rugoso $\alpha$-Hölder} corresponde a una $\mathbf{X} = (X, \mathbb{X})$, donde:

	\begin{itemize}
		\item $X: [0,T] \rightarrow \mathbb{R}^d$ es una trayectoria $\alpha$-Hölder continua.
		\item $\mathbb{X}: \Delta_{[0,T]} \rightarrow \mathbb{R}^{d \times d}$ es $2\alpha$-Hölder continua.
	\end{itemize}

	Además, se cumple la \textit{relación de Chen}:

	\[
		\mathbb{X}_{s,t} = \mathbb{X}_{s,u} + \mathbb{X}_{u,t} + X_{s,u} \otimes X_{u,t}
	\]

	para todo $0 \leq s \leq u \leq t \leq T$.
\end{boxDef}



















% ========================================
%================ SECCIÓN 3. MOVIMIENTO BROWNIANO
% ========================================





% Movimiento Browniano como ejemplo a camino rugosos.

\section{Caminos Rugosos y el movimiento Browniano.}

Recuerde, que la integral de Itô está definina como:

\[
	\int_s^t B_{s,r} dB_r
\]
 
esto es, tomando el punto izquierdo. En esta sección, se va a ver que el movimiento Browniano, en conjunto con esta integral, conforman un camino rugoso para $\alpha \in \left[\frac{1}{3}, \frac{1}{2} \right]$. Con esto, se puede aplicar la teoría de caminos rugosos e integración rugosa, a la integración estocástica, lo que abre un gran mundo de posibilidades.

En primer lugar, se va a demostrar el teorema fuerte de esta sección, que corresponde al \textit{criterio de Kolmogorov de caminos rugosos}, el cuál, parafraseando, mostrará la existencia del una modificación para ciertos procesos estocásticos medibles, cuya modificación será un camino rugoso. Más específicamente, el enunciado dice:

\begin{theorem}[Criterio de Kolmogorov para caminos rugosos]

Sea $(X, \mathbb{X}): \Omega \times [0, T] \rightarrow \mathbb{R}^d \times \mathbb{R}^{d \times d}$ un proceso estocástico medible (Respecto a la $\sigma$-álgebra producto $\mathcal{F} \otimes \mathcal{B}[0, T]$) que, para casi todo $\omega \in \Omega$, satisface la relación de Chen. Sea $q \geq 2$ y $\beta > \frac{1}{q}$. Suponga que existe una constante $C > 0$, tal que, para todo $(s,t)\in \Delta_{[0,T]} $,

\[
	\lVert X_{s,t} \rVert_{L^q} \leq C \lvert t-s \rvert^{\beta}, \quad \lVert \mathbb{X}_{s,t} \rVert_{L^{q/2}} \leq C \lvert t - s \rvert^{2\beta}
\]

(Esta elección que constantes $q$ y $\beta$ se hace para concuerde con la definición de camino rugoso, que $X$ sea $\beta$-Hölder y el levantamiento $\mathbb{X}$ sea $2\beta$-Hölder).

Entonces, para todo $\alpha \in [0, \beta - \frac{1}{q}]$, existe una modificación $(\tilde{X}, \tilde{\mathbb{X}})$ de $(X, \mathbb{X})$ y también existen variables aleatorias $K_{\alpha} \in L^q$, $\mathbb{K}_{\alpha} \in L^{q/2}$ tal que, para todo $(s,t) \in \Delta_{[0,T]}$,

\[
	\lvert \tilde{X}_{s,t} \rvert \leq K_{\alpha} \lvert t - s \rvert^{\alpha}, \quad \lvert \tilde{ \mathbb{X} }_{s,t} \rvert \leq \mathbb{K}_{\alpha} \lvert t - s \rvert^{2\alpha}
\]

(Esta relación es para casi toda $\omega \in \Omega$ ¿?). En particular, si $\beta - \frac{1}{q} > \frac{1}{3}$, entonces, para todo $\alpha \in (\frac{1}{3}, \beta - \frac{1}{q})$, tenemos que $(\tilde{X}, \tilde{\mathbb{X}})$

\end{theorem}

En primer lugar, se puede ver el proceso estocástico como la siguiente función:

\[
	Y_t (\omega) = (X_{s,t} (\omega), \mathbb{X}_{s,t} (\omega) )
\]

donde $\mathbb{X}_{s,t} (\omega) = \int_s^t X_{s,r} (\omega) \otimes dX_{r}$, tal que para casi todo $\omega \in \Omega$, se cumple que:

\[
	\mathbb{X}_{s,t} (\omega) = \mathbb{X}_{s,u} (\omega) + \mathbb{X}_{u,t} (\omega) + X_{s,u} (\omega) \otimes X_{u,t} (\omega)
\]

Esto es, se cumple para casi todo $\omega$ la relación del Chen.

\textbf{Demostración:} En este caso, se probará primero el enunciado, para el conjunto de particiones diádicas, $\mathcal{D}_n = \left\{\ \frac{k}{2^n} \vert k = 0, 1, \cdots, 2^n - 1 \right\}$. Además, se puede, sin pérdida de generalidad, tomar $T = 1$. Defina, así:

\[
	K_n = \max_{t \in \mathcal{D}_n} \lvert X_{t, t + 2^{-n}} \rvert \quad \mathbb{K}_n = \max_{t \in \mathcal{D}_n} \lvert \mathbb{X}_{t, t + 2^{-n}} \rvert
\]

(Este corresponde a la modificación propuesta!)

Echando cuentas, tenemos:

\begin{align*}
	\mathbb{E}[K^q_n] &= \mathbb{E}\left[  \max_{t \in \mathcal{D}_n} \lvert X_{t, t + 2^{-n}} \rvert^q \right] \quad \text{Sumar sobre los otros términos positivos de } \mathcal{D}_n \\
	&\leq \mathbb{E} \left[ \sum_{t \in \mathcal{D}_n} \lvert X_{t, t + 2^{-n}} \rvert^q \right] \quad \text{Aplicar ¿DT?} \\
	&= \sum_{t \in \mathcal{D}_n} \mathbb{E} [ \lvert X_{t, t + 2^{-n}} \rvert^q ] \quad \text{Recuerde} \lVert X_{s,t} \rVert_{L^q} = \mathbb{E}[ \lvert X_{s,t} \rvert^q ]^{1/q} \\
	&= \sum_{t \in \mathcal{D}_n} \lVert X_{t,t+ 2^{-n}} \rVert^q \quad \text{Use las hipótesis acerca de la norma}  \\
	&\leq \sum_{t \in \mathcal{D}_n} C^q \lvert t - (t - 2^{-n}) \rvert^{q \beta} \quad \text{ Hay } 2^n \text{elementos en la partición} \\
	&= C^q \lvert 2^{-n} \rvert^{q \beta} \cdot 2^n \\
	&= C^q 2^{-n \beta q + n} = C^q 2^{n (- \beta q + 1)}
\end{align*}

Puede repetir esta cuenta con el levantamiento $\mathbb{K}_n$, ¡Vamos a ello!: 

\begin{align*}
	\mathbb{E}[ \mathbb{K}_n^{q/2} ] &= \mathbb{E}[ \max_{t \in \mathcal{D}_n} \lvert \mathbb{X}_{t, t + 2^{-n}}  \rvert^{q/2} ] \quad \text{Sumar sobre los otros términos } \mathcal{D}_n \\
	&\leq \mathbb{E}\left[ \sum_{t \in \mathcal{D}_n}  \lvert \mathbb{X}_{t, t + 2^{-n}} \rvert^{q/2} \right] \\
	&= \sum_{t \in \mathcal{D}_n} \mathbb{E} [\lvert \mathbb{X}_{t, t + 2^{-n}} \rvert^{q/2}] \quad \text{Definición de norma} \\
	&= \sum_{t \in \mathcal{D}_n} \lVert X_{t, t + 2^{-n}} \rVert_ {L^{q/2}}^{q/2} \quad \text{Aplicar hipótesis, cota para } \mathbb{X} \\
	&\leq \sum_{t \in \mathcal{D}_n} C^{q/2} 2^{-n \cdot q/2 \cdot 2\beta} \\
	&= C^{q/2} 2^{-nq\beta + n} = C^{q/2} 2^{-n(q \beta - 1 )}
\end{align*}


Ya tenemos cotas para los valores esperados de $K_n^q$ y $\mathbb{K}_n^q$! ¿Y ahora? La idea es, con estas cotas, mostrar que $K_{\alpha} \in L^p$ y $\mathbb{K}_{\alpha} \in L^(p/2)$ (\textbf{VEA, EN EL APÉNDICE, ESPACIOS} $L^p$ ).


%========================%========================

\begin{comment}

Fije $s < t$ en $\cup_{n \leq 0} \mathcal{D}_n$ (La unión de todos los elementos en la partición diádica, esto es, $t = \frac{k}{2^n}$ para algún $k, n$). Seleccione $m$, de tal forma que $2^{-(m+1)} < t - s < 2^{-m} $. 

Note que, $[s,t]$ se puede expresar como la unión finita de intervalos de la forma $[u, v] \in \mathcal{D}_n$, con $n > m + 1$ donde no hay tres intervalos con el mismo tamaño. (Creo que ya entendí :D... ). Dicho de otra forma, tenemos una partición de $[s, t]$ de la forma

\[
	s = u_0 < u_1 < \cdots < u_N = t
\]

donde $[u_i, u_{i+1}] \in \mathcal{D}_n$ para algún $n > m + 1$, y para cada $n \geq m + 1$, hay a lo sumo, dos intervalos tomados de $\mathcal{D_n}$. Argumento de descomposición multiescala: Argumento de encadenamiento.

\[
	\lvert X_{s,t} \rvert \leq \max_{0 \leq i < N} \lvert X_{s, u_{i+1} } \rvert \leq \sum_{i = 0}^{N-1} \lvert X_{u_i, u_{i+1} } \rvert \leq 2 \sum_{n = m+1}^{\infty} K_n
\]

\end{comment}

Dados $s < t$, elementos tomados de $\cup_{n \leq 0} \mathcal{D}_n$ (La unión de todos los elementos en la partición diádica, esto es, $t = \frac{k}{2^n}$ para algún $k, n$). Seleccione algún $m$, de tal forma que 

\[
	2^{-(m + 1)} < t - s < 2^{-m}
\]

Para cada $n \geq m+1$, tome $u_n, v_n \in \mathcal{D}_n$, tal que $u_n < v_n$, $[u_n, v_n] \subset [s,t]$ y $\cup [u_n, v_n] = [s,t]$, esto es, una partición. Note que, no es posible que existan 3 subintervalos con el mismo tamaño, ¿Por qué? Graficando esta situación tenemos:


En este caso, se aplicó unrgumento de descomposición multiescala: Argumento de encadenamiento. Luego, tenemos

\[
	\lvert X_{s,t} \rvert \leq \max_{0 \leq i < N} \lvert X_{s, u_{i+1} } \rvert \leq \sum_{i = 0}^{N-1} \lvert X_{u_i, u_{i+1} } \rvert \leq 2 \sum_{n = m+1}^{\infty} K_n
\]

Y de manera análoga,

\begin{align*}
	\lvert \mathbb{X}_{s,t} \rvert &= \left\lvert \sum_{i = 0}^{N - 1} (\mathbb{X}_{u_i, u_{i+1}}) + X_{s,u_i} \otimes X_{u_i, u_{i + 1}} \right\rvert \quad \text{Aplique desigualdad triangular dos veces} \\
	&\leq \sum_{i = 0}^{N - 1} \lvert \mathbb{X}_{u_i, u_{i+1}} \rvert + \lvert X_{s, u_i} \rvert \lvert X_{u_i, u_{i + 1}} \rvert \quad \text{Saque el máximo} \\
	&\leq \sum_{i = 0}^{N - 1} \lvert \mathbb{X}_{u_i, u_{i+1}} \rvert + \left( \max_{0 \leq i < N} \lvert X_{s, u_i} \rvert \right) \sum_{i = 0}^{N - 1} \lvert X_{u_i, u_{i + 1}} \rvert \quad \text{Use las consecuencias de lo anterior} \\
	&= 2 \sum_{i = m + 1}^{\infty} \mathbb{K}_n + 2 \sum_{i = m + 1}^{\infty} K_n \cdot 2 \sum_{i = m + 1}^{\infty} K_n \\
	&=  2 \sum_{i = m + 1}^{\infty} \mathbb{K}_n + \left( 2 \sum_{i = m + 1}^{\infty} K_n \right)^2
\end{align*}

%========================%========================

Con esto, obtener otra cota para la norma $\alpha$-Hölder de $X$:

\begin{align*}
	\frac{ \lvert X_{s,t} \rvert }{ \lvert t - s \rvert^{\alpha} } &\leq 2 \sum_{n = m + 1}^{\infty} \frac{K_n}{2^{-(m+1) \alpha}} \quad \lvert t - s \rvert \geq 2^{-(m+1)} \\
	&\leq 2 \sum_{n = m + 1}^{\infty} \frac{K_n}{2^{-n \alpha}} \quad \text{Porque } 2^{-n} < 2^{-(m+1)} \\
	&\leq 2 \sum_{n = 0}^{\infty} \frac{K_n}{2^{-n \alpha}} =: K_{\alpha} \quad \text{Porque } 2^{-n} < 2^{-(m+1)} \\
\end{align*}

Esta es la variable aleatoria deseada. Observe, finalmente que:

\begin{align*}
	\lVert K_{\alpha} \rVert_{L^q} &= \left\lVert 2 \sum_{n=0}^{\infty} \frac{K_n}{2^{-n\alpha}} \right\rVert \quad \text{Aplicar desigualdad triangular} \\
	&\leq 2 \sum_{n=0}^{\infty} \frac{ \lVert K_n \rVert_{L^q} }{ 2^{-n\alpha} } \quad \text{Definición de la norma} \\
	&= 2 \sum_{n=0}^{\infty} \frac{ \mathbb{E}[K_n^q]^{1/q} }{ 2^{-n\alpha} }\\
	&\leq 2 \sum_{n=0}^{\infty} \frac{  \left(C^q 2^{-n(\beta q - 1)} \right)^{1/q}  }{ 2^{-n\alpha} } \quad \text{Por el resultado anterior. Hagamos más cuentas!} \\
	&= 2 \sum_{n=0}^{\infty} \frac{  C \cdot 2^{-n(\beta - \frac{1}{q})}  }{ 2^{-n\alpha} } \\
	&= 2C \sum_{n=0}^{\infty} 2^{-n(\beta - \frac{1}{q} - \alpha}) < \infty
\end{align*}

Este último es, debibo a que tendremos una serie geométrica. Note que, por hipótesis, $\beta > \frac{1}{q}$ o $\beta - \frac{1}{q} > 0$. Como $\alpha \in \left[ 0, \beta - \frac{1}{q} \right)$, $\beta - \frac{1}{q} > \alpha$ o $\beta - \frac{1}{q} - \alpha > 0$, por ende, la serie es convergente.

Con esto, queda probado que $\lVert K_{\alpha} \rVert_{L^q}$. Ahora, queda probar el resultado análogo para la variable aleatoria $\mathbb{K}_{\alpha}$. Haciendo los trámites correspondientes:

\begin{align*}
	\frac{ \mathbb{X}_{s,t} }{ \lvert t - s \rvert^{2\alpha} } &\leq \frac{1}{2^{-2(m+1)\alpha}} \left[ 2 \sum_{n = m + 1}^{\infty} \mathbb{K}_n + \left( 2 \sum_{n = m + 1}^{\infty} K_n \right)^2 \right] \quad \text{Usando resultados anteriores, expandiendo...} \\
	&= 2 \sum_{n = m + 1}^{\infty} \frac{ \mathbb{K}_n }{ 2^{-2(m+1)\alpha} } + \left( 2 \sum_{n = m + 1}^{\infty} \frac{K_n}{2^{-(m+1)\alpha}} \right)^2 \quad \text{Use las cuentas anteriores} \\
	&\leq 2 \sum_{n = m + 1}^{\infty} \frac{ \mathbb{K}_n }{ 2^{-2(m+1)\alpha} } + K_{\alpha}^2 =: \mathbb{K}_{\alpha}
\end{align*}

Y análogo a la variable aleatoria $K_{¸\alpha}$, hallamos su norma $\lVert \cdot \rVert_{L^{q/2}}$.

\begin{align*}
	\lVert \mathbb{K}_{\alpha} \rVert_{L^{q/2}} &\leq \left\lVert 2 \sum_{n = m + 1}^{\infty} \frac{ \mathbb{K}_n }{ 2^{-2(m+1)\alpha} } + K_{\alpha}^2 \right\rVert_{L^{q/2}} \quad \text{Aplicando la cota del ejercicio anterior, aplicamos desigualdad triangular} \\
	&\leq 2 \sum_{n = m + 1}^{\infty} \frac{ \lVert \mathbb{K}_n \rVert_{L^{q/2}} }{ 2^{-2(m+1)\alpha} } + \lVert K^2_{\alpha} \rVert_{L^{q/2}} \quad \text{Aplicar definición de norma} \\
	%========
	&= 2 \sum_{n = m + 1}^{\infty} \frac{ \mathbb{E}[ \mathbb{K}_n^{q/2} ]^{2/q} }{ 2^{-2(m+1)\alpha} } + \lVert K^2_{\alpha} \rVert_{L^{q/2}} \quad \text{Aplicar cota para el valor esperado de } \mathbb{K}_n^{q/2} \\
	%=======
	&\leq 2 \sum_{n = m + 1}^{\infty} \frac{ \left( C^{q/2} 2^{-n(q\beta - 1)} \right)^{2/q} }{ 2^{-2(m+1)\alpha} } + \lVert K^2_{\alpha} \rVert_{L^{q/2}} \quad \text{Cuentas, verificar si la serie es geométrica} \\
	%=======
	&= 2 \sum_{n = m + 1}^{\infty} \frac{  C \cdot 2^{-2n(\beta - \frac{1}{q})} }{ 2^{-2(m+1)\alpha} } + \lVert K^2_{\alpha} \rVert_{L^{q/2}} \quad \text{Reindexar, suma de términos positivos} \\
	%=======
	&= 2 C \sum_{n = 0}^{\infty} 2^{-2n(\beta - \frac{1}{q} - \alpha)}  + \lVert K^2_{\alpha} \rVert_{L^{q/2}} < \infty
\end{align*}

Porque la serie es geométrica, con $\beta - \frac{1}{q} - \alpha > 0$, y sabemos que $\lVert K^2_{\alpha} \rVert_{L^{q/2}} < \infty$. Así, queda probado que $\mathbb{K}_{\alpha} \in L^{q/2}$.

Note que únicamente, se ha probado el enunciado para los elementos en la partición diádica. Como tarea final, se extenderá el resultado a todo el intervalo $[0, T]$ con $T = 1$.

Note que, el conjunto de todas las particiones diádicas $\cup_{n = 0}^{\infty} \mathcal{D}_n$ de $[0,1]$ es denso en el conjunto $[0, 1]$, de tal forma, que la clausura, el conjunto de puntos de adherencia, es todo el intervalo $[0, 1]$. Entonces, para todo elemento en este conjunto, es posible hallar una sucesión de tal forma que converja a este elemento. Formalizando, sea $t \in [0, 1]$, existe $\left\{ t_k \right\}_{k \geq 1} \subset \cup_{n \geq 0} \mathcal{D}_n$ sucesión en el conjunto de particiones diádicas, de tal forma que $t_k \rightarrow t$. Por las cuentas anteriores, sabemos que $X$ es $\alpha$-Hölder continua en $\cup_{n \geq 0} \mathcal{D}_n$, de tal forma que se define $\tilde{X}_t = \lim_{k \rightarrow \infty} X_{t_k}$ (Este límite existe). Por lema de Fatou:

\begin{align*}
	\lVert \tilde{X}_t - X_t \rVert_{L^q} &\leq \liminf_{k\rightarrow \infty} \lVert X_{t, t_k}\rVert_{L^q} \quad \text{Aplicar hipótesis del problema} \\
	%=============
	&\leq \liminf_{k\rightarrow \infty} C \lvert t - t_k \rvert^{\beta} = 0
\end{align*}

De esta forma, $\tilde{X}_t = X_t$ en casi todo punto, por ende, $\tilde{X}$ es modificación de $X$. Queda verificar que $\tilde{X}_t$, en casi todo punto, no depende de la elección de la sucesión. Sea $s_k \rightarrow s$ y $t_k \rightarrow t$, con $s_k, t_k \in \cup_{n \geq 0} \mathcal{D}_n$, se tiene:

\[
	\lvert \tilde{X}_{s,t} \rvert = \lim_{k \rightarrow \infty} \lvert X_{s_k, t_k} \rvert \leq \lim_{k \rightarrow \infty} K_{\alpha} \lvert t_k - s_k \rvert^{\alpha} = K_{\alpha} \lvert t - s \rvert^{\alpha}
\]

Lo que prueba lo deseado.

Finalmente (Ahora sí, te juro que sí), queda demostrar $\tilde{\mathbb{X}}_{s,t}$ es una modificación de $\mathbb{X}$. En este caso, se toman cuatro casos distintos:

\begin{itemize}

	\item Sea $(s,t) \in \Delta_{[0,1]}$ de tal forma que $0 < s \leq t < 1$. Defina $\{ s_k \}_{k \geq 1}, \{ t_k \}_{k \geq 1} \subset \cup_{n \leq 0} \mathcal{D}_n$ sucesiones, de tal forma que $s_k \nearrow s$ y $t_k \searrow t$ para $k \rightarrow \infty$. 

	\item Los otros casos corresponden a tomar $s = 0$ y $t < 1$, o $0 < s$ y $t = 1$, o $s = 0$ y $t = 1$. ¿COMPLETAR?

\end{itemize}

Defina $\tilde{X}_{s,t} = \lim_{k \rightarrow \infty} \mathbb{X}_{s_k, t_k}$. Como $s_k \leq s \leq t \leq t_k$, se puede aplicar la desigualdad de Chen:

	\begin{align*}
		\mathbb{X}_{s_k, t_k} - \mathbb{X}_{s,t} &= \mathbb{X}_{s_k, s} + \mathbb{X}_{s, t_k} + X_{s_k, s} \otimes X_{s, t_k} - \mathbb{X}_{s, t} \quad \text{Nuevamente relación de Chen} \\
		% ===============================0
		&= \mathbb{X}_{s_k, s} + \mathbb{X}_{s, t} + \mathbb{X}_{t, t_k} + X_{s, t} \otimes X_{t, t_k} + X_{s_k, s} \otimes X_{s, t_k} - \mathbb{X}_{s, t} \\
		&= \mathbb{X}_{s_k, s} + \mathbb{X}_{t, t_k} + X_{s, t} \otimes X_{t, t_k} + X_{s_k, s} \otimes X_{s, t_k}
	\end{align*}

	Calculando la norma en $\lVert \cdot \rVert_{L^{q/2}}$, queda:

	\begin{align*}
		\lVert \mathbb{X}_{s_k, t_k} - \mathbb{X}_{s,t} \rVert_{L^{q/2}} &= \lVert \mathbb{X}_{s_k, s} + \mathbb{X}_{t, t_k} + X_{s, t} \otimes X_{t, t_k} + X_{s_k, s} \otimes X_{s, t_k} \rVert_{L^{q/2}} \quad \text{Desigualdad triangular} \\
		&\leq \lVert \mathbb{X}_{s_k, s} \rVert_{L^{q/2}} + \lVert \mathbb{X}_{t, t_k} \rVert_{L^{q/2}} + \lVert X_{s, t} \rVert_{L^{q}} \lVert X_{t, t_k} \rVert_{L^{q}} + \lVert X_{s_k, s} \rVert_{L^{q}} \lVert X_{s, t_k} \rVert_{L^{q}} \quad \text{Usando lo que ya tenemos para } \mathcal{D}_n \\
		&\leq C \lvert s - s_k \rvert^{2\beta} + C\lvert t_k - t \rvert^{2\beta} + C^2 \lvert s - s_k \rvert^{\beta} \lvert t_k - s \rvert^{\beta} + C^2 \lvert t - s \rvert^{\beta} \lvert t_k - t \rvert^{\beta}
	\end{align*}

	Por lema de Fatou, aplicando límites a ambos lados, implicará que:

	\[
		\lVert \tilde{\mathbb{X}}_{s, t} - X_{s,t} \rVert \rightarrow 0
	\]

	en casi toda pareja $(s,t) \in \Delta_{[0, T]}$. Finalmente, note que:

	\begin{align*}
		\lvert \tilde{\mathbb{X}}_{s,t} \rvert &= \lim_{n \rightarrow \infty} \lvert \mathbb{X}_{s_k, t_k} \rvert \\
			% =======================
			&\leq \lim_{k \rightarrow \infty} \mathbb{K}_{\alpha} \lvert t_k - s_k \rvert^{2\alpha} \\
			&= \mathbb{K}_{\alpha} \lvert t - s \rvert^{2\alpha}
 	\end{align*}

 	de tal forma que, en casi todo punto, $\tilde{\mathbb{X}}$ cumple la condición dada. Con esto, ya pruebe lo deseado.


\begin{flushright}
	$\Box$
\end{flushright}

% ==============================

\subsection{Integral de Itô}

Sea $B_t \in \mathbb{R}^n$ un movimiento Browniano estándar en dimensión $n$. Ya se conoce que las trayectorias del movimiento Browniano son $\alpha$-Hölder continua para $\alpha \in \left[ \frac{1}{3}, \frac{1}{2} \right)$, entonces, se va a proponer un levantamiento que esea $2\alpha$-Hölder continua. Este levantamiento será la, ya vista en capítulos anteriores, \textit{integral de Itô}:

\[
	\mathbb{B}_{s,t} = \int_s^t B_{s,r} \otimes dB_r
\]

Se puede ver esta integral, como una suma de integrales iteradas:

\[
	\mathbb{B}_{s,t} = \sum_{i,j}^d \int_s^t	 B_{s,r}^{(i)} dB_r^{(j)}
\]

O también, de forma matricial:

\[
	\mathbb{B}_{s,j}^{(i,j)} = \int_s^t	 B_{s,r}^{(i)} dB_r^{(j)}	
\]

\begin{prop}
	Para cualquier $\alpha \in \left( \frac{1}{3}, \frac{1}{2} \right)$, casi siempre se cumple que:

	\[
		\mathbf{B} := (B, \mathbb{B}_{s,t}) \in \mathcal{C}^{\alpha}([0,T]; \mathbb{R}^n)
	\]
\end{prop}

\textbf{Demostración:} Para usar el criterio de Kolmogorov de caminos rugosos, en primer lugar se deben verificar las condiciones del problema.

\begin{itemize}

	\item En primer lugar, se conoce que $B_t$ y $\int_s^t B_{s,r} dB_r$ son medibles respecto a $\mathcal{F}_T$ (La $\sigma$-álgebra generada por $\left\{ B_s \right\}_{s \leq T}$), por ende, $(B, \mathbb{B})$ es medible respecto a $\mathcal{F} \otimes \mathcal{B}[0,T]$.

	\item Verificar que cumple la relación de Chen (Casi siempre). Haciendo cuentas, para $s \leq u \leq t$, observe que:

	\begin{align*}
		\mathbb{B}_{s,t} &= \int_s^t B_{s,r} \otimes dB_r \quad \text{Aplicar propiedades integral Itô, Capítulo 3.} \\
		&= \int_s^u B_{s,r} \otimes dB_r + \int_u^t B_{s,r} \otimes dB_r \quad \text{Sumar ceros. El primer índice de } B_{s,r} \text{ es } u. \\
		&= \int_s^u B_{s,r} \otimes dB_r + \int_u^t \left( B_r - B_s + B_u - B_u \right) \otimes dB_r \quad \text{Aplicar aditividad de la integral de Itô para el integrando} \\
		% ==================================
		&= \int_s^u B_{s,r} \otimes dB_r + \int_u^t  B_{u,r} \otimes dB_r + \int_u^t B_u - B_s \otimes dB_r \quad \text{Definición del levantamiento, sacar constante de la tercera integral}  \\
		% ==================================
		&= \mathbb{B}_{s,u} + \mathbb{B}_{u,t} + (B_{s,u}) \otimes \int_u^t dB_r \quad \text{Por construcción de integral de Itô:} \\
		% ==================================
		&= \mathbb{B}_{s,u} + \mathbb{B}_{u,t} + B_{s,u} \otimes B_{u,t}
	\end{align*} 

	Por ende, la relación de Chen se cumple casi siempre.

	\item Sea $q \geq 2$ (¿$q = 2$?) y $\beta > \frac{1}{q}$. Tome $(s,t) \in \Delta_{[0,T]}$. Observe:

	\[
		\lVert B_{s,t} \rVert_{L^q} = \lVert (t-s)^{1/2} B_1 \rVert_{L^q} = \lvert (t-s) \rvert^{1/2} \lVert B_1 \rVert_{L^q}
	\]

	La primera igualdad viene por la propiedades del escalamiento en el sentido de la distribución de probabilidad. Luego, se aplica homogeneidad de la norma $\lVert \cdot \rVert_{L^q}$.

	Aplicamos la \textit{desigualdad de Burkhölder-David-Gundy} dos veces. Así, tendremos:

	\begin{align*}
		\mathbb{E}[ \lvert \mathbb{B}_{s,t} \rvert^{q/2} ] &= \mathbb{E} \left[ \left\lvert \int_s^t B_{s,r} \otimes dB_r \right\rvert \right] \\
		% ==================================
		&\leq C_q \mathbb{E} \left[ \left\lvert \int_s^t \lvert B_{s,r} \rvert^2 dr \right\rvert^{q/4} \right] \\
		% ==================================
		&\leq C_q \mathbb{E}\left[  \sup_{r \in [s,t]} \lvert B_{s,r} \rvert^{q/2} \right] \lvert t- s \rvert^{q/4} \\
		% ==================================
		&\leq C_q^2 \lvert t - s \rvert^{q/2}
	\end{align*}

	Note que este es un estimador de la norma $\lVert \cdot \rVert_{L^{q/2}}$. Así, tenemos que:

	\[
		\lVert \mathbb{B}_{s,t} \rVert_{L^{q/2}} \leq \left( C_q^2 \lvert t- s \rvert^{q/2} \right)^{2/q} = C_q^{4/q} \lvert t- s\rvert
	\]
 

\end{itemize}

De esta forma, como se cumplen las hipótesis, tomando $\beta = \frac{1}{2}$, por el criterio de Kolmogorov (Caso particular), haciendo $q \rightarrow \infty$, para todo $\alpha \in \left( \frac{1}{3}, \frac{1}{2} \right)$, existe una modificación $(\tilde{B}, \tilde{\mathbb{B}}) \in \mathcal{C}^{\alpha} \times \mathcal{C}^{2\alpha}_2$,y queda  demostrado lo deseado.

\begin{flushright}
	$\Box$
\end{flushright}

Una pregunta natural sería verificar si $(B, \mathbb{B})$ es un camino rugoso geométrico, esto es, si cumple el teorema de integración por partes. Verificando por la fórmula de Itô (Ejercicio 4.3), tendremos:

\[
	\frac{1}{2} B_{s,t} \otimes B_{s,t} = \text{Sym}(\mathbb{B}_{s,t}) + \left[\langle B^i, B^j \rangle_{s,t} \right]_{ij}
\]

Esto es, está el término de variación cuadrática adicional, por ende, no puede ser un camino rugoso geométrico :(, y no puede ser aproximable por una sucesión de caminos suaves.

% ======================== OPTATIVO

\subsection{Integral de Stratonovich}

Dados $X$, $Y$ semimartingalas continuas, puede definir la integral de Stratonovich como:

\[
	\int_0^t Y_s \circ dX_s = \int_0^t Y_s dX_s + \frac{1}{2} \langle Y, X \rangle_t
\]

Esta varianza cuadrática sale al seleccionar el punto medio, que es como se define originalmente la integral de Stratonovich. Esta satisface la fórmula de integración por partes:

\[
	X_t  Y_t = X_0 Y_0 + \int_0^t X_s \circ dY_s + \int_0^t Y_s \circ dX_s
\]

Esta puede ser una alternativa al levantamiento propuesto para el movimiento Browniano (Recuerde que acá la idea es ir proponiendo valores para el levantamiento y capturar más información del camino).

\[
	\mathbb{B}_{s,t}^{\text{Strat}} := \int_s^t B_{s,r} \otimes \circ dB_r
\]

para $(s,t) \in \Delta_{[0,T]}$. Note que, se puede demostrar que la pareja $(B, \mathbb{B}^{\text{Strat}})$ satisface la relación de Chen:

\begin{align*}
	\mathbb{B}_{s,t}^{\text{Strat}} &= \int_s^t B_{s,r} \otimes \circ dB_r \\
	&= \int_s^t B_{s,r} \otimes dB_r + \frac{1}{2} \langle B, B \rangle_{s,t}
\end{align*}

PENDIENTE HACER LA CUENTA!!!!

Podemos demostrar, que, una mejora respecto a la integral de Itô, el movimiento Browniano junto al levantamiento de Stratonovich, es un camino rugoso geométrico.

\begin{prop}
	Para $\alpha \in \left( \frac{1}{3}, \frac{1}{2} \right)$, tenemos que, casi siempre:

	\[
		(B, \mathbb{B}^{\text{Strat}}) \in \mathcal{C}_g^{0, \alpha} ([0,T]; \mathbb{R}^d)		
	\]

\end{prop}

\textbf{Demostración:} Primero, sea la función $A: (s,t) \mapsto \frac{1}{2}(t - s)I$.  Verificar que es $1$-Hölder continua:

\[
	\lvert A_{s,t} \rvert = \frac{1}{2} \lvert (t-s)I \rvert = \frac{1}{2} (t-s) \leq (t-s)
\]

asumiendo que $\lvert \cdot \rvert$ es una norma matricial, que $\lvert I \rvert = 1$. Ahora, como:

\[
	\mathbb{B}_{s,t}^{\text{Strat}} = \mathbb{B}_{s,t} + \frac{1}{2} (t - s) I
\]

Por la proposición anterior, podemos ver:

\begin{align*}
	\lVert  \mathbb{B}^{\text{Strat}} \rVert_{\alpha} &= \left\lVert \mathbb{B}_{s,t} + \frac{1}{2} (t - s) I \right\rVert_{ \alpha } \\
	&\leq \lVert \mathbb{B}_{s,t} \rVert_{\alpha} + \left\lVert \frac{1}{2}(t-s)I \right\rVert_{\alpha} \quad \text{Usar que } \mathbb{B}_{s,t} \text{ es regular 2-alpha } \\
	&\leq C_{2\alpha} \lvert t - s \rvert^{2\alpha} + C_{1} \lvert t - s \rvert \leq \infty
\end{align*}

por ende, la regularidad del camino rugoso $(B, \mathbb{B}^{\text{Strat}})$ se hereda por $(B, \mathbb{B})$, y de este modo $(B, \mathbb{B}^{\text{Strat}}) \in \mathcal{C}^{\alpha}$.

Por ende, tomando $\beta \in \left( \alpha, \frac{1}{2} \right)$, entonces, por las inclusiones dadas (Al inicio) y que $(B, \mathbb{B}^{\text{Strat}}) \in \mathcal{C}_g^{\beta}$ (El movimiento Browniano es un camino rugoso geométrico débil), entonces $(B, \mathbb{B}^{\text{Strat}}) \in \mathcal{C}^{0,\alpha}$ (Esto es, el movimiento geométrico).

\begin{flushright}
	$\Box$
\end{flushright}


Note que, en ambos levantamientos (Itô y Stratonovich), la parte simétrica ya es conocida (Aplicando las propiedades de las integrales estocásticas). Ahora, observando la parte antisimétrica, que es más información codificada por el levantamiento del camino rugoso, $\mathbb{X}_{s,t}$, está dada por:

\[
	\text{Anti}(X_{s,t})^{ij} = \frac{1}{2} \left( \int_s^t X_{s,r}^i dX_r^j - \int_s^t X_{s,r}^j dX^i_r \right)
\]

esta se conoce como \textit{el área de Lévy}, que es un proceso estocástico que modela el área entre una trayectoria del movimiento Browniano y la cuerda.


% ==============================================000
% Integración, Lema de Costura, Integración de Young.
% ===================================================


\section{Integración Rugosa}

% CONSULTAR NOTAS SERGIO NÚÑEZ!!!!

La idea de esta sección, es comenzar a trabajar acerca de la integración respecto a un camino rugoso. Primero, se va a ver un lema, que nos permite obtener unas estimaciones de dos trayectorias que aproximan a una función continua.

\begin{theorem}[Lema de Costura]

	Sea $(E, \lVert \cdot \rVert)$ un espacio de Banach (Espacio vectorial normado completo), y sea $A: \Delta_{[0,T]} \rightarrow E$ una función continua. Para cada tripla $0 \leq s \leq u \leq t \leq T$, sea $\delta A_{s,u,t} := A_{s,t} - A_{s,u} - A_{u,t}$. Suponga que existe constantes $\lambda \leq 0$ y $\epsilon > 0$ tal que:

	\[
			\lVert \delta A_{s,u,t} \rVert \leq \lambda \lvert t - s \rvert^{1 + e}
	\]

	para todo $0 \leq s \leq u \leq t \leq T$. ($\delta A_{s,u,t}$ indica la diferencia entre el área total y el área particionada. La norma es un estimador del error).

	Entonces, existe un camino continuo $\gamma: [0,T] \rightarrow E$, con $ \gamma_0 = 0$, tal que:

	\[
		\lVert \gamma_t - \gamma_s - A_{s,t} \rVert \leq C \lambda \lvert t - s \rvert^{1 + \epsilon}
	\]

	para todo $(s,t) \in \Delta_{[0,T]}$, donde las constantes $C$ depende sólo de $\epsilon$. Más aún, para todo $(s,t) \in \Delta_{[0,T]}$, tenemos que:

	\[
		\lim_{\lvert \pi \rvert \rightarrow 0} \sum_{  [u,v] \in \pi } A_{u,v} = \gamma_t - \gamma_s 
	\]

	donde el límite es tomado sobre un descuento de partición $\pi$ del intervalo $[s,t]$ con el tamaño de malla $\lvert \pi \rvert \rightarrow 0$.

\end{theorem}

Este teorema nos permite dar una aproximación a una función de dos parámetros, con dos funciones de un parámetro.

\textbf{Demostración:} Sea $(s,t) \in \Delta_{[0,T]}$. Para cada $n \leq 0$, sea $\{ s = t_0^n < t_1^n < \cdots < t_{2^n}^n  = t  \}$ una partición, tal que $t_i^n = s + \frac{i}{2^n}(t - s)$, que tiene un tamaño de malla $\lvert \pi^n \rvert = \lvert t_{i + 1}^n - t_i^n \rvert = 2^{-n} \lvert t - s \rvert $. Sea:

\[
	A_{s,t}^n = \sum_{i = 0}^{2^n - 1} A_{ t_i^n, t_{i+1}^n }
\]

Para cada $n$, y cada $i$, sea $u_i^n$ el punto medio del intervalo $[t_i^n, t_{i+1}^n]$, esto es, $u_i^n = \frac{t_{i+1}^n - t_i^n}{2}$. Tenemos que:

\begin{align*}
	A^n_{s,t} - A^{n + 1}_{s,t} &= \sum_{i = 0}^{2^n - 1} A_{t_i^n,t_{i+1}^n} - \sum_{i = 0}^{2^{n+1} - 1} A_{t^n_i, t^n_{i + 1}} \\
	&= \sum_{i = 0}^{2^n - 1} A_{t_i^n, t_{i+1}^n} - A_{t_i^n, u_i^n} -  A_{u_i^n, t_{i+1}^n} \text{, ¿Cómo sale esta cuenta? PENDIENTE} \\
	&= \sum_{n = 0}^{2^n - 1} \delta A_{t_i^n, u_i^n, t_{i+1}^n}
\end{align*}

y por ende;

\begin{align*}
	\lVert  A_{s,t}^n - A_{s,t}^{n+1} \rVert &\leq \sum_{i = 0}^{2^n - 1} \lVert \delta A_{ t_i^n, u_i^n, t_{i+1}^n } \rVert, \text{ desigualdad triangular. Norma del Espacio Banach.} \\
	&\leq \sum_{i = 0}^{2^n - 1} \lambda \lvert  t_{i + 1}^n - t_i^n \rvert^{1+ \epsilon} \text{ hipótesis. } \\
	&= \lambda \lvert t - s \rvert^{1 + \epsilon} \sum_{i = 0}^{2^n - 1} 2^{-n(1 + \epsilon)} \text{ , la longitud de la partición } \\
	&= \lambda \lvert t - s \rvert^{1 + \epsilon} 2^{ -n\epsilon} \text{ sacar término de la sumatoria, queda suma de unos, da } 2^n.
\end{align*}


Luego, con esta estimación:

\[
	\sum_{n = k}^{\infty} \lVert A_{s,t}^n - A_{s,t}^{n+1} \rVert \leq \lambda \lvert t - s \rvert^{1 + \epsilon} \sum_{n = k}^{\infty} 2^{-n\epsilon} = \lambda \lvert t - s \rvert^{1 + \epsilon} \frac{2^{-k \epsilon}}{1 - 2^{-\epsilon}} \rightarrow 0
\]

para $k \rightarrow 0$. De esto podemos deducir que $\{ A_{s,t}^n \}_{n \geq 0}$ es una sucesión de Cauchy. Como $A^n$ es un espacio de Banach, esta sucesión de Cauchy converge, y denotamos al límite:

\[
	\Gamma_{s,t} = \lim_{n \rightarrow \infty} A^n_{s,t}
\]
que sabemos que existe. Luego:

\begin{align*}
	\lVert \Gamma_{s,t} - A_{s,t}^k \rVert &= \left\lVert \sum_{ n = k }^{ \infty } (A_{s,t}^n - A_{s,t}^{n+1}) \right\rVert \\
	&\leq \lambda \lvert t - s \rvert^{1 + \epsilon} \frac{2^{-k\epsilon}}{1 - 2^{-\epsilon}}  
\end{align*}

y por ende, la convergencia es uniforme en $(s,t) \in \Delta_{[0,T]}$, y más aún:

\begin{align*}
	\lVert \Gamma_{s,t} - A^n_{s,t} \rVert \leq \frac{\lambda \lvert t - s \rvert^{1 + \epsilon}}{ 1 - 2^{ -\epsilon} }
\end{align*}

Como $A^n$ es continua, entonces, $\Gamma$ es continua. Note que:

\[
	\Gamma_{s,u} + \Gamma_{u,t} = \lim_{n \rightarrow \infty} A_{s,u}^n + \lim_{n \rightarrow \infty } A_{u,t}^n = \lim_{n \rightarrow \infty} A_{s,t}^n = \Gamma_{s,t}
\]

por porque $A_{s,u}^n + A_{u,t}^n = A_{s,t}^n$. ¿DUDAS DE LOS ESTIMADORES DE ARRIBA?

Esto se tiene para todas las particiones diádicas $s \leq u \leq t$, y por la continuidad (El límite de una sucesión es un punto de acumulación, y las particiones diádicas son un conjunto denso en los reales), se tiene para todo los intervalos de tiempo $s \leq u \leq t$. Note que, $\Gamma$ son los incrementos de un camino continuo (¿Por qué?). Esto es, para $\gamma_t = \Gamma_{0,t}$ para $t \in [0,T]$, entonces tenemos que:

\[
	\gamma_t - \gamma_s = \Gamma_{s,t}, \quad \text{Para todo} (s,t) \in \Delta_{[0,T]}
\]

y en particular, que $\gamma_0 = 0$. Podemos reescribir la desigualdad de arriba como:

\[
	\lVert \gamma_t - \gamma_s - A_{s,t} \rVert \leq \frac{ \gamma \lvert t - s \rvert^{ 1 + \epsilon } }{1 - 2^{-\epsilon}}
\]

de tal forma que, haciendo $C = \frac{1}{1 - 2^{\epsilon}}$, tenemos la primera desigualdad. Finalmente, para cualquier $(s,t) \in \Delta_{[0,T]}$ y cualquier partición $\pi = \left\{  s = t_0 < t_1 < \cdots < t_N = t \right\}$ de $[s,t]$, tenemos:

\begin{align*}
	\left \lVert \gamma_t - \gamma_s - \sum_{ i = 0 }^{N-1} A_{t_i, t_{i+1}} \right\rVert &= \left \lVert \sum_{i = 0}^{N - 1} \gamma_{t_{i+1}} - \gamma_{t_i} - A_{t_i, t_{i+1}} \right \rVert \\
	&\leq \frac{\lambda}{1 - 2^{-\epsilon}} \sum_{ i = 0 }^{N-1} \lvert t_{i+1} - t_i \rvert^{1 + \epsilon} \text{, por la última desigualdad.}\\
	&\leq \frac{\lambda}{1 - 2^{-\epsilon}} \lvert t - s \rvert \lvert \pi \rvert^{\epsilon}
\end{align*}

de tal forma que $\sum_{i = 0}^{N - 1} A_{t_i, t_{i+1}} \rightarrow \gamma_t - \gamma_s$ para $\lvert \pi \rvert \rightarrow 0$.

¡¡¡REVISAR DEMOSTRACIÓN!!!

\begin{flushright}
	$\Box$
\end{flushright}

\subsection{Apuntes del Martin Hairer}

Problema de dar significado a la expreión $\int Y_t dX_t$ para $X \in \mathscr{C}^{\alpha} ([0,T], V)$ y $Y$ alguna función continua con valores en $\mathcal{L}(V,W)$ ($Y$ será un \textit{camino controlado}), este es el espacio de operadores lineales acotados de $V$ a otro espacio de Banach $W$. 

¿Qué funciones harán que la integral sea definida? Además, toca que:

\[
	(X,Y) \mapsto \int Y dX
\]

sea continua en topologías relevantes. ¿Cuáles son esos buenos integrandos para el camino rugoso $X$?. Integral de Young, tomando caminos $\alpha$ y $\beta$ tal que $\alpha + \beta > 1$, el operador es continuo, la integral converga! :D. Desigualdad de Young. 

¿Y qué obtenemos con los caminos rugosos? Romper esa barrera de los $\alpha + \beta > 1$, al obtener una mayor estructura del problema. Postular valores para $\mathbb{X}$ de la integral de $''\int X dX ''$. Más aún, ver a $\int Y dX$ cuando $Y$ es parecida a $X$, al menos en pequeñas escalas.

¿Y cómo? Teniendo a $Y_t = F(X_t)$ para alguna función suave $F: V \rightarrow \mathcal{L}(V,W)$, llamada \textbf{1-forma}.


1-FORMA -> GEOMETRÍA DIFERENCIAL. (Como $f(x) dx$).

\subsubsection{Integración en 1-Formas}

Integrar $Y = F(X)$ en contra de $\mathbf{X} = (X, \mathbb{X}) \in \mathscr{C}^{\alpha}$. Para $F: V \rightarrow \mathcal{L}(V,W)$ en $\mathcal{C}^1$, una aproximación en series de Taylor da

\[
	F(X_r) \approx F(X_s) + DF(X_s) X_{s,r}
\]







% Caminos Controlados (Ejemplos) e Integración Rugosa. 



% OPCIONALES. Comentarios acerca Hairer & Friz... y Lyons.

\section{Comentarios adicionales*.}

\subsection{Movimiento Browniano como camino rugoso.}

Algunos comentarios, teoremas, conceptos... relacionados al capítulo, posiblemente no se incluyan en la versión final del texto.

\begin{prop}
	Considere aproximaciones lineales a trozos diádicas $\left\{ B^{n} \right\}$ a $B$ en $[0,T]$. Esto es, $B_t^n = B_t$ para $t = \frac{iT}{2^n}$ para algún entero $i$ (Se interpola en intervalos de la forma $\left[ \frac{iT}{2^n}, \frac{(i+1)T}{2^n} \right]$). Luego, en el sentido de la probabilidad:

	\[
		\left( B^n, \int_0^t B^n \otimes dB^n \right) \rightarrow (B, \mathbb{B}^{\text{Strat}})
	\]

	en $\mathcal{C}^{\alpha}_g$.

\end{prop}

ESto es, que en el sentido de la probabilidad (Y en $L^q$ con $q < \infty$), el camino rugoso interpolado, converge al camino rugoso del movimiento Browniano con el levantamiento de Stratonovich, con el criterio de Kolmogorovo paa la métrica rugosa. Más aún, este enfoque da cierta tasa $\theta < \frac{1}{2} - \alpha$. 
 

\textbf{Demostración:} Tomando partición diádica. Sea $B^n$ definida como:

\[
	B^n = \mathbb{E}(B \quad \vert \quad \sigma\{ B_{kT 2^{-n}}  : 0 \leq k \leq 2^n  \})
\]

Porque el movimiento Browniano es una martingala. Por la independencia de $B^i$, $B^j$ para $i \neq j$, se cumple para $\mathbb{B}^{\text{Strat}}$ fuera de la diagonal (¿?). Para los elementos de la diagonal $\mathbb{B}^{\text{Strat}, (i,i)}_{s,t} = \frac{1}{2} (B_{s,t}^i)^2$. Por convergencia de martingala, se tiene la convergencia deseada.

Más aún, por el criterio de Kolmogorov, tenemos:

\[
	\lvert B^i_{s,t} \rvert \leq K_{\alpha} (\omega) \lvert t - s \rvert^{\alpha}, \qquad \lvert \mathbb{B}_{s,t}^{\text{Strat}, (i,j)} \rvert \leq \mathbb{K}_{\alpha} (\omega) \lvert t - s \rvert^{2\alpha}
\]

y la probabilidad condicional respecto a $\sigma \left\{ B_{kT2^{-n}} : 0 \leq k \leq 2^n \right\}$, se tienen las mismas cotas para $B^{n,i}$ y para $\int_0^{.} B^{n,i} dB^{n,j}$. De hecho, $K_{\alpha}$, $\mathbb{K}_{\alpha}$ tienen la suficiente integrabilidad para aplicar la \textit{desigualdad maximal de Doob}. Esto nos dará que, con probabilidad 1:

\[
	\sup_n \left\lVert B^{(n)}, \int_0^{.} B^{(n)} \otimes dB^{(n)} \right\rVert < \infty
\]

\begin{flushright}
	$\Box$
\end{flushright}


\subsection{Apuntes en el Lyons acerca de los caminos rugosos.}


\subsection{Geometría de Carnot-Caratheodory}

Se desea obtener un poco más de conocimiento acerca de los caminos rugosos, por lo que se proceden a usar más integrales iteradas. Dado $x \in \mathcal{C}^{\infty} ([0,T], \mathbb{R}^d)$, generalizar:

\[
	\mathbf{x}_t := S_n (x)_{0, t} := \left( 1, \int_0^t dx, \int_{\Delta_{[0,t]}^2 } dx \otimes dx, \cdots, \int_{ \Delta_{[0,t]}^N  } dx \otimes \cdots \otimes dx \right)
\]

que es la \textbf{signatura/firma de paso $N$} de $x$ sobre $[0,t]$, dado $x$ el control o la trayectoria. Toma valores de las álgebras tensoriales truncadas (¿de Lie?):

\[
	T^N (\mathbb{R}^d) := \mathbb{R} \oplus \mathbb{R}^d \oplus \left( \mathbb{R}^d \right)^{\otimes 2} \oplus \cdots (\mathbb{R}^d)^{\otimes N}
\]

Estructura de álgebra tensorial con producto tensorial $\otimes$, usando la base estándar $\left\{ e_1, \cdots, e_d \right\}$ de $\mathbb{R}^d$. Esta álgebra se usa con el fin de extender:

\[
	x_{s,t} \equiv (-x_s) + x_t = \int_s^t dx =: x_{s,t}
\]


a:

\[
	\mathbf{x}_{s,t} \equiv \mathbf{x}_s^{-1} \otimes \mathbf{x}_t = \left( 1, \int_s^t dx, \int_{ \Delta_{[s,t]}^2 } dx \otimes dx, \cdots, \int_{ \Delta_{[s,t]}^N } dx \otimes \cdots \otimes dx \right)
\]

El cuál nos deja como enseñanza la \textit{relación de Chen}, qué nos indica como se pueden \textit{unir} integrales iteradas sobre intervalos adyacentes. 

Note que, el levantamiento de $N$-pasos (La signatura) del camino suave $x$, toma valores en el grupo de Lie de paso libre $N$ nilpotente con $d$ generadores, domo restricción de $T^N (\mathbb{R}^d)$ a 

\[
	G^N (\mathbb{R}^d) = \exp \left( \mathbb{R}^d \oplus \left[ \mathbb{R}^d, \mathbb{R}^d \right] \oplus \left[ \mathbb{R}^d, \left[ \mathbb{R}^d, \mathbb{R}^d \right] \right] \oplus \cdots \right) \equiv \exp( \mathfrak{g}^N (\mathbb{R}^d) )
\]

¿Qué es $\mathfrak{g}^N (\mathbb{R}^d)$? es una álgebra de Lie nilpotente de N-pasos libres, y $\exp$ se define usando la serie basada en $\otimes$.

\begin{boxDef}
	Un \textbf{álgebra de Lie} es un espacio vectorial $\mathfrak{g}$ equipada con un mapeo bilineal alternante $[\cdot, \cdot] $(Esto es, si $x_1, x_2 \in \mathfrak{g}$ son linealmente dependientes, $[x_1, x_2] = 0$, note que esto es \"equivalente\" a decir que si los argumentos son iguales, es igual a 0 por la bilinealidad). Este mapa se conoce como \textbf{corchete de Lie}. \\

	Sea ahora $X$ un conjunto y $i: X \rightarrow L$ un morfismo de conjuntos (.\_.) de $X$ a $L$ que es un álgebra de Lie. El álgebra de Lie $L$ se dice \textbf{libre en X}, si $i$ es el morfismo universal (Para toda álgebra de Lie $A$ con $f: X \rightarrow A$ morfismo, existe un único morfimo de álgebras de Lie $g: L \rightarrow A$ tal que $f = g \circ i$, todo morfismo será composición del morfismo con el morfismo universal). \\

	\textbf{Álgebra de Lie nilpotente}, serie central baja.
\end{boxDef}


...





