\chapter{Caminos Rugosos e Integración Rugosa}

La teoría de caminos rugosos (Vea, por ejemplo \cite{Rough_Paths_TL})  permite extender una teoría de ecuaciones diferenciales controladas, para poder trabajar el caso al tener una señal de entrada que sea \textit{ruidosa}, tal como una \textit{semimartingala}, como lo es el movimiento Browniano. Sea una ecuación diferencial controlada:

\[
	d Y_t = f(Y_t) dX_t, \quad Y_0 = \zeta
\]

Si la ecuación determinística admite solución única, entonces se denota como $Y = I_f(X, \zeta)$, y se conoce a $I_f$ como \textbf{mapeo de Itô asociado a $f$}. Los caminos rugosos se plantean como solución a:

Problemas con integral de Itô, motivación a la integración rugosa.

Espacios de Hölder, Propiedades.
Colocar simulaciones.

Caminos Rugosos, Caminos Rugosos Geométricos (Interpretación prof. Freddy).

Movimiento Browniano como ejemplo a camino rugosos.

Integración, Lema de Costura, Integración de Young.

Caminos Controlados (Ejemplos) e Integración Rugosa.