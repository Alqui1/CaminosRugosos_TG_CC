\chapter{Ecuaciones Diferenciales Rugosas}

Vea el siguiente ejemplo...

PENDIENTES!!!

\begin{itemize}
	\item  \textbf{Capítulo 1:} Limpiar. Teorema de Peano. Gráficas $\alpha$-Hölder
	\item \textbf{Capítulo 2:} Limpiar. Propiedades MB.
	\item \textbf{Capítulo 3:} Lectura Oksendal. CH1, CH3, CH4, CH5. Enfocarnos en ejemplos, integral de Itô y EDE. Ejemplos, suprimir algunas demostraciones.
	\item \textbf{Capítulo 4:} Limpiar (Verificar no repetición). Gráficas Lema de Costura.
	\item \textbf{Capítulo 5:} ¿CH6?, CH7 y CH8. Suprimir algunas demostraciones. Gráficas.  
	\item \textbf{Apéndice.} Tensores, ¿CH6?... no sé si hace falta algo más xd.
\end{itemize}



\section{Ejemplos de Ecuaciones Diferenciales Rugosas.}

Las ecuaciones diferenciales rugosas, se pueden ver como una generalización de las ecuaciones diferenciales controlados. Cuando tenemos un control, cuya regularidad es $\alpha$-Hölder con $\alpha \in \left( \frac{1}{3}, \frac{1}{2} \right]$, podemos inducir un levantamiento:

\[
	dY_t = f(Y_t) dX_t \rightarrow  dY_t = f(Y_t) d\mathbf{X}_t
\]

donde $\mathbf{X}_t = (X_t, \mathbb{X}_t)$. Al escribir el problema de forma integral, se tendrá definida la integral rugosa, que se vió en capítulos anteriores. \\

Antes de pasar a los teoremas de existencia y unicidad, primero se estudian algunos ejemplos.

\begin{itemize}

	\item Dado $\alpha \in \left( \frac{1}{3}, \frac{1}{2} \right]$, y $\mathbf{X} \in \mathscr{C}^{\alpha}$. La siguiente ecuación diferencial rugosa:

	\[
		V_t = 1 + \int_0^t V_u d\mathbf{X}_u
	\]

	tiene como solución única:

	\[
		V_t = \exp \left( X_t - \frac{1}{2} \left[ \mathbf{X} \right]_t \right)
	\]

	y se conoce como la \textbf{exponencial rugosa}. El corchete $[\mathbf{X}]_t$ se conoce como \textbf{corchete de camino rugoso}, y para verificar la solución, se usa la fórmula de Itô para caminos rugosos (Ver el apéndice).

	\item Una ecuación similar a la anterior, está dada por:

	\[
		V_t = 1 + \int_0^t V_u K_u d\mathbf{X}_u
	\]

	En este caso, poseemos un camino controlado $(K, K') \in \mathscr{D}^{2\alpha}_X$, donde $\int_0^{\cdot} K_u d\mathbf{X}_u$ toma valores reales. La solución está dada por:

	\[
		V_t = \exp \left( \int_0^t K_u d\mathbf{X}_u - \frac{1}{2} \int_0^t (K_u \otimes K_u) d[\mathbf{X}]_u  \right)
	\]

\end{itemize}





\section{Ecuaciones Diferenciales de Young*}



\section{Funciones de caminos controlados y ecuaciones diferenciales rugosas.}

Antes de enunciar el teorema más poderoso e interesante de la teoría de caminos rugosos, se van a enunciar unos resultados usados en la demostración, que involucran a las funciones de caminos controlados (Que, son los integrandos trabajados en la integral rugosa). \\

\begin{lema}
	Sea $\alpha \in \left( \frac{1}{3}, \frac{1}{2} \right]$ y $\mathbf{X} \in (X, \mathbb{X}) \in \mathscr{C}^{\alpha}$. Sea $f \in C^2_b$. Para $(Y, Y') \in \mathscr{D}^{2\alpha}_X$, la pareja:

	\[
		\left(  \int_0^{\cdot} f(Y_u) d\mathbf{X}_u, f(Y)  \right) \in \mathscr{D}^{2\alpha}_X
	\]

	es un camino controlado. Más aún, tenemos los estimadores para la derivada de Gubinelli y el residuo:

	\begin{align*}
		\lVert f(Y) \rVert_{\alpha} &\leq C \left(  ( \lvert Y'_0 \rvert + \lVert Y' \rVert_{\alpha} )  \right \lVert X \rVert_{\alpha} + \lVert R^Y \rVert_{2\alpha} T^{\alpha}  ) \\
		\lVert R^{ \int_0^{\cdot} f(Y_u) d\mathbf{X}_u } \rVert_{2\alpha} &\leq C \left(  1 + \lvert Y'_0 \rvert + \lVert Y' \rVert_{\alpha} + \lVert R^Y \rVert_{2\alpha} \right)^{2} (1 + \lVert X \rVert_{\alpha})^2 \lVert \lvert X \rvert \rVert_{\alpha}
	\end{align*}

	donde $C$ depende de $\alpha$, T y $\lVert f \rVert_{C^2_b}$.

\end{lema}

El teorema anterior, se puede interpretar de forma intuitiva, al tomar la trayectoria $\mathbf{X}$ como una trayectoria lo suficientemente suave, cuya integral $\int_0^{\cdot} f(Y_u) dX_u$ exista en el sentido de Riemann-Stieljes. Así, al usar el teorema fundamental del cálculo, la derivada de Gubinelli (O la derivada usual, en este caso), estará dado por $f(Y)$.

\begin{lema}
	Sea $\alpha \in \left( \frac{1}{3}, \frac{1}{2} \right]$ y $\mathbf{X} = (X, \mathbb{X}),  \tilde{ \mathbf{X} } = (\tilde{X}, \tilde{ \mathbb{X} }) \in \mathscr{C}^{\alpha} $. Sea $(Y, Y') \in \mathscr{D}^{2\alpha}_X$, $(\tilde{Y}, \tilde{Y'}) \in \mathscr{D}^{2\alpha}_{ \tilde{X} }$ y $f \in C^3_b$. Sea $M > 0$ una constante tal que $\lVert X \rVert_{\alpha} \leq M$, $\lVert \tilde{X} \rVert_{\alpha} \leq M$, $\lvert Y'_0 \rvert + \lVert Y' \rVert_{\alpha} + \lVert R^Y \rVert_{2\alpha} \leq M$ y $\lvert \tilde{Y}'_0 \rvert + \lVert \tilde{Y}' \rVert_{\alpha} + \lVert R^{ \tilde{Y} } \rVert_{2\alpha} \leq M$. Entonces:

	\begin{align*}
		&\lVert f(Y) - f( \tilde{Y} ) \rVert_{\alpha} \\
		&\leq C \left(  \lvert Y_0 - \tilde{Y}_0 \rvert + ( \lvert Y'_0 - \tilde{Y}'_0 \rvert +  \lVert Y' - \tilde{Y}' \rVert_{\alpha}  )  \lVert X \rVert_{\alpha}  + \lVert R^Y - R^{\tilde{Y}} \rVert_{2\alpha} T^{\alpha} + \lVert X - \tilde{X} \rVert_{\alpha}   \right)
	\end{align*}

	y

	\begin{align*}
		&\lVert R^{ \int_0^{\cdot} f(Y_u) d\mathbf{X}_u  } - R^{ \int_0^{\cdot} f( \tilde{Y}_u ) d\tilde{ \mathbf{ {X} }}_u }  \rVert_{2\alpha} \\
		&\leq C \left(  ( \lvert Y_0 - \tilde{Y}_0 \rvert + \lvert Y'_0 - \tilde{Y}'_0 \rvert  + \lVert  Y' - \tilde{Y}' \rVert_{\alpha} + \lVert  R^Y - R^{\tilde{Y}}  \rVert_{2\alpha}  + \lVert  X - \tilde{X} \rVert_{\alpha} )  \lVert \lvert \mathbf{X} \rvert \rVert_{\alpha} + \right. \\
		&\left. \lVert \mathbf{X}; \tilde{\mathbf{X}} \rVert_{\alpha}  \right)
	\end{align*}


\end{lema}


Este lema muestra, la posibilidad de acotar dos funciones de caminos controlados y dos residuos, usando las trayectorías controladas, lo que, de forma preliminar, va a generar una dependencia a los datos iniciales.

\subsection{Existencia y unicidad en ecuaciones diferenciales rugosas.}

Dado un campo vectorial $f$, lo suficientemente regular. Sea la ecuación diferencial rugosa:

\[
	dY_t = f(Y_t) d\mathbf{X}_t
\]

El siguiente teorema, indica la existencia y unicidad de la solución de ecuaciones de este tipo.

\begin{theorem}
	Sea $\alpha \in \left( \frac{1}{3}, \frac{1}{2} \right]$, y sea $\mathbf{X} = (X, \mathbb{X}) \in \mathscr{C}^{\alpha} ([0,T], \mathbb{R}^d)$ un camino rugoso. Sea $f \in C^3_b ( \mathbb{R}^m; \mathscr{L}( \mathbb{R}^d; \mathbb{R}^m ) )$, y $y \in \mathbb{R}^m$. Entonces, existe una solución única, un camino controlado $(Y, Y') \in \mathscr{D}^{2 \alpha}_X$, tal que $Y' = f(Y)$, y tal que:

	\[
		Y_t = y + \int_0^t f(Y_s) d\mathbf{X}_s	
	\]
	para todo $t \in [0,T]$

\end{theorem}


La demostración es similar al teorema de Picard. Usando una iteración de punto fijo en un espacio de Banach adecuado.

\textbf{Demostración:} Sea $\alpha \in \left( \frac{1}{3}, \frac{1}{2} \right]$. Defina el mapa $\mathcal{M}_t : \mathscr{D}^{2\alpha}_X ([0,t]; \mathbb{R}^m) \mapsto \mathscr{D}^{2\alpha}_X ([0,t]; \mathbb{R}^m)$, dado por:

\[
	\mathcal{M}_t (Y, Y') = \left( y + \int_0^{\cdot} f( Y_s ) d\mathbf{X}_s, f(Y)  \right)
\]


que por el resultado presentado en esta sección, es un camino controlado. Sea:

\[
	\mathcal{B}_t = \left\{  (Y, Y') \in \mathscr{D}^{2\alpha}_X ([0,t]; \mathbb{R}^m) : Y_0 = y, Y'_0 = f(y), \lVert Y, Y' \rVert_{ X, \alpha, [0,t] } \leq 1 \right\}
\]

donde, se define la seminorma de trayectorias controladas como:

\[
	\lVert Y, Y' \rVert_{X, \alpha} = \lVert  Y' \rVert_{\alpha} + \lVert R^Y \rVert_{2\alpha}
\]

Como $\mathcal{B}_t$ es un subconjunto cerrado del espacio de Banach $\mathscr{D}^{2\alpha}_X$, entonces es un espacio métrico inducido con la norma $\lVert \cdot, \cdot \rVert_{X, \alpha}$. También, como la trayectoría $s \mapsto (y + f(y)X_{0,s}, f(y)) \in \mathcal{B}_t$, el conjunto es no vacio.

Sea $M = $. Note que, si $(Y, Y') \in \mathscr{B}_t$, luego $\lvert Y'_0 \rvert + \lVert Y' \rVert_{\alpha} + \lVert R^Y \rVert_{2 \alpha} = \lvert f(y) \rvert + \lVert Y, Y' \rVert_{X, \alpha} \leq 1 + \lVert f \rVert_{C^3_b} + \lVert X \rVert_{ \alpha, [0,T] } $, y se cumple el lema anterior (CITAR!!). Con estas herramientas, se mostrará que $\mathcal{M}_t$ es una aplicación contractiva, para luego aplicar el teorema del punto fijo de Banach. De ahora en adelante, se tiene que $\lesssim$ dependiendo de una constante multiplicativa, que depende de $M = 1 + \lVert f \rVert_{C^3_b} + \lVert X \rVert_{ \alpha, [0,T] }$:

\begin{itemize}
	\item $\mathbf{ \mathcal{M}_t }$ \textbf{es invariante}. Esto es, que va de un espacio, al mismo espacio. Sea $(Y, Y') \in \mathcal{B}_t$. Por el lema 7.6 CITAR, tenemos:

	\begin{align*}
		\lVert \mathcal{M}_t (Y, Y') \rVert_{X, \alpha} =& \lVert f(Y) \rVert_{\alpha} + \lVert R^{\int_0^{\cdot} f(Y_s) d\mathbf{X}_s }  \rVert_{2\alpha} \\
		&\lesssim ( \lvert Y'_0 \rvert + \lVert Y' \rVert_{\alpha}  )\lVert X \rVert_{\alpha} + \lVert R^Y \rVert_{2\alpha} t^{\alpha} \\
		&+ (1 + \lvert Y'_0 \rvert + \lVert Y' \rVert_{\alpha} + 
		 \lVert  R^Y \rVert_{2\alpha})^{2} (1 + \lVert X \rVert_{\alpha})^2 \lVert \lvert \mathbf{X} \rvert \rVert_{\alpha} \\
		&\lesssim  \lVert \lvert \mathbf{X} \rvert \rVert_{\alpha} + t^{\alpha}
	\end{align*}

	y así, para alguna constante $C_1$, se tiene:

	\[
		\lVert \mathcal{M}_t (Y,Y') \rVert_{ X, \alpha, [0,t] } \leq C_1 \left(  \lVert  \lvert \mathbf{X} \rvert  \rVert_{\alpha, [0,t]} + t^{\alpha} \right)
	\]

	Entonces:

	\begin{align*}
		\lVert \mathcal{M}_t (Y,Y') \rVert_{X, \alpha, [0,t]} &\leq  C_1 ( \lVert X \rVert_{\alpha, [0,t]} + \lVert \mathbb{X} \rVert_{2\alpha, [0,t]} + t^{\alpha} ) \\
		&\leq C_1 ( \lVert X \rVert_{\beta, [0,t]} t^{\beta - \alpha} + \lVert \mathbb{X} \rVert_{2 \beta, [0,t]} t^{2 (\beta - \alpha)} + t^{\alpha} )
	\end{align*}

	Finalmente, tomando $t = t_1 > 0$, lo suficientemente pequeño, se tendrá que $\lVert \mathcal{M}_{t_1} (Y, Y') \rVert_{X, \alpha, [0, t_1]} \leq 1$ para todo elemento en $\mathcal{B}_{t_1}$. Por ende, ya se tiene la invarianza.


	\item $\mathbf{\mathcal{M}_t}$ \textbf{es una contracción.} Sea $(Y, Y'), ( \tilde{Y}, \tilde{Y}' ) \in \mathcal{B}_{t}$ para algún $t \in [0,t_1)$. Por el lema anterior (CITAR!), tenemos:

	\begin{comment}
	\begin{align*}
		\lVert  \mathcal{M}_t (Y, Y') - \mathcal{M}_t (\tilde{Y}, \tilde{Y}') \rVert_{X, \alpha} &=  \lVert f(Y) - f(\tilde{Y}) \rVert_{\alpha} + \lVert R^{\int_0^{\cdot} f(Y_s) d\mathbf{X}_s} - R^{\int_0^{\cdot} f( \tilde{Y}_s ) d \mathbf{ \tilde{X} }_s }  \rVert_{2 \alpha} \\
		&\lesssim ( \lVert Y' - \tilde{Y}' \rVert_{\alpha} + \lVert R^Y - R^{\tilde{Y}} \rVert_{2\alpha} )( \lVert \lvert \mathbf{X} \rvert \rVert_{\alpha} + t^{\alpha}  ) 
	\end{align*}
	\end{comment}

	% tal que

	\begin{align*}
		&\lVert \mathcal{M}_t (Y, Y') - \mathcal{M}_t (\tilde{Y}, \tilde{Y}')  \rVert_{X, \alpha, [0,t]} \\
		&\leq C_2 ( \lVert Y' - \tilde{Y}' \rVert_{\alpha, [0,t]} + \lVert R^Y - R^{\tilde{Y}} \rVert_{2\alpha, [0,t]} ) ( \lVert \lvert \mathbf{X} \rvert \rVert_{\alpha, [0,t]} + t^{\alpha} ) \\
		&\leq C_2 ( \lVert Y' - \tilde{Y}' \rVert_{\alpha, [0,t]} + \lVert R^Y - R^{\tilde{Y}} \rVert_{2\alpha, [0,t]} ) ( \lVert X \rVert_{\beta, [0,t]} t^{\beta - \alpha} + \lVert \mathbb{X} \rVert_{2\beta, [0,t]} t^{2(\beta - \alpha)} + t^{\alpha} )
	\end{align*}

	para alguna constante $C_2$ que depende de $M$. Entonces:

	\begin{comment}
	\begin{align*}
		&\lVert \mathcal{M}_t (Y, Y') - \mathcal{M}_t (\tilde{Y}, \tilde{Y}')  \rVert_{X, \alpha, [0,t]} \\
		&\leq C_2 ( \lVert Y' - \tilde{Y}' \rVert_{\alpha, [0,t]} + \lVert R^Y - R^{\tilde{Y}} \rVert_{2\alpha, [0,t]} ) ( \lVert X \rVert_{\beta, [0,t]} t^{\beta - \alpha} + \lVert \mathbb{X} \rVert_{2\beta, [0,t]} t^{2(\beta - \alpha)} + t^{\alpha} )
	\end{align*}
	\end{comment}
	
	Seleccione $t = t_2 \in (0, t_1]$ suficientemente pequeño, podemos garantizar que:

	\[
		\lVert \mathcal{M}_t (Y, Y') - \mathcal{M}_t (\tilde{Y}, \tilde{Y}')  \rVert_{X, \alpha, [0,t_2]} \leq \frac{1}{2} \lVert (Y, Y') - (\tilde{Y}, \tilde{Y}') \rVert_{X, \alpha, [0,t_2]}
	\]

	Y por ende, queda verificado que $\mathcal{M}_{t_2}$ es una contracción $\mathcal{B}_{t_2}$.

\end{itemize}

Entonces por el teorema del punto fijo de Banach, existe un único punto fijo, que es un único elemento $(Y, Y') \in \mathscr{D}^{2\alpha}_X$ de la ecuación diferencial rugosa en el intervalo $[0, t_2]$ que cumple $Y' = f(Y)$. Además, como las constantes $C_1$ y $C_2$ no dependen de las condiciones iniciales, entonces se puede aplicar el argumento para el otro intervalo $[t_2, 2t_2]$. De esta forma, se puede deducir la existencia de solución única $(Y, Y') \in \mathscr{D}^{2\alpha}_X$ sobre todo $[0,T]$. \\

Finalmente, note que únicamente se tiene que $(Y, Y') ,\mathscr{D}^{2\alpha}_X$. Sin embargo, como $\mathbf{X} \in \mathscr{C}^{\beta}$, tenemos que $(Y, Y') \in \mathscr{D}^{2\beta}_X$. Se debe verificar que esta solución es única en $\mathscr{D}^{2\beta}_X$. Entonces, como $Y_{s,t} = Y'_s X_{s,t} + R^Y_{s,t}$, y $X \in C^{\beta}$, vemos que $Y \in C^{\beta}$, y como $Y' = f(Y)$ y $f$ es Lipschitz, tenemos $Y' \in C^{\beta}$. Más aún, como:

\begin{align*}
	\lvert R^Y_{s,t} \rvert = \lvert Y_{s,t} - Y'_s X_{s,t} \rvert &= \left\lvert \int_s^t f(Y_u) d\mathbf{X}_u - f(Y_s) X_{s,t} \right\rvert \\
	&\lesssim \lVert Df(Y) Y' \rVert_{\infty} \lvert \mathbb{X}_{s,t} \rvert + O( \lvert t - s \rvert^{3\alpha} )
\end{align*}

y como $\mathbb{X} \in \mathcal{C}^{2\beta}_2$, entonces también $R^Y \in \mathcal{C}^{2\beta}_2$. COmo $\mathscr{D}^{2\beta}_X \subset \mathscr{D}^{2\alpha}_X$, se tiene que $(Y,Y')$ es la única solución $\mathscr{D}^{2\beta}_X$.

\begin{flushright}
	$\Box$
\end{flushright}
		

Ya una vez probado la existencia y unicidad de soluciones en ecuaciones diferenciales rugosas, se puede demostrar la continuidad del mapa de Itô-Lyons:

\[
	\mathbf{X} \mapsto (Y, Y')
\]

Este es el resultado más importante de la teoría.

\begin{theorem}[Continuidad del mapa de Itô-Lyons.]
	Sea $\beta \in \left( \frac{1}{3}, \frac{1}{2} \right]$ y $f \in C^3_b$. Sea $\mathbf{X} = (X, \mathbb{X})$, $\tilde{\mathbf{X}} = (\tilde{X}, \tilde{\mathbb{X}}) \in \mathscr{C}^{\beta}$ y $y, \tilde{y} \in \mathbb{R}^m$, y dado $(Y, Y') \mathscr{D}_X^{2\beta}$, $(\tilde{Y}, \tilde{Y'}) \in \mathscr{D}_{\tilde{X}}^{2\beta}$, que sean soluciones a la EDR del teorema anterior, con los datos $(y, \mathbf{X})$ y $(\tilde{y}, \tilde{\mathbf{X}})$ respectivamente. Sea $M > 0$ tal que $\lvert \lVert \mathbf{X} \rVert \rvert_{\beta}, \lvert \lVert \mathbf{ \tilde{X} } \rVert \rvert_{\beta} \leq M$. Entonces, para $\alpha \in \left( \frac{1}{3}, \beta \right)$, existe constante $C_M(\alpha, T, \lVert f \rVert_{C^3_b}, M) > 0$, tal que:

    \[
        \lVert Y - \tilde{Y} \rVert_{\alpha} + \lVert Y' - \tilde{Y'} \rVert_{\alpha} + \lVert R^Y - R^{\tilde{Y}}\rVert_{2\alpha} \leq C_M \left( \lvert y - \tilde{y} \rvert + \lVert \mathbf{X}; \mathbf{ \tilde{X} } \rVert_{\alpha} \right)
    \]

\end{theorem}


\textbf{Demostración:} Por el teorema anterior, solución local $(Y, Y')$ de la ecuación diferencial rugosa sobre el intervalo de tiempo $[0, t_2]$ es un elemento de $\mathcal{B}_{t_2}$, lo que significa que $\| Y, Y' \|_{\mathcal{X}, \alpha, [0, t_2]} \leq 1$. De manera similar, tenemos que $\| \tilde{Y}, \tilde{Y}' \|_{\mathcal{X}, \alpha, [0, \tilde{t}_2]} \leq 1$ para algún $\tilde{t}_2 > 0$.

Por el lema anterior (CITAR!!!), para todo $t \in (t_2 \wedge \tilde{t}_2]$, se tiene

\begin{align*}
	&\| Y' - \tilde{Y}' \|_{\alpha} = \| f(Y) - f(\tilde{Y}) \|_{\alpha} \\ 
	& \lesssim  | Y_0 - \tilde{Y}_0 | + ( | Y_0' - \tilde{Y}_0' | + \| Y' - \tilde{Y}' \|_{\alpha} ) \| X \|_{\alpha} + \| R^Y - R^{\tilde{Y}} \|_{2\alpha} t^{\alpha} + \| X - \tilde{X} \|_{\alpha}
\end{align*}
y

\begin{align*}
	&\| R^Y - R^{\tilde{Y}} \|_{2\alpha} = \left\| R^{ \int_0^{\cdot} f(Y_u)  d\mathbf{X}_u} -  R^{ \int_0^{\cdot} f( \tilde{Y}_u ) d\mathbf{ \tilde{X} }_u } \right\|_{2\alpha} \\
	& \lesssim ( |Y_0 - \tilde{Y}_0| + | Y_0' - \tilde{Y}_0' | + \| Y' - \tilde{Y}' \|_{\alpha} + \| R^Y - R^{\tilde{Y}} \|_{2\alpha} + \| X - \tilde{X} \|_{\alpha} ) \| | \mathbf{X} | \|_{\alpha} + \lVert \mathbf{X}; \mathbf{ \tilde{X} } \rVert_{\alpha}
\end{align*}

Observando que $|Y_0' - \tilde{Y}_0'| = |f(Y_0) - f(\tilde{Y}_0)| \lesssim |Y_0 - \tilde{Y}_0|$, se tiene

\begin{align*}
	&\| Y' - \tilde{Y}' \|_{\alpha} + \| R^Y - R^{\tilde{Y}} \|_{2\alpha}\\
	&\leq C_3 \left( |Y_0 - \tilde{Y}_0| + (\| Y' - \tilde{Y}' \|_{\alpha} + \| R^Y - R^{\tilde{Y}} \|_{2\alpha}) (\| | \mathbf{X} | \|_{\alpha} + t^{\alpha}) + \| \mathbf{X}; \tilde{\mathbf{X}} \|_{\alpha} \right)
\end{align*}

para alguna constante $C_3$. Tenemos que

\[
\| | \mathbf{X} | \|_{\alpha, [0,t]} + t^{\alpha} \leq \| X \|_{\beta, [0,t]} t^{\beta - \alpha} + \| \mathbb{X} \|_{2\beta, [0,t]} t^{2(\beta - \alpha)} + t^{\alpha}
\]

Eligiendo $t = t_3 \in (t_2 \wedge \tilde{t}_2]$ lo suficientemente pequeño, de tal forma que: 

\[
	C_3 \left( \| X \|_{\beta, [0,t_3]} t_3^{\beta - \alpha} + \| \mathbb{X} \|_{2\beta, [0,t_3]} t_3^{2(\beta - \alpha)} + t_3^{\alpha} \right) \leq \frac{1}{2}
\]

y reordenando, obtenemos

\[
\| Y' - \tilde{Y}' \|_{\alpha, [0,t_3]} + \| R^Y - R^{\tilde{Y}} \|_{2\alpha, [0,t_3]} \lesssim |Y_0 - \tilde{Y}_0| + \| \mathbf{X}; \tilde{\mathbf{X}} \|_{\alpha, [0,t_3]}.
\]

Entonces, por estimador de la estabilidad de la integración rugosa se concluye que:

\[
\| Y - \tilde{Y} \|_{\alpha, [0,t_3]} \lesssim |Y_0 - \tilde{Y}_0| + \| \mathbf{X} - \tilde{\mathbf{X}} \|_{\alpha, [0,t_3]}.
\]

Se sigue que existe un $\delta > 0$, que depende de $\alpha$, $T$, $\| f \|_{C_b^3}$ y $M$, tal que, para cualquier intervalo $[s,t] \subset [0,T]$ con $|t-s| \leq \delta$, se tiene

\begin{align}
\| Y' - \tilde{Y}' \|_{\alpha, [s,t]} + \| R^Y - R^{\tilde{Y}} \|_{2\alpha, [s,t]} &\lesssim |Y_0 - \tilde{Y}_0| + \| \mathbf{X} ; \tilde{\mathbf{X}} \|_{\alpha, [s,t]} \\ %, \tag{7.5} \\
\| Y - \tilde{Y} \|_{\alpha, [s,t]} &\lesssim |Y_0 - \tilde{Y}_0| + \| \mathbf{X} ; \tilde{\mathbf{X}} \|_{\alpha, [s,t]}. % \tag{7.6}
\end{align}

Tomemos una partición $\pi$ del intervalo $[0,T]$ donde $|\pi| \leq \delta$. Las estimaciones proporcionadas entonces se mantienen en cada intervalo $[s,t] \in \pi$, y al combinar las estimaciones en diferentes intervalos se puede deducir que las mismas estimaciones se mantienen sobre todo el intervalo $[0,T]$, tal cuál se hizo en el teorema anterior.


\begin{flushright}
	$\Box$
\end{flushright}

El teorema anterior, como se mencionaba previamente, es uno de los más importantes, ya que, según el control de la ecuación diferencial, $\mathbf{X}, \mathbf{\tilde{X}}$, y las condiciones iniciales $y, \tilde{y}$.


\section{Ecuaciones diferenciales estocásticas en el contexto de caminos rugosos.}

Ahora, una vez probada la existencia y unicidad de solución de ecuaciones diferenciales rugosas, se retoma el contexto de las ecuaciones diferenciales estocásticas. Para finalizar, se va a mostrar que el cálculo estocástico de Itô, y las soluciones fuertes que presentan para las EDE, coincidirán con las obtenidas al trabajar por el método de camino rugosos.

En primer lugar, se verifica que la integral de Itô está bien definida.


\begin{prop}
	Sea $(\Omega, \mathcal{F}, \left\{ \mathcal{F}_t \right\}_{t \in [0O,T]}, \mathbb{P}  )$ un espacio de probabilidad filtrado. Sea $\alpha \in \left( \frac{1}{3}, \frac{1}{2} \right)$ y sea $\mathbf{X} = \mathbf{B}^{\text{Itô}} = (B, \mathbb{B}) = (B, \mathbb{B}^{\text{Itô}})$ un camino rugoso Browniano con el levantamiento de Itô $\mathcal{F}_t$-adaptado, tal que $\mathbf{B} \in \mathscr{C}^{\alpha}$ casi siempre. Sea $(Y, Y')$ un proceso estocástico adaptado, tal que $(Y(\omega), Y'(\omega)) \in \mathscr{D}^{2\alpha}_{B(w)}$ para casi todo $\omega \in \Omega$. Entonces:

	\[
		\int_0^T Y_u d\mathbf{B}_u = \int_0^T Y_u dB_u
	\]

	casi siempre.
\end{prop}


\textbf{Demostración: } Sea $(\pi^{n})_{n\geq 1}$ una sucesión de particiones con $|\pi^{n}|\to 0$ cuando $n\to\infty$. Recordemos que la integral de Itô con respecto al movimiento browniano puede escribirse como el límite en probabilidad, en norma $L^2$ para el espacio de probabilidad:

\[
\sum_{[s,t]\in\pi^{n}}Y_{s}B_{s,t}\stackrel{\mathbb{P}}{\longrightarrow}\int_{0}^{T}Y_{u}\,\mathrm{d}B_{u}
\]

cuando $n \rightarrow \infty$. Existe entonces una subsucesión $(n_{k})_{k\geq 1}$ tal que

\[
\sum_{[s,t]\in\pi^{n_{k}}}Y_{s}B_{s,t} \to \int_{0}^{T}Y_{u}\,\mathrm{d}B_{u}
\]

cuando $k \rightarrow \infty$, casi siempre. Como

\[
	\int_0^T Y_u d\mathbf{B}_u (\omega) = \lim_{ \lvert \pi \rvert \rightarrow 0 } \sum_{ [s,t] \in \pi } Y_s (\omega) B_{s,t} (\omega) + Y'_s(\omega) \mathbb{B}_{s,t} (\omega)
\]

tenemos que

\[
\sum_{[s,t]\in\pi^{n_{k}}}Y^{\prime}_{s}\mathbb{B}_{s,t} \to \int_{0}^{T}Y_{u}\,\mathrm{d}\mathbf{B}_{u} - \int_{0}^{T}Y_{u}\,\mathrm{d}B_{u} 
\]

cuando $k \rightarrow \infty$ casi siempre. Dado que la integral de Itô $\int_{0}^{\cdot}B_{u}\otimes\mathrm{d}B_{u}$ es una martingala, se aplica la ortogonalidad de los incrementos de martingala. Es decir, para $[u,v],[s,t]\in\pi$ con $v\leq s$, se tiene: 

\[
\mathbb{E}[Y^{\prime}_{u}\mathbb{B}_{u,v}Y^{\prime}_{s}\mathbb{B}_{s,t}]=\mathbb{E}\big[\mathbb{E}[Y^{\prime}_{u}\mathbb{B}_{u,v}Y^{\prime}_{s}\mathbb{B}_{s,t}\,|\,\mathcal{F}_{s}]\big]=\mathbb{E}[Y^{\prime}_{u}\mathbb{B}_{u,v}Y^{\prime}_{s}\mathbb{E}[\mathbb{B}_{s,t}\,|\,\mathcal{F}_{s}]]=0.
\]

Para cualquier partición $\pi$, se tiene que

\[
\mathbb{E}\bigg[\bigg|\sum_{[s,t]\in\pi}Y^{\prime}_{s}\mathbb{B}_{s,t}\bigg|^{2}\bigg]=\sum_{[s,t]\in\pi}\mathbb{E}\big[|Y^{\prime}_{s}\mathbb{B}_{s,t}|^{2}\big].
\]


que es análoga a la isometría de Itô. Supongamos por el momento que $\|Y^{\prime}\|_{L^{\infty}(\Omega\times[0,T])}\leq M$ para alguna constante $M>0$. Por la demostración del levantamiento de Itô representa un camino rugoso (CITAR!!!), se tiene que $\mathbb{E}[\left|\mathbb{B}_{s,t}\right|^{2}]\leq C|t-s|^{2}$ para una constante $C$. Entonces

\[
\mathbb{E}\bigg[\bigg|\sum_{[s,t]\in\pi}Y^{\prime}_{s}\mathbb{B}_{s,t}\bigg|^{2}\bigg]=\sum_{[s,t]\in\pi}\mathbb{E}\big[|Y^{\prime}_{s}\mathbb{B}_{s,t}|^{2}\big]\leq CM^{2}\sum_{[s,t]\in\pi}|t-s|^{2}\leq CM^{2}T|\pi|. % \tag{8.4}
\]

Aplicando esto con $\pi=\pi^{n_{k}}$, se tiene que

\[
\sum_{[s,t]\in\pi^{n_{k}}}Y^{\prime}_{s}\mathbb{B}_{s,t}\,\xrightarrow{L^{2}(\mathbb{P})}\,0\quad \text{cuando}\;\;k\to\infty.
\]

Entonces, como esta sucesión de variables aleatorias convergen casi siempre, entonces los límites deben ser iguales casi siempre.

Si $\|Y^{\prime}\|_{L^{\infty}(\Omega\times[0,T])}$ no es finita, se puede usar un argumento de localización: Sea $M>0$ y el tiempo de parada $\tau_{M}=T\wedge\inf\{t\in[0,T]:|Y^{\prime}_{t}|\geq M\}$. Por el argumento anterior con los procesos parados $(Y^{\prime})^{\tau_{M}}$ y $\mathbf{B}^{\tau_{M}}=(B^{\tau_{M}},\mathbb{B}^{\tau_{M}})$, se deduce que $\int_{0}^{\tau_{M}}Y_{u}\,\mathrm{d}\mathbf{B}_{u}=\int_{0}^{\tau_{M}}Y_{u}\,\mathrm{d}B_{u}$ casi siempre. Como $\tau_{M}\to T$ cuando $M\to\infty$ casi siempre. Al hacer $M \rightarrow \infty$, se tiene lo deseado.


\begin{flushright}
	$\Box$
\end{flushright}

Este resultado, se tiene también para las integrales estocásticas de Stratonovich.


\begin{prop}
	Sea $(\Omega, \mathcal{F}, \left\{ \mathcal{F}_t \right\}_{t \in [0O,T]}, \mathbb{P}  )$ un espacio de probabilidad filtrado. Sea $\alpha \in \left( \frac{1}{3}, \frac{1}{2} \right)$ y sea $\mathbf{X} = \mathbf{B}^{\text{Strat}} = (B, \mathbb{B}) = (B, \mathbb{B}^{\text{Strat}})$ un camino rugoso Browniano con el levantamiento de Stratonovich $\mathcal{F}_t$-adaptado, tal que $\mathbf{B} \in \mathscr{C}^{\alpha}$ casi siempre. Sea $(Y, Y')$ un proceso estocástico adaptado, tal que $(Y(\omega), Y'(\omega)) \in \mathscr{D}^{2\alpha}_{B(w)}$ para casi todo $\omega \in \Omega$. Entonces:

	\[
		\int_0^T Y_u d\mathbf{B}_u = \int_0^T Y_u \circ dB_u
	\]

	casi siempre.
\end{prop}

La demostración es análoga a la anterior. \\

Ahora, dada la consistencia de la integral rugosa y estocástica, entonces, no es muy dificil demostrar la consistencia de ecuaciones diferenciales rugosas y estocásticas.

\begin{prop}
	Sea $(\Omega, \mathcal{F}, \left\{ \mathcal{F}_t \right\}_{t \in [0O,T]}, \mathbb{P}  )$ un espacio de probabilidad filtrado. Sea $f \in C^3_b$, $\alpha \in \left( \frac{1}{3}, \frac{1}{2} \right)$ y $y \in L^2 (\mathbb{P})$. Sea $B$ un movimiento Browniano $d$-dimensional.

	\begin{itemize}
		\item Sea $\mathbf{B}^{\text{Itô}} = (B, \mathbb{B}^{\text{Itô}}) \in \mathscr{C}^{\alpha}$ es un camino rugoso con el levantamiento de Itô, y $(Y, Y')$ a solución a la ecuación diferencial rugosa

		\[
			dY_t = f(Y_t) d\mathbf{B}^{\text{Itô}}_t, \qquad Y_0 = y	
		\]

		entonces $Y$ es la única solución fuerte de la ecuación diferencial estocástica (En el sentido de Itô):

		\[
			dY_t = f(Y_t) dB_t, \qquad Y_0 = y
		\]

		\item Sea $\mathbf{B}^{\text{Strat}} = (B, \mathbb{B}^{\text{Strat}}) \in \mathscr{C}^{0, \alpha}_g$ es un camino rugoso con el levantamiento de Stratonovich, y $(Y, Y')$ a solución a la ecuación diferencial rugosa

		\[
			dY_t = f(Y_t) d\mathbf{B}^{\text{Itô}}_t, \qquad Y_0 = y	
		\]

		entonces $Y$ es la única solución fuerte de la ecuación diferencial estocástica (En el sentido de Stratonovich):

		\[
			dY_t = f(Y_t) \circ dB_t, \qquad Y_0 = y
		\]

	\end{itemize}

\end{prop}

Se acabú la tesis.