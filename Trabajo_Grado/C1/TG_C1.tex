
\chapter{Preliminares}


En este capítulo, nos dedicaremos a repasar conceptos de teoría de la probabilidad, teoría de integración y ecuaciones diferenciales estocásticas. Para una mayor información, en cada sección



%================ SECCIÓN 1. PROBABILIDAD



\section{Conceptos de Probabilidad}

Sea $\Omega$ un conjunto abstracto. Denotamos por $2^{\Omega}$ el conjunto de partes de $\Omega$.

\begin{boxDef}
Definimos a $\mathcal{F}$ una $\mathbf{\sigma}$\textbf{-álgebra} es subconjunto de $2^{\Omega}$ que cumple las siguientes propiedades:

	\begin{itemize}
		\item $\emptyset$, $\Omega$ $\in \mathcal{F}$
		\item Si $A \in \mathcal{F}$, luego $A^{c} \in \mathcal{F}$
		\item Dado $\{ A_i \}_{i \in I}$ una sucesión de subconjuntos de $\Omega$ a lo más contable. Luego, si para todo $i \in I$, $A_i \in \mathcal{F}$, entonces $\cup_{i \in I} A_i \in \mathcal{F}$ 
	\end{itemize}

\end{boxDef}

Los elementos en $\mathcal{F}$



Variables aleatorias.

Medida de probabilidad.

Probabilidad Condicional.

Independencia de variables aleatorias.

Inegración respecto a medida de probabilidad.

Variables aleatorias Gaussianas.

Convergencia de esta mondá, Ley de los grandes números.

Esperanza Condicional, L2 y Espacios de Hilbert.

¿Función característica, generadora de momentos, etc...?


INTEGRACIÓN.

Integración usual RS, Integral de Young. Conceptos de p-variación y $\alpha$-Hölder.

Ecuaciones Diferenciales Ordinarias. EDO Controladas. ¿EDP?

Teorema de Picard, Teorema de Peano.