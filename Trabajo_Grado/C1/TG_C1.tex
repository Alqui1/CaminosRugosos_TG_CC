
\chapter{Preliminares}


En este capítulo, nos dedicaremos a repasar conceptos de teoría de la probabilidad, teoría de integración y ecuaciones diferenciales estocásticas. Para una mayor información, en cada sección



%================ SECCIÓN 1. PROBABILIDAD



\section{Conceptos de Probabilidad}

Sea $\Omega$ un conjunto abstracto. Denotamos por $2^{\Omega}$ el conjunto de partes de $\Omega$.

\begin{boxDef}
Definimos a $\mathcal{F}$ una $\mathbf{\sigma}$\textbf{-álgebra} es subconjunto de $2^{\Omega}$ que cumple las siguientes propiedades:

	\begin{itemize}
		\item $\emptyset$, $\Omega$ $\in \mathcal{F}$
		\item Si $A \in \mathcal{F}$, luego $A^{c} \in \mathcal{F}$
		\item Dado $\{ A_i \}_{i \in I}$ una sucesión de subconjuntos de $\Omega$ a lo más contable. Luego, si para todo $i \in I$, $A_i \in \mathcal{F}$, entonces $\cup_{i \in I} A_i \in \mathcal{F}$ 
	\end{itemize}

El espacio $\left( \Omega, \mathcal{F} \right)$ se llama \textbf{espacio medible}.

\end{boxDef}



Los elementos en $\mathcal{F}$ se llamarán \textit{eventos}.



\textbf{Ejemplo:}

\begin{itemize}

	\item Para $\Omega$ un conjunto abstracto, $\mathcal{F} = \left\{ \emptyset, \Omega \right\}$ es la $\sigma$-álgebra trivial. 

	\item Sea $A \subset \Omega$, entonces $\sigma(A) = \left\{ \emptyset, A, A^c, \Omega \right\}$ también es una $\sigma$-álgebra, llamada la \textbf{menor} $\mathbb{\sigma}$-\textbf{álgebra} que contiene a $A$, que se genera mediante la intersección de todas las $\sigma$-álgebras que contienen a $A$. 

	\item Para $\Omega = \mathbb{R}$, una $\sigma$-álgebra para este conjunto es la $\sigma$-\textbf{álgebra de Borel}, que se puede generar con intervalos de la forma $(-\infty, a]$ para todo $a \in \mathbb{Q}$. También, es la generada por todos los conjuntos abiertos (O cerrados, o semiabiertos...). Para más información consulte CITAR DONALD COHN Y PROTTER.  

\end{itemize}

\begin{flushright}
	$\Box$
\end{flushright}


\begin{boxDef}
	Una \textbf{medida de probabilidad} definida en una $\sigma$-álgebra $\mathcal{F}$ de $\Omega$, es una función $P: \mathcal{F} \rightarrow [0,1]$ que cumple:

	\begin{itemize}
		\item $P(\Omega) =  1$\\
		\item Para toda colección contable $\left\{ F_n \right\}_{n \geq 1}$ de elementos en $\mathcal{F}$ que son disyuntos par a par, se tiene:

		\[
			P\left( \cup_{n=1}^{\infty}  \right) = \sum_{n = 1}^{\infty} P\left( A_n \right)
		\]

		Es decir, la función es \textit{contablemente aditiva}. Se llama a $P(A)$ como la \textit{probabilidad del evento A}.

	\end{itemize}

La tripla $(\Omega, \mathcal{F}, P)$ se conoce como \textbf{espacio de probabilidad}.

\end{boxDef}

De forma general, la medida de probabilidad, es un caso específico de una \textit{función de medida}, en este caso, tendremos un \textit{espacio de medida}. Vea COHN.\\

Vamos a revisar algunas propiedades de las funciones de probabilidad, sin demostración. Para consultar los detalles, puede consultar PROTTER.

\begin{theorem}
	Sea $(\Omega, \mathcal{F})$ un espacio medible.
\end{theorem}



% Variables aleatorias.

% Medida de probabilidad.

% Probabilidad Condicional.

% Independencia de variables aleatorias.

% Inegración respecto a medida de probabilidad.

% Variables aleatorias Gaussianas.

% Convergencia de esta mondá, Ley de los grandes números.

% Esperanza Condicional, L2 y Espacios de Hilbert.

% ¿Función característica, generadora de momentos, etc...?

% INTEGRACIÓN.

% Integración usual RS, Integral de Young. Conceptos de p-variación y $\alpha$-Hölder.

% Ecuaciones Diferenciales Ordinarias. EDO Controladas. ¿EDP?

% Teorema de Picard, Teorema de Peano.-