\documentclass[12pt, twocolumns]{book}

\usepackage[utf8]{inputenc}
\usepackage[spanish]{babel}


\usepackage{amsmath}
\usepackage{amssymb}
\usepackage{amsfonts}

\usepackage[many]{tcolorbox}  % for COLORED BOXES (tikz and xcolor included)


\definecolor{sub_def}{HTML}{F8D9AC}
\definecolor{main_def}{HTML}{FAAC41}

% DEFINITION BOXES
\tcbset{
    sharp corners,
    colback = white,
    before skip = 0.2cm,    % add extra space before the box
    after skip = 0.5cm      % add extra space after the box
}                           % setting global options for tcolorbox

% You can copy any following box you like to your code.
\newtcolorbox{boxDef}{
    %fontupper = \bf,
    colback = sub_def,
    boxrule = 0pt,
    leftrule = 6pt,
    colframe = main_def % frame color
}



% TEMP THEOREMS

\newtheorem{theorem}{Teorema}
\newtheorem{coro}{Corolario}





% Portada Temporal
\title{Caminos rugosos y soluciones de ecuaciones diferenciales.}
\author{David Alejandro Alquichire Rincón}
\date{\today}

\begin{document}

\maketitle

Idea: Estudiar ecuaciones diferenciales estocásticas por medio de caminos rugosos.

¿Hasta dónde? Peter Fritz... soluciones a PDE estocásticas... ¿Métodos Numéricos?

Propuesta capítulos:


\begin{enumerate}

\item Introducción y Preliminares

\begin{enumerate}

	\item Conceptos de Probabilidad y Teoría de la medida (LB, D. Cohn, Protter y el otro libro). 
	\item Conceptos en Convergencia de Procesos Estocásticos.
	\item Conceptos de Procesos Estocásticos (Notas de Freddy, apoyo de Capinski) 

	\item Integración de Riemann Stieltjes
	\item Teoría de la medida y la integral de Lebesgue
	\item Análisis Funcional
	\item Ecuaciones Diferenciales Ordinarias (Existencia y Unicidad)
	\item Ecuaciones Diferenciales Parciales

\end{enumerate}

\item Construcción del Movimiento Browniano

\begin{enumerate}
	\item a
\end{enumerate}


\item Construcción de la Integral de Itô

\begin{enumerate}
	\item a
\end{enumerate}


\item Ecuaciones Diferenciales Estocásticas por Itô -> Oksendal

\begin{enumerate}

	\item Integral de Itô, Cálculo Estocástico
	\item Ecuaciones Diferenciales Estocásticas, Solución clásica de Itô.
	\item Teoremas de Existencia y Unicidad.

\end{enumerate}



\item Ecuaciones Diferenciales Estocásticas por caminos rugosos



\item EDP Estocásticas*
\item Métodos Numéricos y Aplicaciones*
\item Conclusiones
\item Bibliografía

\end{enumerate}

\newpage

% ========================
% ======= ABSTRACT =====
% ========================


\textbf{Título} 

Caminos rugosos y soluciones de ecuaciones diferenciales.\\

\textbf{Title}

Rough paths and solutions to differential equations. \\

\textbf{Resumen: }  \\

\textbf{Abstract: } \\

\textbf{Palabras clave:} \\

\textbf{Keywords:} \\ 




% ========================
% ========================
% ========================

\tableofcontents


% ========================
% ======= CAPÍTULO 1 =====
% ========================


\chapter{Preliminares}

\section{Conceptos de Probabilidad}

Sea $\Omega$ un conjunto abstracto. Denotamos por $2^{\Omega}$ el conjunto de partes de $\Omega$.

\begin{boxDef}
Definimos a $\mathcal{F}$ una $\mathbf{\sigma}$\textbf{-álgebra} es subconjunto de $2^{\Omega}$ que cumple las siguientes propiedades:

	\begin{itemize}
		\item $\emptyset$, $\Omega$ $\in \mathcal{F}$
		\item Si $A \in \mathcal{F}$, luego $A^{c} \in \mathcal{F}$
		\item Dado $\{ A_i \}_{i \in I}$ una sucesión de subconjuntos de $\Omega$ a lo más contable. Luego, si para todo $i \in I$, $A_i \in \mathcal{F}$, entonces $\cup_{i \in I} A_i \in \mathcal{F}$ 
	\end{itemize}


\end{boxDef}

% ========================
% ========================
% ========================


% ========================
% ======= CAPÍTULO 2 =====
% ========================

\chapter{Procesos Estocásticos y el Movimiento Browniano}

En este capítulo, inicialmente se hablará acerca de los procesos estocásticos y se darán algunos conceptos básicos. Luego, se estudiará un proceso estocástico a tiempo continuo, que es muy importante en la teoría de ecuaciones diferenciales estocásticas, que es el \textit{movimiento Browniano}. Primero, se hablará sobre la construcción de este proceso (Usando los teoremas de Kolmogorov), luego se darán algunas propiedades bastante importantes.

% ========================
% ======= SECCIÓN 1: PRELIMINARES DE PE =====
% ========================

\section{Conceptos de Procesos Estocásticos}

En esta sección, se introducirán los conceptos básicos de procesos estocásticos en general, tanto de tiempo discreto como de tiempo continuo.

\begin{boxDef}
	Un \textbf{proceso estocástico} corresponde a la colección $\left\{ X_t : \Omega \rightarrow S \right\}_{t \in T}$ de variables aleatorias definidas en un mismo espacio de probabilidad $(\Omega, \mathcal{F}, P)$. \\

	A $S$ se le conoce como \textbf{espacio de estados} y $T$ se le conoce como el tiempo.	
\end{boxDef}

Si $T$ es un conjunto finito o contable, entonces el proceso corresponde a un \textit{proceso a tiempo discreto}. De otro modo, se dice que se tiene un \textit{proceso a tiempo continuo}. \\

\textbf{Ejemplo: } El \textit{movimiento Browniano}, posee un espacio de estados continuo, $S = \mathbb{R}^n$, y también tiempo continuo, $T = [0, \infty)$. Otro ejemplo es el \textit{paso aleatorio simple}, tal que su espacio de estados es discreto, $S = \left\{ s_1, s_2, \cdots \right\}$ (Se puede interpretar los vértices en un grafo que se pueden visitar), y además, también tiene tiempos discretos, por ejemplo, $T = \left\{ t_1, t_2, \cdots \right\}$.

\begin{flushright}
	$\Box$
\end{flushright}

Mayoritariamente, en procesos estocásticos, uno se interesa más en las distribuciones conjuntas de las variables aleatorias. Esto motiva la siguiente definición:

\begin{boxDef}
	Las distribuciones conjuntas de $(X_{t_1}, X_{t_2}, \cdots)$ son llamadas \textbf{distribuciones finito-dimensionales} del proceso $\left\{ X_t \right\}_{t \in T}$.
\end{boxDef}

% qué escribo acá...

\begin{boxDef}
	Dado $\left\{ X_t \right\}_{t \in T}$ un proceso con espacio de estados $S$ definido en $(\Omega, \mathcal{F}, P)$. Para cada $w \in \Omega$, se define como la \textbf{trayectoria}, a la función:

	\begin{align*}
		X(w): T &\rightarrow S \\
		t &\mapsto X_t (w)
	\end{align*}



\end{boxDef}

Note que, hay una equivalencia, entre hablar una probabilidad $\mu_X$ sobre el conjunto de las trayectorias $X: T \rightarrow S$, y una distribución conjunta de todos los tiempos $t \in T$ para el proceso $\left\{ X_t \right\}_{t \in T}$.

\begin{boxDef}
	Un proceso estocástico $\left\{ X_t \right\}_{t \in T}$ en $(\Omega, \mathcal{F})$ se llama \textbf{conjuntamente medible} si:

	\begin{align*}
		X: T \times \Omega &\rightarrow \mathbb{R} \\
		(t, w) &\mapsto X_t (w)
	\end{align*}

	es medible respecto a la $\sigma$-álgebra producto $\mathcal{B}(T) \otimes \mathcal{F}.$

\end{boxDef}

\begin{boxDef}
	Dados $\left\{ X_t \right\}_{t \in T}$ y $\left\{ Y_t \right\}_{t \in T}$ en un mismo espacio de probabilidad $(\Omega, \mathcal{F}, P)$.

	\begin{itemize}
		\item Se dice que $\left\{ Y_t \right\}$ es \textbf{versión} de $\left\{ X_t \right\}$ si:

		\[
			P( X_t = Y_t ) = P( \left\{ w \in \Omega \vert X_t (w) = Y_t(w) \right\}) = 1
		\]
		para todo $t \in T$, esto es para todo tiempo fijo.

		\item Se dice que estos dos procesos son \textbf{indistinguibles} si:

		\begin{align*}
			P[X_t = Y_t \text{ para todo } t \in T] &= \\ 
			P[\left\{ w \in \Omega \vert X_t (w) = Y_t (w) \text{ para todo } t \in T \right\}] =& 1
		\end{align*}
	\end{itemize}


\end{boxDef}

Note que la segunda definición es mucho más fuerte que la primera definición, porque dos procesos \textit{indistinguibles} tendrás trayectorias iguales (Salvo en un conjunto de puntos de medida cero, como un conjunto de puntos discreto).

\begin{boxDef}
	Una familia $\left\{ \mathcal{F}_t \right\}$ de sub-$\sigma$-álgebras de $\mathcal{F}$ se dice \textbf{filtración} en $(\Omega, \mathcal{F})$ si, para $t_1 \leq t_2$, $\mathcal{F}_{t_1} \subset \mathcal{F}_{t_2}$, esto es, una familia creciente de sub-$\sigma$-álgebras.\\

	Dado $\left\{ X_t \right\}_{t \in T}$ un proceso estocástico, se le llamará \textbf{proceso} $\mathcal{F_t}$\textbf{-adaptado} si para todo $t \in T$, $X_t$ es medible respecto a $\mathcal{F}_t$, para todo t.
\end{boxDef}

Si se toma como $\mathcal{F}_{t}^{X} = \sigma (\left\{ X_s \right\}_{s < t})$, esta es la menor filtración tal que $\left\{X_t\right\}$ es un proceso adaptado a esta filtración, conocida como \textbf{filtración natural de }$\left\{X_t\right\}_{t \in T}$.

\begin{boxDef}
	Una variable aleatoria $\tau$ en $(\Omega, \mathcal{F}, P)$ se le conoce como \textbf{tiempo de parada} respecto a la filtración $\left\{ \mathcal{F}_t\right\}_{t\in T}$ si:

	\begin{itemize}
		 \item $P[\tau < \infty] = 1$
		 \item $\left\{ \tau \leq t \right\}  = \left\{ w \in \Omega \vert \tau(w) \leq t \right\}\in \mathcal{F}_t$
	\end{itemize}	

\end{boxDef}

% ========================
% ========================
% ========================






% ========================
% ======= SECCIÓN 2: MOVIMIENTO BROWNIANO =====
% ========================

\section{Construcción del Movimiento Browniano}

Primero, damos la definición de un movimiento Browniano, y luego, se hará la construcción. Esta, se puede hacer por dos maneras distintas:

\begin{enumerate}
	\item Teoremas de existencia y continuidad de Kolmogorov.
	\item Teorema de Donsker (Caso más general).
\end{enumerate}

En el presente trabajo, haremos la construcción usando el Teorema de Donsker, y más tarde, se enunciarán los teoremas de Kolmogorov (Sin demostración).

\begin{boxDef}
	Dado $\{ W_t \}$ un proceso estocástico, en el espacio de probabilidad $(\Omega, \mathcal{F}, P)$. El proceso $\{ W_t \}$ es un \textbf{movimiento Browniano} en una dimensión, si se cumplen las siguientes condiciones:

	\begin{itemize}
		\item Para casi todo $\omega$, los caminos $W_t (\omega)$ son continuos (En el sentido de la probabilidad).
		\item $\{ W_t \}$ es un proceso Gaussiano, es decir, para $k \geq 1$, y todo $0 \leq t_1 \leq \cdots \leq t_k$, el vector aleatorio, $Z = (W_{t_1}, \cdots, W_{t_k}) \in \mathbb{R}^{n}$ tiene distribución multinormal (O Gaussiana).
	\end{itemize}

\end{boxDef}

% ========================
% ========================
% ========================








% ========================
% ======= EXTRAS!!!! =====
% ========================


\section{Miscelánea del Movimiento Browniano (Opcional)}

El movimiento Browniano satisface varias propiedades. Como

\begin{itemize}
	\item $\left\{ B_t \right\}$ es un proceso Gaussiano (Multinormal).
	\item $\mathbb{E}[B_t] = 0$ y $\mathbb{E}[B_t B_s] = \min\{ s, t \}$
	\item Para casi todo $\omega \in \Omega$, $t \mapsto B_t (\omega)$ es continua.
\end{itemize}

% https://www.stat.berkeley.edu/~pitman/s205s03/lecture15.pdf

Observe lo siguiente:

\begin{itemize}
	\item \textbf{Escalamiento: } Para todo $s > 0$, $\left\{ s^{-1/2} B_{st} \vert t \geq 0 \right\}$ es un movimiento Browniano comenzando en 0. Más aún, se cumple que:

	\[
		\left\{ B_{s,t} \vert t \geq 0 \right\}  \stackrel{d}{=}  \left\{ s^{1/2} B_t \vert t \geq 0 \right\}
	\]
\end{itemize}



% ========================
% ========================
% ========================


% ========================
% ========================
% ========================


% ========================
% ======= CAPÍTULO 3 =====
% ========================

\chapter{La Integral de Itô}

En este capítulo, comenzaremos la construcción de la integral de Itô. Para cumplir este objetivo, usaremos fuertemente los hechos vistos en el capítulo anterior del movimiento Browniano. 

Ahora, ¿Por qué es necesario construir una nueva integral? Veamos el objetivo inicial, solucionar una ecuación diferencial que tiene cierto ruido:

\[
	\frac{dX}{dt} = b(t, X_t) + \sigma (t, X_t) \cdot W_t
\]

Note que el ruido se puede representar como el proceso estocástico $W_t$. Bajo experimentación, se interponen las siguientes condiciones sobre el ruido:

\begin{itemize}
	\item Dos variables del proceso $W_{t_1}$ y $W_{t_2}$ con $t_1 \neq t_2$ son independientes.
	\item $\{ W_t \}$ es un proceso estacionario.
	\item $\mathbb{E}[W_t] = 0$ para todo $t$.
\end{itemize} 

No hay algún proceso estocástico tradicional que cumpla las condiciones dadas. Por ende, lo podemos ver como un proceso estocástico generalizado, un \textbf{proceso de ruido blanco}, esto es, un proceso que se puede construir como medida de probabilidad en cierto espacio sútil de funcionales $\mathcal{C}[ 0, \infty )$. 

Por ende, se nos sugiere que el proceso $\{W_t\}$ será el movimiento Browniano. Discretizando la ecuación inicial...

\[
	\int_0^t f(s,w) dB_s(w)
\]

%  (Oksendal 3.11, Kallianpur 1980 p.10

% ========================
% ========================
% ========================



\end{document}