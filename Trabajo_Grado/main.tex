\documentclass[12pt, twocolumns]{book}

\usepackage[utf8]{inputenc}
\usepackage[spanish]{babel}


\usepackage{amsmath}
\usepackage{amssymb}
\usepackage{amsfonts}

\usepackage[many]{tcolorbox}  % for COLORED BOXES (tikz and xcolor included)


\definecolor{sub_def}{HTML}{F8D9AC}
\definecolor{main_def}{HTML}{FAAC41}

% DEFINITION BOXES
\tcbset{
    sharp corners,
    colback = white,
    before skip = 0.2cm,    % add extra space before the box
    after skip = 0.5cm      % add extra space after the box
}                           % setting global options for tcolorbox

% You can copy any following box you like to your code.
\newtcolorbox{boxDef}{
    %fontupper = \bf,
    colback = sub_def,
    boxrule = 0pt,
    leftrule = 6pt,
    colframe = main_def % frame color
}



% TEMP THEOREMS

\newtheorem{theorem}{Teorema}
\newtheorem{coro}{Corolario}





% Portada Temporal
\title{Caminos rugosos y soluciones de ecuaciones diferenciales.}
\author{David Alejandro Alquichire Rincón}
\date{\today}

\begin{document}

\maketitle

Idea: Estudiar ecuaciones diferenciales estocásticas por medio de caminos rugosos.

¿Hasta dónde? Peter Fritz... soluciones a PDE estocásticas... ¿Métodos Numéricos?

Propuesta capítulos:


\begin{enumerate}

\item Introducción y Preliminares

\begin{enumerate}

	\item Conceptos de Probabilidad y Teoría de la medida (LB, D. Cohn, Protter y el otro libro). 
	\item Conceptos en Convergencia de Procesos Estocásticos.
	\item Conceptos de Procesos Estocásticos (Notas de Freddy, apoyo de Capinski) 

	\item Integración de Riemann Stieltjes
	\item Teoría de la medida y la integral de Lebesgue
	\item Análisis Funcional
	\item Ecuaciones Diferenciales Ordinarias (Existencia y Unicidad)
	\item Ecuaciones Diferenciales Parciales

\end{enumerate}

\item Construcción del Movimiento Browniano

\begin{enumerate}
	\item a
\end{enumerate}


\item Construcción de la Integral de Itô

\begin{enumerate}
	\item a
\end{enumerate}


\item Ecuaciones Diferenciales Estocásticas por Itô -> Oksendal

\begin{enumerate}

	\item Integral de Itô, Cálculo Estocástico
	\item Ecuaciones Diferenciales Estocásticas, Solución clásica de Itô.
	\item Teoremas de Existencia y Unicidad.

\end{enumerate}



\item Ecuaciones Diferenciales Estocásticas por caminos rugosos



\item EDP Estocásticas*
\item Métodos Numéricos y Aplicaciones*
\item Conclusiones
\item Bibliografía

\end{enumerate}

\newpage

% ========================
% ======= ABSTRACT =====
% ========================


\textbf{Título} 

Caminos rugosos y soluciones de ecuaciones diferenciales.\\

\textbf{Title}

Rough paths and solutions to differential equations. \\

\textbf{Resumen: }  \\

\textbf{Abstract: } \\

\textbf{Palabras clave:} \\

\textbf{Keywords:} \\ 




% ========================
% ========================
% ========================

\tableofcontents


% ========================
% ======= CAPÍTULO 1 =====
% ========================


\chapter{Preliminares}


En este capítulo, nos dedicaremos a repasar conceptos de teoría de la probabilidad, teoría de integración y ecuaciones diferenciales estocásticas. Para una mayor información, en cada sección


% ========================================
%================ SECCIÓN 1. PROBABILIDAD
% ========================================



% ========================================
%================ SECCIÓN 1.1 Espacios de probabilidad.
% ========================================



\section{Conceptos de Probabilidad.}

\subsection{Espacios de probabilidad.}

Sea $\Omega$ un conjunto abstracto. Denotamos por $2^{\Omega}$ el conjunto de partes de $\Omega$.

\begin{boxDef}
Definimos a $\mathcal{F}$ una $\mathbf{\sigma}$\textbf{-álgebra} es subconjunto de $2^{\Omega}$ que cumple las siguientes propiedades:

	\begin{itemize}
		\item $\emptyset$, $\Omega$ $\in \mathcal{F}$
		\item Si $A \in \mathcal{A}$, luego $A^{c} \in \mathcal{A}$
		\item Dado $\{ A_i \}_{i \in I}$ una sucesión de subconjuntos de $\Omega$ a lo más contable. Luego, si para todo $i \in I$, $A_i \in \mathcal{A}$, entonces $\cup_{i \in I} A_i \in \mathcal{A}$ 
	\end{itemize}

El espacio $\left( \Omega, \mathcal{A} \right)$ se llama \textbf{espacio medible}.

\end{boxDef}



Los elementos en $\mathcal{A}$ se llamarán \textit{eventos}.



\textbf{Ejemplo:}

\begin{itemize}

	\item Para $\Omega$ un conjunto abstracto, $\mathcal{A} = \left\{ \emptyset, \Omega \right\}$ es la $\sigma$-álgebra trivial. 

	\item Sea $A \subset \Omega$, entonces $\sigma(A) = \left\{ \emptyset, A, A^c, \Omega \right\}$ también es una $\sigma$-álgebra, llamada la \textbf{menor} $\mathbb{\sigma}$-\textbf{álgebra} que contiene a $A$, que se genera mediante la intersección de todas las $\sigma$-álgebras que contienen a $A$. 

	\item Para $\Omega = \mathbb{R}$, una $\sigma$-álgebra para este conjunto es la $\sigma$-\textbf{álgebra de Borel}, que se puede generar con intervalos de la forma $(-\infty, a]$ para todo $a \in \mathbb{Q}$. También, es la generada por todos los conjuntos abiertos (O cerrados, o semiabiertos...). Para más información consulte CITAR DONALD COHN Y PROTTER.  

\end{itemize}

\begin{flushright}
	$\Box$
\end{flushright}


\begin{boxDef}
	Una \textbf{medida de probabilidad} definida en una $\sigma$-álgebra $\mathcal{A}$ de $\Omega$, es una función $P: \mathcal{A} \rightarrow [0,1]$ que cumple:

	\begin{itemize}
		\item $P(\Omega) =  1$\\
		\item Para toda colección contable $\left\{ A_n \right\}_{n \geq 1}$ de elementos en $\mathcal{A}$ que son disyuntos par a par, se tiene:

		\[
			P\left( \cup_{n=1}^{\infty}  \right) = \sum_{n = 1}^{\infty} P\left( A_n \right)
		\]

		Es decir, la función es \textit{contablemente aditiva}. Se llama a $P(A)$ como la \textit{probabilidad del evento A}.

	\end{itemize}

La tripla $(\Omega, \mathcal{A}, P)$ se conoce como \textbf{espacio de probabilidad}.

\end{boxDef}

De forma general, la medida de probabilidad, es un caso específico de una \textit{función de medida}, en este caso, tendremos un \textit{espacio de medida}. Vea COHN.\\

Note que, podemos ver una propiedad más débil que el axioma (2) en la anterior definición. Para toda colección $\left\{  A_k \right\}_{k = 1}^{n}$ \textit{finita}, de disyuntos par a par, si tenemos:

\[
	P\left( \cup_{k=1}^n A_k \right) = \sum_{k = 1}^{n} P(A_k)
\]
entonces la función $P$ es \textbf{aditiva (O finitamente aditiva)}.

Vamos a revisar algunas propiedades de las funciones de probabilidad, sin demostración. Para consultar los detalles, puede consultar PROTTER.

\begin{theorem}
	Sea $(\Omega, \mathcal{A})$ un espacio medible, y $P: \mathcal{A} \rightarrow [0,1]$ una función finitamente aditiva y $P(\Omega) = 1$. Entonces, tenemos las siguientes equivalencias:

	\begin{itemize}
		\item La función es contablemente aditiva.
		\item Si $A_n \in \mathcal{A}$ y $A_n \downarrow \emptyset$, luego $P(A_n) \downarrow 0$.
		\item Si $A_n \in \mathcal{A}$ y $A_n \downarrow A$, luego $P(A_n) \downarrow P(A)$.
		\item Si $A_n \in \mathcal{A}$ y $A_n \uparrow \Omega$, luego $P(A_n) \uparrow 1$.
		\item Si $A_n \in \mathcal{A}$ y $A_n \uparrow A$, luego $P(A_n) \uparrow P(A)$.
	\end{itemize}

	Más aún, si $P$ es una medida de probabilidad, y dado $\left\{ A_n \right\}$ sucesión de eventos que converge a $A$. Entonces $A \in \mathcal{A}$ y $\lim_{n \rightarrow \infty} P(A_n) = P(A)$ 

\end{theorem}


% ¿Colocar teorema de la clase monótona?


% ========================================
%================ SECCIÓN 1.2 Variables aleatorias.
% ========================================



\subsection{Variables aleatorias.}

En esta sección, tommaos a $(\Omega, \mathcal{A}, P)$ un espacio abstracto, donde $\Omega$ no es necesariamente contable.

\begin{boxDef}
	Sean $(E, \mathcal{E})$ y $(F, \mathcal{F})$ dos espacios medibles (No necesariamente tienen una medida de probabilidad). Una función $X: E \rightarrow F$ es una \textbf{función medible} si $X^{-1}(\Lambda) \in \mathcal{E}$ para todo $\Lambda \in \mathcal{F}$.\\

	Si $(E, \mathcal{E})$ es un espacio de probabilidad, $X$ posee el nombre de \textbf{variable aleatoria}.
\end{boxDef}

Nuevamente, tenemos varias propiedades para las funciones medibles, que enunciaremos acá, sin la demostración respectiva. Para esto, consulte, PROTTER.

\begin{coro}
	Sea $(E, \mathcal{E})$ un espacio medible aleatorio, y $(\mathbb{R}, \mathcal{B})$. Sea $X, X_n: E \rightarrow \mathbb{R}$ funciones:

	\begin{itemize}
		\item X es medible si y sólo si $\left\{ X \leq a \right\} = X^{-1}( (-\infty, a] ) \in \mathcal{E}$, para todo $a \in \mathbb{R}$.
		\item Si cada $X_n$ es medible, luego $\sup X_n$, $\inf X_n$, $\limsup X_n$ y $\liminf X_n$ son medibles.
		\item Si cada $X_n$ es medible, y $\left\{X_n\right\}$ converge puntualmente a $X$, luego $X$ es medible. 
	\end{itemize}

\end{coro}

\begin{theorem}
	Dasd
\end{theorem}
	




% Variables aleatorias.

% Medida de probabilidad.

% Probabilidad Condicional.

% Independencia de variables aleatorias.

% Inegración respecto a medida de probabilidad.

% Variables aleatorias Gaussianas.

% Convergencia de esta mondá, Ley de los grandes números.

% Esperanza Condicional, L2 y Espacios de Hilbert.

% ¿Función característica, generadora de momentos, etc...?

% INTEGRACIÓN.
-
% Integración usual RS, Integral de Young. Conceptos de p-variación y $\alpha$-Hölder.

% Ecuaciones Diferenciales Ordinarias. EDO Controladas. ¿EDP?

% Teorema de Picard, Teorema de Peano.-

% ========================
% ========================
% ========================


% ========================
% ======= CAPÍTULO 2 =====
% ========================

\chapter{Movimiento Browniano}


En 1828, el botánico Sueco, \textit{Robert Brown}, observó que los granos de polen en un líquido se movian de forma irregular.... más contexto histórico.



% ========================
% ======= SECCIÓN 1 =====
% ========================

\section{Conceptos de Procesos Estocásticos}


% ========================
% ========================
% ========================






% ========================
% ======= SECCIÓN 2 =====
% ========================

\section{Movimiento Browniano}

Primero, se da la definición de un movimiento Browniano, y luego, se hará la construcción. Esta, se puede hacer por tres métodos distintas (Consulte \cite{Brownian_Motion_Karatzas}):

\begin{enumerate}
	\item Teoremas de existencia y continuidad de Kolmogorov.
	\item Construcción propuesta por Levý, Wiener y Ciesielski, que usa fuertemente conceptos de espacios de Hilbert, aprovechando que el movimiento Browniano es Gaussiano. Usando funciones de Haar.
	\item Principo de invarianza de Donsker, donde se busca una convergencia débil a una medida de Wiener.
\end{enumerate}

En el presente trabajo, se enunciarán los teoremas de Kolmogorov, y en el apéndice se discutirán brevemente las demás construcciones.

\begin{boxDef}
	Dado $\{ W_t \}$ un proceso estocástico, en el espacio de probabilidad $(\Omega, \mathcal{F}, P)$. El proceso $\{ W_t \}$ es un \textbf{movimiento Browniano} en una dimensión, si se cumplen las siguientes condiciones:

	\begin{itemize}
		\item Para casi todo $\omega$, los caminos $W_t (\omega)$ son continuos (En el sentido de la probabilidad).
		\item $\{ W_t \}$ es un proceso Gaussiano, es decir, para $k \geq 1$, y todo $0 \leq t_1 \leq \cdots \leq t_k$, el vector aleatorio, $Z = (W_{t_1}, \cdots, W_{t_k}) \in \mathbb{R}^{n}$ tiene distribución multinormal (O Gaussiana), con media el vector $0$, y la matriz de covarianza como $B(t_i, t_j) = \mathbb{E}[W_{t_i} W_{t_j}] = \min(t_i, t_j)$.
	\end{itemize}

\end{boxDef}

Más general, podemos considerar el movimiento Browniano respecto a una filtración.

\begin{boxDef}
	Un proceso $W_t$ en un espacio de probabilidad $(\Omega,\mathcal{F}, P)$ adaptado a una filtración $(\mathcal{F}_t)_{t > 0}$ es un \textbf{movimiento Browniano relativo a la filtración} $\mathcal{F}_t$, si:

	\begin{itemize}
		\item Los caminos $W_t(\omega)$ son trayectorias continuas de $t$ para casi todo $\omega$.
		\item $W_0 (\omega) = 0$ para casi todo $\omega$.
		\item Para $0 \leq s \leq t$, los incrementos $W_t - W_s$ son variables aleatorias Faussianas con media $0$ y varianza $t - s$.
		\item Para $0 \leq s \leq t$, los incrementos $W_T - W_s$ son independientes de las $\sigma$-álgebras $\mathcal{F}_s$.
	\end{itemize}

\end{boxDef}

Una pregunta que natural, es acerca de la existencia y de la unicidad de este proceso. Kolmogorov propone dos teoremas para verificar que el proceso existe y es único (En el sentido de distribuciones). Antes, se define el concepto de familia \textit{consistente}.

\begin{boxDef}
	Sea $T$ un conjunto de sucesiones finitas $\tilde{t} = (t_1, \cdots, t_n)$ de números positivos distintos. Suponga que para cada $\tilde{t}$ de longitud $n$, existe una medida de probabilidad $Q_{\tilde{t}}$ en $(\mathbb{R}, \mathcal{B}(\mathbb{R}^n))$. Luego, $\left\{ Q_{\tilde{t}} \right\}_{\tilde{t} \in T}$ es una \textbf{familia de distribuciones finito-dimensionales}. La familia se dice \textbf{consistente} si cumple las siguientes condiciones:

	\begin{itemize}
		\item Si $\tilde{s} = (t_{i_1}, \cdots, t_{i_n})$ es una permutación de $\tilde{t} = (t_1, \cdots, t_n)$, luego para todo $A_i \in \mathcal{B}(\mathbb{R})$ con $i = 1, \cdots, n$ tenemos:

		\[
			Q_{\tilde{t}} (A_1 \times \cdots \times A_n) = Q_{\tilde{s}} (A_{i_1} \times \cdots \times A_{i_n})
		\]

		esto es, es invariante bajo permutaciones.

		\item Si $\tilde{t} = (t_1, \cdots, t_n)$ con $n \geq 1$, y $\tilde{s} = (t_1, \cdots, t_{n-1})$, con $A \in \mathcal{B}(\mathbb{R}^{n-1})$, luego:

		\[
			Q_{\tilde{t}} (A \times \mathbb{R}) = Q_{\tilde{s}} (A)
		\]
	\end{itemize}

\end{boxDef}

Note que, al tener un espacio de probabilidad, $(\mathbb{R}^{[0, \infty)}, \mathcal{B}(\mathbb{R}^{[0, \infty)} ), P )$, se puede definir una familia de distribuciones finito-dimensionales:

\[
	Q_{\tilde{t}} (A) = P[ \left\{ w \in \mathbb{R}^{ [0,\infty) } \vert ( w(t_1), \cdots, w(t_n) ) \in A  \right\} ]
\]

Para construir el movimiento Browniano, se usará el hecho, de tener las distribuciones finito-dimensaionales, para construir una medida de probabilidad en el espacio, como primer paso. Para ello, se usa el siguiente teorema:


\begin{theorem}[Teorema de consistencia de Kolmogorov y Daniell.]
	Sea $\left\{ Q_{ \tilde{t} } \right\}$ una familia de distribuciones finito-dimensonales consistentes. Así, existe una medida de probabilidad $P$ en $(\mathbb{R}^{ [0, \infty) }, \mathcal{B}(\mathbb{R}^{ [0, \infty) })  )$ tal que:

	\[
		Q_{\tilde{t}} (A) = P[ \left\{ w \in \mathbb{R}^{ [0,\infty) } | (w(t_1), w(t_2), \cdots, w(t_n)) \right\} ]
	\] 	
	se cumpla para todo $\tilde{t} = \left\{ t_1, t_2, \cdots, t_n \right\} \in T$.
\end{theorem}

Como consecuencia, se tiene el siguiente resultado:

\begin{coro}
	Existe una medida de probabilidad $P$ en $(\mathbb{R}^{ [0, \infty) }, \mathcal{B}(\mathbb{R}^{ [0, \infty) })  )$ tal que el proceso: 

	\[
		B_t (w) = w(t) \text{, } w \in \mathbb{R}^{[0, \infty)}, t \geq 0	
	\]
	tiene incrementos estacionarios e independientes. Además, los incrementos $B_t - B_s$, con $0 \leq s \leq t$, es normalmente distribuido con media $0$ y varianza $t - s$,

\end{coro}

Ya se tiene una medida de probabilidad. Ahora, se debe construir el proceso en todo el espacio $\mathbb{R}^{[0, \infty)}$. Sin embargo, hay que notar que $C[0, \infty) \notin \mathcal{B}( \mathbb{R}^{ [0, \infty) } )$, y por ende, se debe construir una modificación continua del proceso. Se tiene entonces, el segundo teorema de Kolmogorov.

\begin{theorem}[Teorema de continuidad de Kolmogorov y Chenstov.]
	Suponga que un proceso $X = \left\{  X_t \vert 0 \leq t \leq T \right\}$ en un espacio de probabilidad $(\Omega, \mathcal{F}, P)$ cumple la condición:

	\[
		\mathbb{E}[ \lvert X_t - X_s \rvert^{\alpha} ] \leq C \lvert t - s \rvert^{ 1 + \beta}
	\]
	con $0 \leq s, t \leq T$, para constantes positivas $\alpha$, $\beta$ y $C$. Entonces, existe una modificación $\tilde{X} = \left\{ \tilde{X_t} \vert  0 \leq t \leq T  \right\}$ de $X$, que es localmente Hölder-continua, con exponente $\gamma$, para todo $\gamma \in (0, \beta / \alpha)$, esto es:

	\[
		P \left[ w \vert \sup_{  0 < t - s < h(w)\text{, } s,t\in [0,T]}  \frac{\lvert  \tilde{X_t}(w) - \tilde{X_s}(w) \rvert }{\lvert t - s \rvert^{\gamma}}  \leq \delta \right] = 1
	\]

	donde $h(w)$ es una variable aleatoria positiva \textit{casi siempre}m y $\delta > 0$ una constante apropiada.

\end{theorem}

Como resultado final, tendremos:

\begin{coro}
	Hay una medida de probabilidad $P$ en $(\mathbb{R}^{ [0, \infty) }, \mathcal{B}(\mathbb{R}^{ [0, \infty) })  )$ y un proceso estocástico $W = \left\{ W_t, \mathcal{F}_t^W \vert t \leq 0 \right\}$ en el mismo espacio, tal que $W$ es un movimiento Browniano respecto a $P$.
\end{coro}

Como consecuencia, cada trayectoria del movimiento Browniano $\left\{ W_t(w) \vert 0 \leq t < \infty \right\}$ es localmente Hölder-continua con exponente $\gamma$, para $\gamma \in \left(0, \frac{1}{2}\right)$.




% ========================
% ========================
% ========================


% ========================
% ========================
% ========================


% ========================
% ======= CAPÍTULO 3 =====
% ========================

\chapter{La Integral de Itô}

En este capítulo, comenzaremos la construcción de la integral de Itô. Para cumplir este objetivo, usaremos fuertemente los hechos vistos en el capítulo anterior del movimiento Browniano. 

Ahora, ¿Por qué es necesario construir una nueva integral? Veamos el objetivo inicial, solucionar una ecuación diferencial que tiene cierto ruido:

\[
	\frac{dX}{dt} = b(t, X_t) + \sigma (t, X_t) \cdot W_t
\]

Note que el ruido se puede representar como el proceso estocástico $W_t$. Bajo experimentación, se interponen las siguientes condiciones sobre el ruido:

\begin{itemize}
	\item Dos variables del proceso $W_{t_1}$ y $W_{t_2}$ con $t_1 \neq t_2$ son independientes.
	\item $\{ W_t \}$ es un proceso estacionario.
	\item $\mathbb{E}[W_t] = 0$ para todo $t$.
\end{itemize} 

No hay algún proceso estocástico tradicional que cumpla las condiciones dadas. Por ende, lo podemos ver como un proceso estocástico generalizado, un \textbf{proceso de ruido blanco}, esto es, un proceso que se puede construir como medida de probabilidad en cierto espacio sútil de funcionales $\mathcal{C}[ 0, \infty )$. 

Por ende, se nos sugiere que el proceso $\{W_t\}$ será el movimiento Browniano. Discretizando la ecuación inicial...

\[
	\int_0^t f(s,w) dB_s(w)
\]

%  (Oksendal 3.11, Kallianpur 1980 p.10

% ========================
% ========================
% ========================



\end{document}